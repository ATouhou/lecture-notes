% !TEX program = pdflatex

\documentclass{lnotes}

% Packages
\usepackage[osf]{libertineRoman}
\usepackage{epigraph}

% Metadata
\title{Fields and Galois Theory}
\course{\textsc{math} 370}
\place{yale university}
\term{spring}
\year{2019}
\blurb{
	These are lecture notes for \textsc{math} 370b, ``Fields and Galois Theory,'' taught by Asher Auel at Yale University during the spring of 2019.
	These notes are not official, and have not been proofread by the instructor for the course.
	They live in my lecture notes respository at 
	\[\text{\url{https://github.com/jopetty/lecture-notes/tree/master/MATH-370}.}\] 
	If you find any errors, please open a bug report describing the error and label it with the course identifier, or open a pull request so I can correct it.
}

% Bibliography
\usepackage[
	style = alphabetic,
]{biblatex}
\addbibresource{references.bib}

\begin{document}

\section*{Syllabus}

\begin{tabularx}{\textwidth}{rX}
\toprule
\textbf{Instructor} & Asher Auel, \url{asher.auel@yale.edu} \\
\textbf{Lecture} & \textsc{tr} 11:35 \textsc{am} -- 12:50 \textsc{pm} in \textsc{lom} 215 \\
\textbf{Peer Tutor} & Arthur Azvolinsky, \url{arthur.azvolinsky@yale.edu} \\ 
\textbf{Section} & Math Lounge \\
\textbf{Exams} & Midterm 1: Febuary 19; \quad Midterm 2: April 9; \quad Final: May 3. \\
\textbf{Textbook} & \fullcite{textbook} \\
\bottomrule
\end{tabularx} \\

The main object of study in Galois theory are roots of single variable polynomials.
Many ancient civilizations (Babylonian, Egyptian, Greek, Chinese, Indian, Persian) knew about the importance of solving quadratic equations.
Today, most middle schoolers memorize the ``quadratic formula'' by heart.
While various incomplete methods for solving cubic equations were developed in the ancient world, a general ``cubic formula'' (as well as a ``quartic formula'') was not known until the 16th century Italian school.
It was conjectured by Gauss, and nearly proven by Ruffini, and then finally by Abel, that the roots of the general quintic polynomial could not be solvable in terms of nested roots.
Galois theory provides a satisfactory explanation for this, as well as to the unsolvability (proved independently in the 19th century) of several classical problems concerning compass and straight-edge constructions (e.g., trisecting the angle, doubling the cube, squaring the circle).
More generally, Galois theory is all about symmetries of the roots of polynomials.
An essential concept is the field extension generated by the roots of a polynomial.
The philosophy of Galois theory has also impacted other branches of higher mathematics (Lie groups, topology, number theory, algebraic geometry, differential equations).

This course will provide a rigorous proof-based modern treatment of the main results of field theory and Galois theory.
The main topics covered will be irreducibility of polynomials, Gauss's lemma, field extensions, minimal polynomials, separability, field automorphisms, Galois groups and correspondence, constructions with ruler and straight-edge, theory of finite fields.
The grading in Math 370 is very focused on precision and correct details.
Problem sets will consist of a mix of computational and proof-based problems.

Your final grade for the course will be determined by
\[ \max\left\{
	\begin{array}{cccc}
		\text{20\% homework} + \text{25\% midterm 1} + \text{25\% midterm 2} + \text{30\% final} \\
		\text{20\% homework} + \text{25\% midterm 1} + \text{15\% midterm 2} + \text{40\% final} \\
		\text{20\% homework} + \text{15\% midterm 1} + \text{25\% midterm 2} + \text{40\% final}
	\end{array}
\right\}. \]

\printbibliography

%% Sanskrit
\section{August 30, 2018}

Linguistics was the model science in ancient india

Today, we'll talk about phonology --- range of sounds, how they're identified.
Phonology is the sound structure of a language, as opposed to syntax or semantics. For the time being we'll transliterate Sanskrit into the Latin script; tomorrow, we'll move on to devanagari. Sanskrit has been written in a variety of scripts, but devanagari is the most commonly used.
Much of the historic writings in Sanskrit are written in Devanagari or one of its predecessors.

Sanskrit's phonological system is very ``scientific'' --- it's very systematic and structured. Grammarians divide sounds in Sanskrit into two groups: the \emph{svara}, or vowels; and the \emph{vyañjana}, or consonants.

\subsection*{Svara}
In Sanskrit, svara (meaning sound, tone, or accent) are sounds which can be pronounced independently, contrasting with consonants which have an implied vowel associated with them.
There are 14 vowels in Sanskrit (really only 13) which are organized by two orthogonal qualities: the place of articulation and the length.

\begin{wrapfigure}{l}{0.25\textwidth}
\begin{tabular}{@{}cc@{}}
\toprule
Hrasva & Dīrga \\
\midrule
a & ā \\
i & ī \\
u & ū \\
ṛ & ṝ \\
ḷ & \emph{ḹ}\footnote{This vowel is never attested, and only exists to complete the chart} \\
\midrule
e & ai \\
o & au \\
\bottomrule
\end{tabular}
\end{wrapfigure}

Long and short vowels are distinguished by the number of \emph{mātrā} they consume.
You can think of mātrā as a kind of tempo marking for the language; short vowels, or \emph{hrvasa}, last for ``one beat'' of time, while long vowels, or \emph{dīrga}, last for two. In ancienct Vedic Sanskrit, there was a third category of vowels, called protracted vowels or \emph{pluta}, which lasted for three beats, but these were lost in Classical Sanskrit.
Vowels can also be simple (\emph{śuddha}, or ``pure''), or complex (\emph{samyukta}, or ``joined''). Samyukta are known in linguistics as diphthongs, and are made from the combination of two or more śuddha. In the table on the left, the samyukta are the bottom four.

Here are some examples of the vowels in Sanskrit words:

\begin{figure}[h]
\begin{center}
\begin{tabular}{@{}ll@{}}
\toprule
Abhava (absense, destruction) & Ātman (self) \\
Idam (this) & Īśvara (lord) \\
Ubhaya (both) & Abhūt (was) \\
Kṛṣṇa (Krishna) & Nṝṇām\footnote{The consonant ṝ is very rare, appearing only in nominative and genetive nouns} (Of men) \\
Kḷpti (arrangement) & --- \\
Etad (this) & Bhojana (food) \\
Chaitanya (a saint) & Saubhagya (good fortune) \\
\bottomrule
\end{tabular}
\end{center}
\end{figure}

\subsection*{Consonants}

 There are three kinds of consonants in SKRT. Consontants are classified by the place of articulation (sthāna):
 	- Velar or kan.t.hya
 	- soft pallette, palatal (talva)
 	- hard pallatte, retroflex (murtanyo)
 	- dental (danta)
 	- labial (os.t.ha)

 Also classified by the closure of the lips
 	- full contact, stops (sparśa)

 Voiced/Unvoiced
 	- ghośavat (voiced)
 	- aghośa 	(unvoiced)
 
 Apsiration
 	- in english, spin vs pin, cut vs king
 	- Alpha prana 	(unaspirated)
 	- maha prana	(aspirated)



\end{document}