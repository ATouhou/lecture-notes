% !TEX program = pdflatex

\documentclass{lnotes}

% Packages
\usepackage[osf]{libertineRoman}

% Metadata
\title{Fields and Galois Theory}
\course{\textsc{math} 370}
\place{yale university}
\term{spring}
\year{2019}
\blurb{
	These are lecture notes for \textsc{math} 370b, ``Fields and Galois Theory,'' taught by Asher Auel at Yale University during the spring of 2019.
	These notes are not official, and have not been proofread by the instructor for the course.
	These notes live in my lecture notes respository at 
	\[\text{\url{https://github.com/jopetty/lecture-notes/tree/master/MATH-370}.}\] 
	If you find any errors, please open a bug report describing the error, and label it with the course identifier, or open a pull request so I can correct it.
}

% Bibliography
\usepackage[
	style = alphabetic,
]{biblatex}
\addbibresource{references.bib}

\begin{document}

\section*{Syllabus}

\begin{tabularx}{\textwidth}{rX}
\toprule
\textbf{Instructor} & Asher Auel, \url{asher.auel@yale.edu} \\
\textbf{TF} & --- \\ 
\textbf{Lecture} & \textsc{tr} 11:35 \textsc{am} -- 12:50 \textsc{pm} in \textsc{lom} 215 \\
\textbf{Section} & --- \\
\textbf{Exams} & --- \\
\textbf{Textbook} & \fullcite{textbook} \\
\bottomrule
\end{tabularx} \\

The main object of study in Galois theory are roots of single variable polynomials.
Many ancient civilizations (Babylonian, Egyptian, Greek, Chinese, Indian, Persian) knew about the importance of solving quadratic equations.
Today, most middle schoolers memorize the ``quadratic formula'' by heart.
While various incomplete methods for solving cubic equations were developed in the ancient world, a general ``cubic formula'' (as well as a ``quartic formula'') was not known until the 16th century Italian school.
It was conjectured by Gauss, and nearly proven by Ruffini, and then finally by Abel, that the roots of the general quintic polynomial could not be solvable in terms of nested roots.
Galois theory provides a satisfactory explanation for this, as well as to the unsolvability (proved independently in the 19th century) of several classical problems concerning compass and straight-edge constructions (e.g., trisecting the angle, doubling the cube, squaring the circle).
More generally, Galois theory is all about symmetries of the roots of polynomials.
An essential concept is the field extension generated by the roots of a polynomial.
The philosophy of Galois theory has also impacted other branches of higher mathematics (Lie groups, topology, number theory, algebraic geometry, differential equations).

This course will provide a rigorous proof-based modern treatment of the main results of field theory and Galois theory.
The main topics covered will be irreducibility of polynomials, Gauss's lemma, field extensions, minimal polynomials, separability, field automorphisms, Galois groups and correspondence, constructions with ruler and straight-edge, theory of finite fields.
The grading in Math 370 is very focused on precision and correct details.
Problem sets will consist of a mix of computational and proof-based problems.

Your final grade for the course will be determined by
\[ \max\left\{
	\begin{array}{cccc}
		\text{20\% homework} + \text{25\% midterm 1} + \text{25\% midterm 2} + \text{30\% final} \\
		\text{20\% homework} + \text{25\% midterm 1} + \text{15\% midterm 2} + \text{40\% final} \\
		\text{20\% homework} + \text{15\% midterm 1} + \text{25\% midterm 2} + \text{40\% final}
	\end{array}
\right\}. \]

\printbibliography

\end{document}