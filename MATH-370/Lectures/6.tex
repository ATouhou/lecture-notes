% !TEX root = ../notes.tex

\section{Thursday, January 31}

Recall from last lecture: We have a field $F \subseteq \Omega$ and some $\alpha \in \Omega$ with an evaluation homomoprhism $F[x] \to \Omega$ where $x \mapsto \alpha$ and the image is $F[\alpha]$. Then either this is injective, which means that $\alpha$ is trancendental, or we have a nontrivial kernel equal to $(m_\alpha(x)) \subseteq F[x]$ where $m_\alpha(x)$ is the minimal polynomial of $\alpha$ over $F$.

Recall also from \textsc{math 350} that if $R$ is a \textsc{pid} but not a field and $I \subseteq R$ is a prime ideal then $I \subseteq R$ is a maximal ideal.

\begin{corollary}
If $\alpha$ is algebraic over $F$ then $F[x]/(m_\alpha(x))$ is a field which is isomorphic to $F[\alpha]$ by the first isomorphism theorem.
\end{corollary}

\begin{proof}
Recall that the minimal polynomial is irreducible, and in a \textsc{pid} irreducibility implies that it is prime. Then $(m_\alpha)$ is a prime ideal and so maximal. Then $F[x]/(m_\alpha)$ is a field.
\end{proof}

\begin{corollary}
If $\alpha$ is algebraic over $F$ then $F[\alpha] = F(\alpha)$.
\end{corollary}

\begin{proof}
By definition $F[\alpha] \subseteq F(\alpha)$. Since $F[\alpha]$ is a field containing $F,\alpha$ and $F(\alpha)$ is the smallest extension of $F$ containing $\alpha$ then $F(\alpha) \subseteq F[\alpha]$, and so they are equal by dual containment.
\end{proof}

\begin{theorem}
Let $f(X) \in F[x]$ be irreducible of degree $n$. Then $F[x]/(f)$ is a field extension of $F$ of degree $n$ with an $F$-basis $\{\bar{1}, \bar{x}, \bar{x}^2, \dotsc, \bar{x}^{n-1}\}$ where $\bar{x}$ is the coset in $F[x]/(f)$ represented by $x$.
\end{theorem}

\begin{proof}
By the corollaries, $K = F[x]/(f)$ is a field extension of $F$. Let $f(x) = a_0 + a_1x + \cdots + a_nx^n$. We claim then taht $1, \bar{x}, \dotsc, \bar{x}^{n-1}$ is generates $K$ over $F$. We know that $K$ is generated by the infinite set $\bar{1}, \bar{x}, \bar{x}^2,\dotsc$. However we see that $\bar{x}^n = 1\frac{1}{a_n}(a_0 + a_1\bar{x} + \cdots + a_{n-1}\bar{x}^{n-1})$, and so on for $\bar{x}^{k}$ for $k \geq n$. All of these expressions are linear combinations of $\bar{1}, \bar{x}, \dotsc, \bar{x}^{n-1}$. Then we must show that these are all linearly independent. Suppose that 
\[b + b_1\bar{x} + \cdots + b_{n-1}\bar{x}^{n-1} = 0. \tag{\text{\star}}  \]
Notice that $\bar{x} \in K$ satisfies $f(\bar{x}) = 0$ and $f$ is irreducible ofer $F$. Then $\frac{1}{a_n}f(x)$ is the minimal polynomial of $\bar{x} \in K$. Then no polynomial of smaller degree can have $\bar{x}$ as a root, yet we just found $\star$ to have $\bar{x}$ as a root, and so $b_i = 0$ for all $i$ and this set is linearly independent.
\end{proof}

\begin{corollary}
Let $\alpha, \beta \in \Omega$ have the same minimal polynomial over $F$. Then $F(\alpha) \cong F(\beta)$.
\end{corollary}

\begin{definition}
$F(\alpha_1, \dotsc, \alpha_n)$ is a field extension of $F$ which is finitely generated.
\end{definition}

\begin{theorem}
Let $F$ be a field.
\begin{enumerate}
\item Any finite extension $k/F$ is algebraic, where every elemetn $\alpha \in k$ is algebraic over $F$.
\item Any finitely-generated and algebraic extension $k/F$ is finite.
\end{enumerate}
\end{theorem}