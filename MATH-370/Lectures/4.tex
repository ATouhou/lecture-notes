% !TEX root = ../notes.tex

\section{Thursday, January 24}

\begin{definition}[Extension]
Let $k$ be a field. Then $F \subseteq k$ is a subfield if $F \subseteq k$ is a unital subring and $F$ is a field.
We say that $k$ is an extension of $F$ and write $k/F$.
\end{definition}

\begin{example}
If $k$ is an extension of $F$ then $k$ has the structure of an $F$-vector space. The cannonical example for this is thinking of $\C$ as a two-dimensional $\R$-vector space with basis $\{1, i\}$.
\end{example}

\begin{definition}
An extension $k/F$ is finite if $k$ is a finite dimensional $F$-vector space. The degree of $k/F$, written as $[k : F] = \dim_Fk$ is the $F$ dimension of $k$.
\end{definition}

\begin{example}[Examples of degrees]
\begin{enumerate}
\item $\C/\R$ is finite and $[\C : \R] = 2$.
\item $\R/\Q$ is infinite since $n$-tuples of $\Q$ are countable and $\R$ is uncountable.
\item Let $\Q(i) = \{x + iy \in \C \mid x,y \in \Q\}$. Then $\Q(i)/\Q$ is finite with degree $2$.
\end{enumerate}
\end{example}

\subsection{Field extensions generated by elements}

Let $F$ be a field contained in some field $\Omega$. For $\alpha_1, \dotsc, \alpha_n \in \Omega$ we can consider two objects:
\begin{enumerate}
\item $F[\alpha_1, \dotsc, \alpha_n] \subseteq \Omega$ is a subring
\item $F(\alpha_1, \dotsc, \alpha_n) \subseteq \Omega$ is a subfield
\end{enumerate}

We define these equantities in the following way. 
\[ F[\alpha_1, \dotsc, \alpha_n] = \bigcap_{\substack{R \subseteq \Omega \\ F \subseteq R, \alpha_i \in R}} R = \left\{\sum a_{i_1} \cdots a_{i_n}\alpha_1^{i_1} \cdots \alpha_n^{i_n} \mid a_{i_1,\dotsc,i_n} \in F\right\} \]
and 
\[ 
	F(\alpha_1, \dotsc, \alpha_n) = \bigcap_{\substack{k \subseteq \Omega \\ F \subseteq k, \alpha_i \in k}} k = 
	\left\{
		\frac{\alpha}{\beta} \mid \alpha,\beta \in F[\alpha_1, \dotsc, \alpha_n], \beta \not= 0
	\right\}.
\]
We say that $F[\alpha_1, \dotsc, \alpha_n]$ is the subring of $\Omega$ generated by $\alpha_1, \dotsc, \alpha_n$ and $F(\alpha_1, \dotsc, \alpha_n)$ is the subfield of $\Omega$ generated by $\alpha_1, \dotsc, \alpha_n$.

\begin{example}
Consider $\Q[i]$ and $\Q(i)$. We say that $\Q[i]$ are rational polynomials in $i$ and $\Q(i)$ are quotients of these polynomials, understanding that even powers of $i$ and odd powers of $i$ to collapse it into rational and imaginary components. These are equal to one another. However, $\Q[\pi] \subset \Q(\pi)$ but they are not equal (since the field extension is not finite as $\pi$ is not algebraic).
\end{example}

The Laurent series $F((x))$ is important for Complex Analysis and is analogous to the formal power series $R[[x]]$.

\begin{theorem}[Tower Law]
Let $L/k$ and $k/F$ be field extensions. Then $[L : F] = [L : k] \cdot [k : F]$.
\end{theorem}

\begin{proof}
If $L$ is a finite dimensional $F$-vector space then $L$ is a finite dimensional $k$-vector space since if $z_1,\dotsc,z_p$ is an $F$-basis for $L$ then in turn $\alpha = \sum a_iz_i \in F $ for all $\alpha \in L$ so $z_1,\dotsc,z_p$ is a generating set. On the other hand, $k \subseteq L$ is an $F$-vector space. Assume that $[L : k] = n$ with $x_i$ a $k$-basis for $L$ and $[k : F] = m$ with $y_i$ and $F$-basis for $k$. Then we first show that $\{x_iy_j\}$ is an $F$-basis of $L$ so $[L:F] = nm$. First, linear independence: assume that $\sum a_{i,j}x_iy_j = 0$ for $a_{i,j} \in F$. Rewrite this as $\sum\qty(\sum a_{i,j}y_i)x_i = 0$, so $\sum a_{i,j}y_i = 0$ so $a_{i,j} = 0$ and the set is linearly independent. Next, we show this generates $L$ over $F$. Let $\alpha \in L$ be written as $\alpha = \sum \beta_ix_i$ for $\beta_i \in k$. But for each $i$ we have $\beta_i = \sum a_{i,j}y_i$ so $\alpha = \sum\qty(\sum a_{i,j}y_i)x_i = \sum a_{i,j}x_iy_i$, so these generate. Then this is a basis, and the proposition holds.
\end{proof}