% !TEX root = ../notes.tex

\section{Thursday, January 24}

\epigraph{``Ice cream is valuable because of the pleasurable sensations.''}{Shelly}

\epigraph{``I just don't know what to say about masochists.''}{Shelly}

\subsection{What is the correct theory of the good?}

Generally we want to have a theory which is impartial and considers everyone equally. Sometimes we don't want to single out some people over others. But that still doesn't tell us how to pick one set of results over another.

\begin{proposition}
Let's measure how well-off people are after the different results. The modest way says that this is a relevant factor, the bolder way says it is the only thing thats relevant.
\end{proposition}

\begin{definition}[Welfarism]
The view that people's welfare is the only relevant criterion in determining which actions are morally better than another.
\end{definition}

This still doesn't give us a good account of well-being, and it doesn't tell us how to aggregate individual well-being to understand how good or bad an outcome is as a whole. We have to answer both of these questions. Should it be based on material or personal goods, like money or vacations or respect or sex or one's books to be published? In turn we need to understand why each of these good (or bad) things is valuable in its own right. It seems like the root of all these things is ``pleasure'' --- everything else is an instrument to acquire pleasure.

\begin{problem}
Can pleasure be both inherently good and instrumentally good? Yes, cf.\ reproduction of genes through pleasurable activities.
\end{problem}

\begin{example}
The pain you get at the dentist is intrinsically bad (because pain is bad) but instrumentally good (since it helps you avoid further pain and live a healthier life).
\end{example}

\begin{problem}
Can something be both intrinsically good and intrinsically bad? What about masochism?
\end{problem}

\begin{problem}
How do we know if our list of inherently good and bad things complete? Are all pleasures equally good? Are all pains equally bad?
\end{problem}

For this first question, Welfarism contends that pleasure and pain are the only good and bad things on that list.

\begin{definition}[Hedonism]
The view that the value of a life is the sum difference of pleasures and pains.
\end{definition}

\begin{example}[Painless Death]
If you life is more good than bad, death would probably be bad. If your life is worse than it is good, death might be good. If you believe that the afterlife is amazing then death wouldn't really be a problem.
\end{example}

\begin{problem}
Do Hedonists consider the pleasure of other people?
\end{problem}

\begin{definition}[Quantitative Hedonists]
A hedonist who wants to measure the amount of pain or pleasure in an experience to rank them.
\end{definition}