% !TEX root = ../notes.tex

\section{Tuesday, January 29}

\epigraph{``I've never been a pig.''}{Shelly}

Mill differentiates himself from Bentham by accepting that the quality of pleasures is important, even moreso than the quantity of the pleasures. This leads to a hierarchy of pleasures which we should prefer. The social and intellectual and spiritual and aesthetic pleasures are `higher' than the more base pleasures. In order to distinguish which are better than others, Mill proposes that if we ask those who understand and experience both and a decided majority support one over the other then that is the higher good.

\begin{problem}
How do we actually know that this is the answer people would give? Mill seems to priveledge these particular `highbrow pleasures' but then \emph{justify} that choice by saying that `people will naturally choose that.' But what if people don't choose that --- would Mill still hold those higher virtues above the lower ones?
\end{problem}

\begin{problem}
If we accepted Qualitative Hedonism, wouldn't that imply that we should force people to do these higher pleasures over the lower ones?

\begin{solution}[Response]
Right now we're discussing a theory of well-being, not a theory of what one ought to do. We might accept that someone might be happier if we forced them to do certain things, but we don't say that we must force people to be happer, only recognize that this is something which would make them happier. Furthermore, \emph{how} we experience the world is just as important as the thing we percieve; if we don't like classical music then we aren't experiencing any pleasure regardless of the supposed benefit of seeing it.
\end{solution}
\end{problem}

\begin{proposition}
In order to understand and appreciate the higher virtues, we must train them to appreciate them through education.
\end{proposition}

\begin{problem}
Five-year-olds seem to be a lot happier than adults. 

\begin{solution}[Response]
Would you really choose to be a five-year-old for the rest of your life? Most people (I think) wouldn't.
\end{solution}
\end{problem}

\begin{problem}
Is getting the right mental state all that matters? Imagine that scientists can stimulate your brain directly to create experiences identical to those you would have while doing anything else. For example, you could fully experience what it would feel like to climb mount everest while inside the machine the whole time. But while you are in this machine, you don't know that you are in the machine since this knowledge would change your perspective. Do you want to be hooked up to the machine?

\begin{solution}[Response (Personal)]
I don't think so.
\end{solution}
\end{problem}
