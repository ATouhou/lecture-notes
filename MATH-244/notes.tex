\documentclass{notes}

% Biblography
\usepackage{biblatex}
\addbibresource{references.bib}

\title{Discrete Mathematics}
\courseid{MATH 244}
\place{Yale University}
\term{Fall}
\year{2018}

\blurb{
	These are lecture notes for MATH 244a, ``Discrete Mathematics,'' taught by Ross Berkowitz at Yale University during the fall of 2018.
	These notes are not official, and have not been proofread by the instructor for the course.
	These notes live in my lecture notes respository at 
	\[\text{\url{https://github.com/jopetty/lecture-notes/tree/master/MATH-244}.}\] 
	If you find any errors, please open a bug report describing the error, and label it with the course identifier, or open a pull request so I can correct it.
}

\begin{document}

\section*{Syllabus}

\begin{center}
\begin{tabular}{@{}rp{10cm}@{}}
\toprule
\textbf{Instructor} & Ross Berkowitz, \url{ross.berkowitz@yale.edu} \\
\textbf{Lecture} & 11:35 \textsc{am} -- 12:35 \textsc{pm}, WLH 201 \\
\textbf{Textbook} & \fullcite{JJ} \\
\textbf{Final} & Sunday, December 16, 2018, 2:00 \textsc{pm} \\
\bottomrule
\end{tabular} \\[3ex]
\end{center}

Discrete  math  is  the  study  of  discrete,  and  frequently  finite,  mathematical  structures.   It  is  a broad subject, and is perhaps best understood by seeing the problems it considers.  In this course the three main topics we will cover are enumerative combinatorics, graph theory, and probability. Enumerative combinatorics is,  narrowly speaking,  the art of counting and estimating the size of various structures, sequences, or sets.  Graphs encode information about pairwise relations between objects (consider a network of people with a connecting line between them if they are friends), and their properties form a rich subject of study.  Finally, probability theory, the study of chances, both deals with finite objects directly, and is a surprisingly useful tool for studying discrete structures of all stripes. Just  as  important  as  the  material  itself,  this  course  will  serve  as  a  primer  in  mathematical thinking. We will see important problem solving methods and learn how to make rigorous arguments by seeing and by doing. We will break the course down into five sections,
\begin{itemize}
\item \textbf{Preperatory Material:}  Mathematical notation, functions, induction, proofs;
\item \textbf{Enumeration:}  The binomial theorem, permutations, stars and bars, estimates, inclusion exclusion;
\item \textbf{Graphs:} Eulerian graphs, connectivity, Turan’s theorem, trees, planar graphs, graph coloring;
\item \textbf{Generating Functions:}  Recurrence relations, rational generating functions, algebraic manipulation;
\item \textbf{Probability:}  linearity of expectation, first moment method.
\end{itemize}

\printbibliography

%% Sanskrit
\section{August 30, 2018}

Linguistics was the model science in ancient india

Today, we'll talk about phonology --- range of sounds, how they're identified.
Phonology is the sound structure of a language, as opposed to syntax or semantics. For the time being we'll transliterate Sanskrit into the Latin script; tomorrow, we'll move on to devanagari. Sanskrit has been written in a variety of scripts, but devanagari is the most commonly used.
Much of the historic writings in Sanskrit are written in Devanagari or one of its predecessors.

Sanskrit's phonological system is very ``scientific'' --- it's very systematic and structured. Grammarians divide sounds in Sanskrit into two groups: the \emph{svara}, or vowels; and the \emph{vyañjana}, or consonants.

\subsection*{Svara}
In Sanskrit, svara (meaning sound, tone, or accent) are sounds which can be pronounced independently, contrasting with consonants which have an implied vowel associated with them.
There are 14 vowels in Sanskrit (really only 13) which are organized by two orthogonal qualities: the place of articulation and the length.

\begin{wrapfigure}{l}{0.25\textwidth}
\begin{tabular}{@{}cc@{}}
\toprule
Hrasva & Dīrga \\
\midrule
a & ā \\
i & ī \\
u & ū \\
ṛ & ṝ \\
ḷ & \emph{ḹ}\footnote{This vowel is never attested, and only exists to complete the chart} \\
\midrule
e & ai \\
o & au \\
\bottomrule
\end{tabular}
\end{wrapfigure}

Long and short vowels are distinguished by the number of \emph{mātrā} they consume.
You can think of mātrā as a kind of tempo marking for the language; short vowels, or \emph{hrvasa}, last for ``one beat'' of time, while long vowels, or \emph{dīrga}, last for two. In ancienct Vedic Sanskrit, there was a third category of vowels, called protracted vowels or \emph{pluta}, which lasted for three beats, but these were lost in Classical Sanskrit.
Vowels can also be simple (\emph{śuddha}, or ``pure''), or complex (\emph{samyukta}, or ``joined''). Samyukta are known in linguistics as diphthongs, and are made from the combination of two or more śuddha. In the table on the left, the samyukta are the bottom four.

Here are some examples of the vowels in Sanskrit words:

\begin{figure}[h]
\begin{center}
\begin{tabular}{@{}ll@{}}
\toprule
Abhava (absense, destruction) & Ātman (self) \\
Idam (this) & Īśvara (lord) \\
Ubhaya (both) & Abhūt (was) \\
Kṛṣṇa (Krishna) & Nṝṇām\footnote{The consonant ṝ is very rare, appearing only in nominative and genetive nouns} (Of men) \\
Kḷpti (arrangement) & --- \\
Etad (this) & Bhojana (food) \\
Chaitanya (a saint) & Saubhagya (good fortune) \\
\bottomrule
\end{tabular}
\end{center}
\end{figure}

\subsection*{Consonants}

 There are three kinds of consonants in SKRT. Consontants are classified by the place of articulation (sthāna):
 	- Velar or kan.t.hya
 	- soft pallette, palatal (talva)
 	- hard pallatte, retroflex (murtanyo)
 	- dental (danta)
 	- labial (os.t.ha)

 Also classified by the closure of the lips
 	- full contact, stops (sparśa)

 Voiced/Unvoiced
 	- ghośavat (voiced)
 	- aghośa 	(unvoiced)
 
 Apsiration
 	- in english, spin vs pin, cut vs king
 	- Alpha prana 	(unaspirated)
 	- maha prana	(aspirated)



\end{document}