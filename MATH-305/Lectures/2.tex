% !TEX root = ../notes.tex

\section{January 16, 2019}

Recall from last lecture the definitions of open and closed sets.

\begin{proposition}
The union of finitely many closed sets is closed. The intersection of arbitrarily many closed sets is closed.
\end{proposition}

Note that $\qty(\bigcup S_i)^\complement = \bigcap S_i^\complement$ and $\qty(\bigcap S_i)^\complement = \bigcup S_i^\complement$.

\begin{proof}
By the above property, we konw that the union of finitely many closed sets is the complement of the intersection of finitely many open sets, which is open. We also know that the intersection of closed sets is the complement of the union of their complements, which are open, so the second part of the proposition is true as well.
\end{proof}

\begin{example}[Counterexamples to illustrate the conditions]
\begin{enumerate}
\item In $\R$, we know that $\bigcup\, \qty[\frac{1}{n}, 1-\frac{1}{n}]$ for $n \geq 2 = (0,1)$. This demonstrates the necessity of ``finiteness'' in our proposition.
\end{enumerate}
\end{example}

\begin{definition}[Closure]
For $A \subset X$ the closure of $A$ is the set of all limit points of $A$, usually written as $\bar{A}$.
\end{definition}

\begin{proposition}
For any $A \subset X$, we know that $\bar{A}$ is closed and in fact is the smallest closed set containing $A$.
\end{proposition}

\begin{proof}[Proof that $\bar{A}$ is closed]
We prove $\bar{A}$ is closed by showing that $X \setminus \bar{A}$ is open. For all $x \in X \setminus \bar{A}$ we know that $x$ is not a limit point of $A$ since otherwise it would be an element of $\bar{A}$. Then there exists some $r > 0$ such that $B(x,r) \subset X \setminus A$. Since $B(x,r)$ is open we know in fact that $B(x,r) \subset X \setminus \bar{A}$. Then $X \setminus \bar{A}$ is open, so we know that $\bar{A}$ is closed.
\end{proof}

\begin{proof}[Proof that $\bar{A}$ is the smallest closed set containing $A$]
Let $B$ be a closed subset containing $A$. We want to show that $\bar{A} \subset B$. If $x \in \bar{A}$ then $x$ is a limit point of $A$ and so $x$ is a limit point of $B$ as well since $B$ is closed and contains $A$. Since $B$ is closed, $B$ must contain all limit points, so $x \in B$ as well. Then $\bar{A} \subset B$.
\end{proof}

\begin{corollary}
A set $A \subset X$ is closed if and only if $A = \bar{A}$ and it is open if and only if $A = A^\circ$.
\end{corollary}

\begin{proposition}
For $x \in X$ and $r > 0$, consider the set $B[x,r] = \{y \in X \mid d(x,y) \leq r\}$ is closed. This is called the closed ball of radius $r$ centered at $x$.
\end{proposition}

\begin{proof}
We show that the complement of $B[x,r]$ is open. Let $z \in B[x,r]^\complement$. Then $d(x,z) > r$. Let $r_0 = d(x,z) - r$. Then $B(z,r_0)$ does not intersect $B[x,r]$. If this were false, then there would exist a $y$ such that $d(y,z) < r_0$ and $d(x,y) \leq r$ which would mean that $d(x,z) < r_0 + r = d(x,z)$ which is a contradiction.
\end{proof}

\subsection{Compact Sets}

\begin{definition}[Covering]
Let $B \subset X$ be nonempty. A collection $\mathcal{U} = \{U_\alpha \mid \alpha \in I\}$ of subsets of $X$ is called a cover(ing) if $B \subset \bigcup U_\alpha$. If every $U_\alpha$ is open then this $\mathcal{U}$ is an \emph{open cover} of $B$. A subcollection $\mathcal{V} = \{U_\alpha \mid \alpha \in J\}$ and $J \subset I$ is called a subcover of $\mathcal{U}$ if $B \subset \bigcup U_\alpha$ where $\alpha \in J$.
\end{definition}

\begin{definition}[Compact set]
A nonempty set $B$ is compact if every open cover of $B$ admits a finite subcover, so if $B \subset \mathcal{U}$ where $\mathcal{U}$ is open then there exists some finite collection $\mathcal{V}$ which still contains $B$.
\end{definition}

\begin{example}
\begin{enumerate}
\item $\Z \subset \R$ is not compact since $\bigcup B(n, 1/2)$ for all $n \in \Z$ is an open cover of $\Z$ which does not admit any finite subcover.

\item $\R$ is not compact. Consider $B(n,1)$ for all $z \in \Z$. This is an open cover of $\R$ which does not admit any finite subcover.

\item $(0,1]$ is not compact. Consider the cover created by $\bigcup \qty(1/n, \infty) = (0,\infty)$. This is an open cover of $(0,1]$ but no finite subset of this will contain $(0,1]$ so it admits no finite subcover.
\end{enumerate}
\end{example}

\begin{definition}[Bounded]
A set $A \subset X$ is bounded if there exists some $r > 0$ such that $A \subset B(x,r)$ for some $x \in X$.
\end{definition}

\begin{proposition}
Any compact subset of a metric space is closed and bounded.
\end{proposition}

\begin{proof}
Let $B$ be a compact subset of $X$. First we will show that $B$ is closed, or equivalently that $B^\complement$ is open. Let $x \in B^\complement$. Let $U_n = \{y \in X \mid d(x,y) > \frac{1}{n}\}$. This is the set $B\qty[x,\frac{1}{n}]^\complement$, which is open since it's the complement of a closed ball. Consider that $\bigcup U_n = X \setminus \{x\}$. In particular, this union contains $B$ so $B \subset \bigcup U_n$. Since $B$ is compact, there must be some finite subcover of $\bigcup U_n$ so $B \subset U_k$ for some $k$. Then $B(x,\frac{1}{n}) \cap B = \emptyset$ so $B(x,\frac{1}{n}) \subset B^\complement$. Since $x$ was arbitrary, we know that $B^\complement$ is open.

Next we will show that $B$ is bounded. Let $x \in B$ and consider $B \subset \bigcup B(x,n)$. Since $B$ is compact there is some $k$ such that $B \subset B(x,k)$, and so $B$ is bounded.
\end{proof}

\begin{corollary}
In a metric space $(X,d)$ the set $X$ is always closed.
\end{corollary}

\begin{example}[Note on the converse]
The converse of this is not true. In an arbitrary metric space, not all closed and bounded sets are compact; consider $\Z$ equipped with the discrete metric. In this metric, $\Z \subset B(0,2)$ and it is closed since it contains all of its limit points. However, it is not compact since $\Z \subset \bigcup B(n,\frac{1}{2})$ but this cover admits no finite subcover.
\end{example}

\begin{theorem}[Heine-Borel]
Any closed and bounded subset of $\R^n$ is compact.
\end{theorem}

\begin{lemma}
Any closed subset of a compact set is compact.
\end{lemma}

\begin{proof}
Let $B$ be a compact subset of $X$ and $C$ is a closed subset of $B$. Let $\mathcal{U}$ (defined in the normal way) be an open cover of $C$, so $C \subset \mathcal{U}$. Then $B \subset \mathcal{U} \cup (X \setminus C) = X$. Since $X \setminus C$ is open we know that $\mathcal{U} \cup (X \setminus C)$ is an open cover of $B$. Since $B$ is compact, we know that $\mathcal{U} \cup (X \setminus C)$ admits some finite cover $U_\alpha \cup \cdots \cup U_\zeta \cup (X \setminus C)$. Then it must be true that $C \subset U_\alpha \cup \cdots \cup U_\zeta$, so there is a finte open cover of $C$ so it is compact.
\end{proof}






