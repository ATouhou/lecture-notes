% !TEX root = ../notes.tex

\section{Wednesday, January 13}

\begin{multicols}{2}
\begin{enumerate}
\item Cogito ergo sum
\item Ambulo ergo sum
\item Transparency
\item Cogitatur
\item Quid igitur sum?
\item Sum res cogitant
\item The essence of the mind
\item Truth rule
\item Ugly head
\item The great chain of being
\item Degrees of reality
\end{enumerate}
\end{multicols}

Doubt presupposes existence, so the existence of doubt is an argument for the existence of `I.' Something like \emph{walking} wouldn't work since we can only be certain that we \emph{think} that we're walking, but we can be sure that we are thinking. In this way, Descartes believes that he is certain of the content of his own mind. This certainty opens a gap between the mind and the world, everything the mind percieves. The mind is transparent to itself, however.

\begin{problem}
How can Descartes be sure that the `I' which thinks is preserved between thoughts? Likewise, how can't we really only be sure that thought is going on?
\[ \text{thinking exists} \implies \text{there is a thinking thing} \implies \text{I am that thinking thing} \]
Descartes first argues that thought must have a bearer in some substance, and then argues that there is an inherent quality of self about thoughts (you can't be aware of other people's thoughts, so any thoughts one percieve must be one's own). \note{How does Descartes address the idea that the I which thinks each thought might be different each time, just with different false memories of other beings? Kant considers this, but Descartes likely views this as an implicit part of a thinking substance.}
\end{problem}

Descartes makes some claims about the nature of a thinking mind. Here are some possible claims one could make about the essence of the mind/self: The first two claims are addressed in the second meditation, the third is addressed in the sixth meditation.
\begin{enumerate}
\item I am essentially (necessarily) a thinking thing; Descartes said ``A thought alone is not seperable from me.''
\item I am not essentially (necessarily) extended;
\item I am not an extended thing; \note{What exactly does \emph{extended} mean?}
\item I am not identical to my body;
\end{enumerate}

\subsection*{The Truth Rule}

Descartes reflects on his certainty that he is a thinking thing, and seeks to extrapolate to understand what is a sufficient condition for being sure of something. He ends up with \emph{whatever I perceive very clearly and distinctly is true.} What constitutes clarity and distinctness is not really something Descartes addresses. Descartes also worries that a deceiving god could construct a clear and distinct argument which is actually false, and we could not be sure that we do not fall victim to our own assumtions about this claim. Thus Descartes seeks to first establish that God exists and that God is not a deceiver, and then the truth rule can necessarily follow. If he cannot then even the \emph{cogito ergo sum} argument is called into doubt once again. The proof Descartes proposed is almost universally taken to be wrong (but that doesn't mean that it's not valuable or not worth studying).

\subsection*{Descartes' Proof of God's Existence}

There are many different possible kinds of beings which exist in the world, and they have different degrees of reality. All beings are made of substance, meaning that they have properties but are not themselves properties of anything else. \note{Need substances be material? Is the mind a substance?} Furthermore, substances have a greater degree of reality than their properties, states, or modes. This is because the properties of a substance depend on the substance yet the substance doesn't depend on the properties: You can have a table which isn't rectangular, but you can't just have `rectangular' without it being a rectangular \emph{something}. The highest degree of being would be an infinite substance since it would be independent of anything else, uncreated and permanent. Below this are created, finite substances like a table or a mind. Below even this are the properties of finite substances. Since God is superlative, if God exists then he must be an infinite substance. The kind of reality which substances have is called \emph{formal reality}, reality by the virtue of existance. This is contrasted with objective reality.