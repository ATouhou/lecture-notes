\documentclass[diagram]{notes}

% Bibliography
\usepackage{biblatex}
\addbibresource{references.bib}

% Document Information
\title{Introduction to Systems Programming and Computer Organization}
\courseid{CPSC 323}
\place{Yale University}
\term{Fall}
\year{2018}

\blurb{
	These are lecture notes for CPSC 323a, ``Introduction to Systems Programming and Computer Organization,'' taught by Stanley Eisenstat at Yale University during the fall of 2018.
	These notes are not official, and have not been proofread by the instructor for the course.
	These notes live in my lecture notes respository at 
	\[\text{\url{https://github.com/jopetty/lecture-notes/tree/master/CPSC-323}.}\] 
	If you find any errors, please open a bug report describing the error, and label it with the course identifier, or open a pull request so I can correct it.
}

%% Custom Circuitikz commands
\pgfdeclareshape{fadder}{
	\anchor{center}{\pgfpointorigin} % within the node, (0,0) is the center
	
	\anchor{text} % this is used to center the text in the node
	{\pgfpoint{-.5\wd\pgfnodeparttextbox}{-.5\ht\pgfnodeparttextbox}}

	\savedanchor\faddina{\pgfpoint{-.25cm}{.625cm}} % input a
	\anchor{in1}{\faddina}
	\savedanchor\faddinb{\pgfpoint{.25cm}{.625cm}} % input b
	\anchor{in2}{\faddinb}
	\savedanchor\faddsum{\pgfpoint{0.0cm}{-.625cm}} % sum
	\anchor{sum}{\faddsum}
	\savedanchor\faddcin{\pgfpoint{1.5cm}{0.0cm}} % carry int
	\anchor{cin}{\faddcin}
	\savedanchor\faddcout{\pgfpoint{-1.5cm}{0.0cm}} % carry out
	\anchor{cout}{\faddcout}

	\foregroundpath{
		\pgfsetlinewidth{0.01cm}
		 \pgfpathrectanglecorners{\pgfpoint{1cm}{.625cm}}{\pgfpoint{-1cm}{-.625cm}}
		 \pgfusepath{draw} %draw rectangle
	}

}

\begin{document}

\section*{Syllabus}

\begin{center}
\begin{tabular}{@{}rp{10cm}@{}}
\toprule
\textbf{Instructor} & Stanley C. Eisenstat, \url{sce@cs.yale.edu} \\
\textbf{Lecture} & MW 1:00--2:15 \textsc{pm}, DL 220 \\
\textbf{Recitation} & TBA \\
\textbf{Textbooks} & \fullcite{CompA}; \\
& \fullcite{BLP} \\
\textbf{Midterms} & Monday, October 15, 2018 \\
& Wednesday, December 5, 2018 \\
\bottomrule
\end{tabular}
\end{center}

\subsection*{Coursework}
The class will NOT meet during reading period. There will be 6 assignments requiring an average of 6-9 hours per week (or an
average of 15-25 hours per C assignment, somewhat less per non-C assignment). There will be an in-class examination on Monday, October 15th, and a final on
Wednesday, December 5th. Homework will constitute $\sim$70\% of the final grade; the examinations will
constitute the remainder.

Programs will be submitted electronically and checked using test scripts:  a
public script, which will generally be available at least one week before the
assignment is due; and a more comprehensive private script, which will be
used to assign a grade.  C programs may also be evaluated for ``style''.

Programs should be submitted electronically by 2:00 \textsc{am} on the day specified
in the assignment.  Late work not authorized by a Dean's excuse will be
assessed a penalty of 5\% per calendar day or part thereof and MAY not be
graded at all if more than ten days late or after solutions are released.
The submit-times of the sources determine when the program was completed.

\subsection*{Schedule}
\begin{tabular}{@{}rccccl@{}}
\toprule
\textbf{What} & \textbf{Points} & \textbf{Due} & \textbf{Spec} & \textbf{Script} & \textbf{Program} \\
Homework \#1 & 60 & 09/14 (F) & 08/29 & 09/07 & farthing (C) \\
Homework \#2 & 50 & 09/28 (F) & 09/12 & 09/21 & bshParse (C) \\
Homework \#3 & 30 & 10/12 (F) & 09/26 & 10/05 & cpp2018 (script) \\
Homework \#4 & 60 & 11/02 (F) & 10/10 & 10/26 & LZW2018 (C) \\
Homework \#5 & 60 & 11/16 (F) & 10/31 & 11/09 & bshShell (C) \\
Homework \#6 & 40 & 12/13 (R) & 11/14 & 12/06 & Networking (script) \\
\bottomrule
\end{tabular}

\subsection*{Topics Covered}
\textbf{Systems programming in a high level language:} user-level interfaces to a typical operating system (Linux), writing programs (e.g., a shell) that interact with the operating system;

\textbf{Elementary machine architecture / computer organization:} computer arithmetic and general structure/organization of machines, approaches to parallelism (vector, SIMD, MIMD, networks), instruction set architectures and pipelining instruction execution;

\textbf{Operating systems:} implications of concurrency, implementation of semaphores at machine level (a la Dijkstra), implementation and ramifications of virtual memory and caches;

\textbf{Other:} data compression, error detection and correction, computer networks.

\printbibliography

%% Sanskrit
\section{August 30, 2018}

Linguistics was the model science in ancient india

Today, we'll talk about phonology --- range of sounds, how they're identified.
Phonology is the sound structure of a language, as opposed to syntax or semantics. For the time being we'll transliterate Sanskrit into the Latin script; tomorrow, we'll move on to devanagari. Sanskrit has been written in a variety of scripts, but devanagari is the most commonly used.
Much of the historic writings in Sanskrit are written in Devanagari or one of its predecessors.

Sanskrit's phonological system is very ``scientific'' --- it's very systematic and structured. Grammarians divide sounds in Sanskrit into two groups: the \emph{svara}, or vowels; and the \emph{vyañjana}, or consonants.

\subsection*{Svara}
In Sanskrit, svara (meaning sound, tone, or accent) are sounds which can be pronounced independently, contrasting with consonants which have an implied vowel associated with them.
There are 14 vowels in Sanskrit (really only 13) which are organized by two orthogonal qualities: the place of articulation and the length.

\begin{wrapfigure}{l}{0.25\textwidth}
\begin{tabular}{@{}cc@{}}
\toprule
Hrasva & Dīrga \\
\midrule
a & ā \\
i & ī \\
u & ū \\
ṛ & ṝ \\
ḷ & \emph{ḹ}\footnote{This vowel is never attested, and only exists to complete the chart} \\
\midrule
e & ai \\
o & au \\
\bottomrule
\end{tabular}
\end{wrapfigure}

Long and short vowels are distinguished by the number of \emph{mātrā} they consume.
You can think of mātrā as a kind of tempo marking for the language; short vowels, or \emph{hrvasa}, last for ``one beat'' of time, while long vowels, or \emph{dīrga}, last for two. In ancienct Vedic Sanskrit, there was a third category of vowels, called protracted vowels or \emph{pluta}, which lasted for three beats, but these were lost in Classical Sanskrit.
Vowels can also be simple (\emph{śuddha}, or ``pure''), or complex (\emph{samyukta}, or ``joined''). Samyukta are known in linguistics as diphthongs, and are made from the combination of two or more śuddha. In the table on the left, the samyukta are the bottom four.

Here are some examples of the vowels in Sanskrit words:

\begin{figure}[h]
\begin{center}
\begin{tabular}{@{}ll@{}}
\toprule
Abhava (absense, destruction) & Ātman (self) \\
Idam (this) & Īśvara (lord) \\
Ubhaya (both) & Abhūt (was) \\
Kṛṣṇa (Krishna) & Nṝṇām\footnote{The consonant ṝ is very rare, appearing only in nominative and genetive nouns} (Of men) \\
Kḷpti (arrangement) & --- \\
Etad (this) & Bhojana (food) \\
Chaitanya (a saint) & Saubhagya (good fortune) \\
\bottomrule
\end{tabular}
\end{center}
\end{figure}

\subsection*{Consonants}

 There are three kinds of consonants in SKRT. Consontants are classified by the place of articulation (sthāna):
 	- Velar or kan.t.hya
 	- soft pallette, palatal (talva)
 	- hard pallatte, retroflex (murtanyo)
 	- dental (danta)
 	- labial (os.t.ha)

 Also classified by the closure of the lips
 	- full contact, stops (sparśa)

 Voiced/Unvoiced
 	- ghośavat (voiced)
 	- aghośa 	(unvoiced)
 
 Apsiration
 	- in english, spin vs pin, cut vs king
 	- Alpha prana 	(unaspirated)
 	- maha prana	(aspirated)



\end{document}