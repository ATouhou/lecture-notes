\section{Friday, 16 November 2016}

\epigraph{``No one really cares about left ideals.''}{Miki}

\begin{problem}[Challenge Problem]
What is $\Aut(Z_6 \times Z_2)$?
\end{problem}

\subsection{Ring Maps}

\begin{definition}[Ring Homomorphisms]
A \emph{ring homomorphism} s a map $\phi : R \to S$ where $\phi(a + b) = \phi(a) + \phi(b)$ and $\phi(ab) = \phi(a)\phi(b)$. The kernel of $\phi$ is the underlying group kernel; that is, $\ker \phi = \{a \in R \mid \phi(a) = 0\}$. If $\phi$ is bijective it is a ring isomorphism.
\end{definition}

\begin{example}
\begin{enumerate}
\item Consider $\Z \to \Z/n/Z$ via $a \mapsto \bar{a}$. Then $\bar{a}\bar{b} = \overline{ab}$ and $\bar{a} + \bar{b} = \overline{a + b}$.
\item Consider $\Q[x] \to \Q$ via $f(x) \mapsto f(0)$. That is, we evaluate the function at zero. Multiplying and adding polynomials works so this map is a homomorphism.
\end{enumerate}
\end{example}

\begin{proposition}
If $\phi : R \to S$ is a homomorphism then $\Im(\phi)$ is a subring of $S$ and $\ker \phi$ is a subring of $S$.
\end{proposition}

\begin{corollary}
The important fact about the kernel is that if $a \in \ker\phi$ then $\phi(a)\phi(x) = 0$ for \emph{any} $x \in R$. This is a really strong property. It means that the kernel isn't just closed under multiplication, its closed under multiplication by \emph{anything}.
\end{corollary}

\begin{definition}[Ideals]
A left ideal $I \subseteq R$ is a subring of $R$ with the property that $rI \subseteq I$ for all $r \in R$. A similar definition exists for a right idea, where $Ir \subseteq I$ for all $r \in R$. If $I$ is both a left and right ideal, then it is just called an \emph{ideal}.
\end{definition}

\begin{example}
For any ring $R$ and any homomorphism $\phi$, then $\ker \phi$ is an ideal in $R$.
\end{example}

\subsection{Quotient Rings}

Consider $\phi : \Z \mapsto \Z/2\Z$. Then let $\ker\phi = I$, so $(I,+) \leq (\Z,+)$. We know that we have two cosets, $0 + I$ and $1 + I$ which are derived from our cosets of $\Z/2\Z$. Then we can form a new object, $\Z/I$. We know from group structure that the multiplication is well defined, and multiplication also works since $\textsc{odd} \times \textsc{even} = \textsc{even}$ and $\textsc{even} \times \textsc{even} = \textsc{even}$.

\begin{example}[Example where this doesn't work]
Let $R = \Z[x]$ and $S \subset R$ be the polynomials in $x^2$, so $s = 1 + x^2 + 3x^6$, for example. Notice that $R/S$ doesn't work. For example, we want that $\bar{1} \cdot \bar{x} = \bar{x}$. But $(1 + x^2) \cdot x = x + x^3$ which is not in $x + S$.
\end{example}

The punchline for this is that quotient rings only work when $S$ is an ideal. This is kind of like the condition for normality for groups. For any $S \subseteq R$ we know that $(R/S,+)$ is okay since $(S,+) \normal (R,+)$ since $R$ is commutative in addition, but the multiplication is where we get tripped up. For any $r_1,r_2 \in R$ and any $a \in S$ we want that $r_1(r_2 + a) \in r_1r_2 + S$. But $r_1(r_2 + a) = r_1r_2 + r_1a$ which means that $r_1a$ must be in $S$ for \emph{any} $r_1$ which happens only if $S$ is a left  ideal. Similarly, we need that $(r_1 + a)r_2 = r_1r_2 + ar_2$, which tells us that $S$ must be a right idea, so it must be an idea.

\begin{proposition}
If $I \subseteq R$ is an ideal then $R/I$ is a ring and $(r + I) + (s + I) = (r+s) + I$ and $(r+I) + (s + I) = rs + I$.
\end{proposition}

\begin{example}
\begin{parts}
\part Consider $\Q[x] \to \Q$ again where $p(x) \mapsto p(0)$. The kernel of this is the set of polynomials with a zero constant term. Notice then that $\Q[x]/\ker\phi \cong \Q$ which looks a hell of a lot like the first isomorphism theorem for groups.
\part Consdier $\Q[x]$ and $I = \{\text{polynomials of degree $\geq 2$}\}$. This doesn't work out well.
\end{parts}
\end{example}

\subsection{Using Quotient Rings}

Consider $\pi : \Z \to \Z/p\Z = \F_p$. This induces a map $\Z[x] \to \F_p[x]$ where $a_nx^n \mapsto \bar{a}_nx^n$ (looking at the coefficients modulo $p$). Suppose there exists an $a \in Z$ such that $f(a) = 0$. Then $\bar{f}(\bar{a}) = \bar{0}$.

\begin{example}
Consider $f(x) = 3x^{52} + 4x^{49} - 16x^{20} - 17x - 1755$. Let's look at it modulo $2$. Then $\bar{f}(x) = x^{52} + x + 1$. Then $\bar{f}(\bar{0}) \not= \bar{0}$ and $\bar{f}(\bar{1}) \not= \bar{0}$ so there are no integral roots.
\end{example}

\begin{problem}
If we have a ring homomorphism $\phi : R \to S$, can we always induce a homomorphism $\psi : R[x] \to S[x]$.
\end{problem}