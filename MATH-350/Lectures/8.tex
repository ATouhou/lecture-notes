\section{September 17, 2018}

\epigraph{``The only thing I learned for years was how to count hedgehogs in a field.'' (In the midst of a wonderful and inspirational talk about being a mathematician.)}{Miki}

\subsection{Finite Cyclic Groups}

Today, we'll cover finite cyclic groups. This will be very similar to the previous lecture on infinite cyclic groups. As a reminder, here are the propositions for infinite cyclic groups:

\begin{proposition}[Infinite Cyclic Groups]
Let $G$ be an infinite cyclic group.
\begin{enumerate}
\item The order of $G$ is infinite, with $G = \{\dotsc,x^{-1},1,x,x^2,\dotsc\}$ all distinct.
\item The group $G$ is isomorphic to $\Z$.
\item The group $G$ is generated by $x^n$ if and only if $n = \pm 1$.
\item Every subgroup $G$ is cyclic.
\end{enumerate}
\end{proposition}

Now for the finite case. 

\begin{proposition}[Finite Cyclic Groups]
 Let $G = \langle x \rangle$ with $\abs{G} = n < \infty$.
 \begin{enumerate}[label=P\arabic*.]
 \item The group $G$ is exactly $\{1,x,\dotsc,x^{n-1}\}$.
 \item The group $G$ is isomorphic to $\Z/n\Z$.
 \item The group $G$ is generated by $x^k$ if and only if $\gcd(k,n) = 1$.
 \item Every subgroup of $G$ is also cyclic. That is, for all $k > 0$ where $k$ divides $n$ we get a subgroup $H$ of order $k$ generated by $x^{n/k}$
 \end{enumerate}
\end{proposition}
\begin{proof}[Proof of~P1]
We know that $1,\dotsc,x^{n-1}$ are all in $G$. Suppose that $x^a = x^b$ for some distinct $a,b$. Then $x^{b-a} = 1$ for $0 < b-a < n$, which is a contradiciton since $\abs{x} = n$. In fact, this set enumerates $G$. Suppose that $x^k \in G$ for some $k \in \Z$. We use the divison algorithm to write that $k = an + r$ for some $a,r \in \Z$. Then $x^k = x^{an+r} = (x^n)^ax^r = x^r$, so $x^k$ is in $G$.
\end{proof}
\begin{proof}[Proof of~P2]
Let $\phi : \Z/n\Z \to G : \bar{k} \mapsto x^k$ where $k$ is any representative of $\bar{k} \in \Z$. To show that $\phi$ is well-defined, consider another representative $\ell$ of $\bar{k} \in \Z$. Then $\ell = k + an$, so $x^\ell = x^{k+an} = x^k(x^n)^a = x^k$. To show that $\phi$ is a homomorphism, consider that $\phi(\bar{m}+\bar{n}) = x^{m+n} = x^{m}x^n$, so $\phi$ is multiplicative. Finally, we know that $\phi$ is surjective and injective by P1. This tells us that, up to an isomorphism, there are only really two cyclic groups; $\Z$ if the group is of infinite order, or $\Z/n\Z$ if it is finite.
\end{proof}
\begin{proof}[Proof of~P3]
This is more of a sketch. Recall that $\langle \abs{x^k}\rangle = \abs{x^k}$, and this is $n$ if and only if $\gcd(k,n) = 1$. In general, $\abs{x^k} = n / \gcd(k,n)$.
\end{proof}
\begin{proof}[Proof of~P4]
Exactly the same as the infinite case.
\end{proof}

Now that we've covered cyclic groups, it's helpful to introduce some notation to represent them.
\begin{notation}[$\Z_n, C_n$]
We write the multiplicative cyclic group of order $n$ as $\Z_n$. The additive cyclic group of order $n$, which we've been writing as $\Z/n\Z$, is commonly written as $C_n$.
\end{notation}

\subsection{Subgroups}

\cite{DF} uses the notation $S \subset G$ to mean that $S$ is a subset of $G$, and $H \leq G$ to mean that $H$ is a subgroup of $G$.

\begin{definition}[Subgroup]
Let $S \subset G$ be nonempty. Let $H = \{a_1^{\varepsilon_1}\cdots a_k^{\varepsilon_k}\}$ where $a_i \in S$ and $\varepsilon_i = \pm 1$ for $k \in \Z_{\geq 0}$. This sequence of $a_i^{\varepsilon_i}$ is called a \emph{word}. Note that $a_i$ need not be distinct. Then $H$ is a subgroup of $G$.
\end{definition}

\begin{proof}
First, we know that $H \subset G$. We know that $1 \in H$. Since any concatenation of words is also a word, we know that $H$ is closed under multiplication. Finally, since $((ab)^n)^{-1} = b^{-n}a^{-n}$ we know that the inverses of a word are words themselves, and so $H$ is closed under inversion.
\end{proof}