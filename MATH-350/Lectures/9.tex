\section{September 19, 2018}

\epigraph{``Funny things happen with groups, which is why they're fun!''}{Miki}

Recall from last time how we defined a subgroup $H$ of $G$ in terms of words where the powers of each element was $\pm 1$. If $G$ is abelian we can combine elements of like bases to get powers which can be any integral value. If we assume that $\abs{a_i} = d_i$ is finite for all $a_i \in H$, then we know that $\abs{H} \leq d_1 \cdots d_k$. This gives us a limit on the order of a subgroup; if $G$ is abelian then the order of a subgroup is bounded above by the product of the orders of the generating elements. On the other hand, if $G$ is not abelian then this does not always hold. Consider $G = \langle a,b \mid a^2 = b^2 = 1 \rangle$. If $G$ isn't commutative, then $(ab)^n \not= a^nb^n$ for all $n$ and so we can just create infinitely many words by appending $ab$ to one another and so the order is infinite.

\begin{lemma}
Let $G = \{a_1^{n_1} \cdots a_k^{n_k}\}$ be abelian, and let each $a_i$ have finite order $d_i$. Then $\abs{G} \leq d_1 \cdots d_i$.
\end{lemma}

\begin{proposition}
Let $G$ be a group and let $\mathcal{L}$ be a collection of subgroups of $G$. Then \[ K = \bigcap_{L \in \mathcal{L}} L \]
is a subgroup of $G$.
\end{proposition}

\begin{definition}[Subgroup]
Let $S \subset G$ and let $\mathcal{L} = \{L \leq G \mid s \subset L\}$. Then the subgroup generated by $S$ is \[ K = \bigcap_{L \in \mathcal{L}} L. \]
\end{definition}

What do we know from this definition? Well, $S \subset K$ and $K \leq G$. We want to say that $K \in \mathcal{L}$ is the minimal element, so $K = L_i$ for some $i$.

\begin{definition}[Minimal Element]
Let $\mathcal{M}$ be a collection of subsets of $G$. A minimal element is an element $M$ of $\mathcal{M}$ such that if $M' \in \mathcal{M}$ and $M' \subset M$ then $M = M'$. It's like ``the smallest element'' except there could be multiple minimal elements.
\end{definition}

We want to show that $K$ is \emph{the} minimal element of $\mathcal{L}$.

\begin{proof}[Proof $K$ is minimal]
Let $L \in \mathcal{L}$. Then $K \subset L$. Then either $K = L$ or $L$ is not minimal.
\end{proof}

\begin{proof}[Proof $K$ is \emph{the} minimal element]
Suppose there is another minimal $M$ in $\mathcal{L}$. By definition $K \subset M$ so by minimality $M = K$.
\end{proof}

\begin{proposition}
Our two definitions for subgroup (generated by words $H$ and minimal element of collection $K$) containing elements of $S \subset G$ are equivalent.
\end{proposition}

\begin{proof}
$H \leq G$ and $S \subset H$ by the construction of $1$-letter words. Then $H \in \mathcal{L}$ so $K \subset H$. On the other hand, $S \subset K$ and $K$ is a group. Then $K$ contains all inverses and products of elements in $S$, so it contains all words and therefore contains $H$. Then $H \subset K$. Putting this together, we have that $K = H$.
\end{proof}

\begin{definition}[Lattice]
Given a group $G$, a lattice is a diagram showing all subgroups of $G$ which shows containment between the subgroups.
\end{definition}

\begin{figure}[h]
	\caption{Lattice Diagram of $C_2$}
\end{figure}

Recall from last time that for $C_n$ the subgroups are paired with the divisors $k$ of $n$; then $\langle k \rangle$ generates subgroup of order $n / k$.

\begin{figure}[h]
	\caption{Lattice Diagram of $C_4$}
\end{figure}

\begin{figure}[h]
	\caption{Lattice Diagram of $C_8$}
\end{figure}

\begin{figure}[h]
	\caption{Lattice Diagram of $C_6$}
\end{figure}

\begin{figure}[h]
	\caption{Lattice Diagram of $S_3$}
\end{figure}