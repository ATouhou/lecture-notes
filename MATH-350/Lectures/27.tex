\section{Wednesday, 7 November 2018}

\subsection{Classify Groups of Order \texorpdfstring{$12$}{12}}
Online. Look it up.

\subsection{``A Simple Song''}

This may be the best thing I have ever seen. Lyrics to follow shortly.

\begin{raggedright}
\raggedright\itshape\small
What are the orders of all simple groups? \\
I speak of the honest ones, not of the loops. \\
It seems that old Burnside their orders has guessed. \\
Except for the cyclic ones, even the rest. \\ \hspace{1em} \\

Groups made up with permutes will produce
some more: \\
For A\textsubscript{n} is simple, if n exceeds 4. \\
Then, there was Sir Matthew who came into
view \\
Exhibiting groups of an order quite new. \\ \hspace{1em} \\

Still others have come on to study this thing. \\
Of Artin and Chevalley now we shall sing. \\
With matrices finite they made quite a list. \\
The question is: Could there be others they've missed? \\ \hspace{1em} \\

Suzuki and Ree then maintained it's the case \\
That these methods had not reached the end of
the chase. \\
They wrote down some matrices, just four by
four, \\
That made up a simple group. Why not make
more? \\ \hspace{1em} \\

And then came the opus of Thompson and Feit, \\
Which shed on the problem remarkable light. \\
A group, when the order won't factor by two, \\
Is cyclic or solvable. That's what is true. \\ \hspace{1em} \\

Suzuki and Ree had caused eyebrows to raise, \\
But the theoreticians they just couldn't faze. \\
Their groups were not new: if you added a twist, \\
You could get them from old ones with a flick of the wrist. \\ \hspace{1em} \\

Still, some hardy souls felt a thorn in their side. \\
For the five groups of Mathieu all reason defied; \\
Not $A_n$, not twisted, and not Chevalley, \\
They called them sporadic and filed them away. \\ \hspace{1em} \\

Are Mathieu groups creatures of heaven or hell? \\
Zvonimir Janko determined to tell. \\
He found out what nobody wanted to know: \\
The masters had missed 1 7 5 5 6 0. \\ \hspace{1em} \\

The floodgates were opened! New groups were the rage! \\
(And twelve or more sprouted, to greet the new age.) \\
By Janko and Conway and Fischer and Held \\
McLaughlin, Suzuki, and Higman, and Sims. \\ \hspace{1em} \\

No doubt you noted the last lines don't rhyme. \\
Well, that is, quite simply, a sign of the time. \\
There's chaos, not order, among simple groups; \\
And maybe we'd better go back to the loops
\end{raggedright}

\subsection{Introductions to Rings}

\begin{definition}[Ring]
A \emph{ring} is a tuple $(R,+,\times)$ where $(R,+)$ is an abelian group, $\times$ is an associative binary operaation, and the distributive laws hold.
\end{definition}