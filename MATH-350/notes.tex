\documentclass{notes}

% Bibliography
\usepackage[
	style = alphabetic ,
]{biblatex}
\addbibresource{references.bib}

% Document Information
\title{Introduction to Abstract~Algebra}
\courseid{MATH 350}
\place{Yale University}
\term{Fall}
\year{2018}

\blurb{
	These are lecture notes for MATH 350a, ``Introduction to Abstract Algebra,'' taught by Marketa Havlickova at Yale University during the fall of 2018.
	These notes are not official, and have not been proofread by the instructor for the course.
	These notes live in my lecture notes respository at 
	\[\text{\url{https://github.com/jopetty/lecture-notes/tree/master/MATH-350}.}\] 
	If you find any errors, please open a bug report describing the error, and label it with the course identifier, or open a pull request so I can correct it.
}

\begin{document}

\section*{Syllabus}

\begin{center}
\begin{tabular}{@{}rp{10cm}@{}}
\toprule 
\textbf{Instructor} & Marketa Havlickova, \url{miki.havlickova@yale.edu} \\
\textbf{Lecture} & MWF 10:30--11:20 \textsc{am}, LOM 205 \\
\textbf{Recitation} & TBA \\
\textbf{Textbook} & \fullcite{DF} \\
\textbf{Midterms} & Wednesday, October 10, 2018 \\
& Wednesday, November 14, 2018 \\
\textbf{Final} & Monday, December 17, 2018, 2:00--5:30 \textsc{pm} \\
\bottomrule
\end{tabular} \\[3ex]
\end{center}

Abstract Algebra is the study of mathematical structures carrying notions of ``multiplication'' and/or ``addition''. Though the rules governing these structures seem familiar from our middle and high school training in algebra, they can manifest themselves in a beautiful variety of different ways. The notion of a group, a structure carrying only multiplication, has its classical origins in the study of roots of polynomial equations and in the study of symmetries of geometrical objects. Today, group theory plays a role in almost all aspects of higher mathematics and has important applications in chemistry, computer science, materials science, physics, and in the modern theory of communications security. The main topics covered will be (finite) group theory, homomorphisms and isomorphism theorems, subgroups and quotient groups, group actions, the Sylow theorems, ring theory, ideals and quotient rings, Euclidean domains, principal ideal domains, unique factorization domains, module theory, and vector space theory. Time permitting, we will investigate other topics. This will be a heavily proof-based course with homework requiring a significant investment of time and thought. The course is essential for all students interested in studying higher mathematics, and it would be helpful for those considering majors such as computer science and theoretical physics.

Your final grade for the course will be determined by
\[ \max\left\{
	\begin{array}{cccc}
		25\%\text{ homework} + 20\%\text{ exam 1} + 20\%\text{ exam 2} + 35\%\text{ final} \\
		25\%\text{ homework} + 10\%\text{ exam 1} + 20\%\text{ exam 2} + 45\%\text{ final} \\
		25\%\text{ homework} + 20\%\text{ exam 1} + 10\%\text{ exam 2} + 45\%\text{ final}
	\end{array}
\right\}. \]

\printbibliography

\section{August 31, 2018}

As always, Miki began class at precicely 10:25 \textsc{am}. She wrote a review of last lecture on the bard, and then posed the following question as a warm up. She also talked about how the DUS department is arguing over whether money should be spent on T-shirts or chocolate (Miki thinks chocolate).

\begin{problem}[Warm Up]
Are these groups?
\begin{parts}
\part[w:1a] $(\Z/n\Z, \times)$;
\part[w:1b] $(\Z/n\Z \setminus \{0\}, \times)$
\end{parts}
\end{problem}

\begin{solution}
The solutions to the warm-up
\begin{subproof}[Solution to~\ref{w:1a}]
No, since $0$ has no inverse.
\end{subproof}

\begin{subproof}[Solution to~\ref{w:1b}]
No, this only works when $n$ is prime. For any factors $a,b$ of $n$, $a \times b = 0$, which isn't in the group. We say that $(\Z/p\Z, \times)$ is a group for all prime $p$.
\end{subproof}
\end{solution}

\begin{theorem}[Fermat's little theorem]
For prime $p$ and composite $a = np$, hen $a^{p-1} \equiv 1 \pmod{p}$. 
\end{theorem}

\begin{lemma}
If $\bar{a} \in \Z/p\Z \setminus \{0\}$, then $\bar{a}$ has an inverse in $(\Z/p\Z \setminus \{0\}, \times)^*$.
\end{lemma}

\begin{definition}[Units]
A unit is something which has an inverse. The units of a group are denoted by putitng an asterisk after teh group, eg $(\Z/p\Z \setminus \{0\}, \times)^*$.
\end{definition}

\begin{example}
For integers modulo 4, $(\Z/4\Z, \times)^* = \{\bar{1}, \bar{3}\}$.
\end{example}

\begin{problem}[On Homework]
What are the conditions for determining the units of a group? We know it must have an inverse, but that's hard to check. Instead, we know that $a$ is a unit if and only if $\gcd(a,n) = 1$. Prove this.
\end{problem}

\subsection*{Symmetries of a regular $n$-gon}
Miki is angry with the book because she doesn't like how it treats symmetries, I think because she wants $D_{2n}$ to be called $D_n$.

Miki drew a triangle on the board, and began talking about the different operations we can preform on that triangle to preserve symmetries. She introduced $s$ to mean a reflection, and $r$ to mean a rotation. For a triangle, there are three distinct reflections,
\[ s = \{s_1, s_2, s_3 \}, \]
where $s_i$ is the reflection across the line $OA_1$. We can also rotate the triangle in two directions.

We know that these are all the symmetries, since we can count the permutations of the triangle. We've exhauseted then, so we know that there can't be any more elements of the triangle-symmetry group $D_6$. In fact, because of the permutation fact, we konw that $\abs{D_{2n}} = 2n$. Some other observations about $D_{2n}$:
\begin{itemize}
\item $s^2 = e \implies s = s^{-1}$;
\item rotating twice clockwise is the same as rotating counterclockwise, so these aren't unique elememnts;
\item $r^n = e$
\item $rs = s_2$, so $s_n$ is just a combination of $r$ and $s$ --- then we can generate the entire group with just $r$ and $s$.
\end{itemize}
These things lead us to discover a new object.

\begin{definition}[Generators]
For a group $G$, the generators of $G$ is a set $ S = \{a,b,\dotsc : a,b,\dotsc \in G\}$ where $G$ is equal to all possible sombinations of elements of $S$. For $D_{2n}$, we could say that $D_{2n}$ is generated by $r$ and $s$.
Usually there isn't a way to guess the generators of a group easily.
\end{definition}

\begin{definition}[Relations]
A relation is a way of writing equivalent elements of groups. For example, in $D_{2n}$,
\[ r^3 \equiv 1, \qquad s^2 \equiv 1, \qquad sr \equiv r^2s. \]
Relations allow us to define how we can commute elements of the group.
\end{definition}

\begin{definition}[Presentation]
A presentation of a group are the generators combined with the relations necessary to create the group. The largest group which is generated from the generators and which satisfies the relations, and has no other relations, is our group. A presentation is written as $\langle a,b \mid \text{relations between $a$ and $b$} \rangle$, where $a$ and $b$ are the generators of the group.
\end{definition}

Now Miki told us that the group of the  symmetires of a regular $n$-gon is the dihedral group of order $2n$, written either as \{$D_{2n}$ or $D_{n}$\}, depending on if you are a representation theorist or not.

\begin{problem}[HW]
Why is the order of $D_{2n}$ always $2n$?
\end{problem}

\subsection*{Symmetric group on $n$ elements}

Miki defined the symmetric group on $n$ elements $S_n$, which is just the permutations of $n$ elemnts. Notice that $D_{2n}$ is a subgroup of $S_n$. We know that the order of $S_n = n!$ and the order of $D_{2n} = 2n$.

[Insert diagrams of different ways to denote permuations, like the cycle notation]

\section{September 4, 2018}

\begin{definition}
For a set $\Omega$, the symmetric group on $\Omega$ is $S_\Omega = \{\text{bijective maps $\Omega \to \Omega$}\}$. For $n \in \N$, we say that $S_n = S_{\{1,\dotsc,n\}}$. This is usually called the symmetric group on $n$ letters.
\end{definition}

Let's consider this example for $S_4$ (warning, there's some cyclic decomposition for $g_1, g_2$?)

\begin{example}
Consider the following maps $g_1, g_2 \in S_4$,
\[
	\begin{array}{c}
	g_1 \\
	1 \to 2 \\
	2 \to 1 \\
	3 \to 4 \\
	4 \to 2
	\end{array} \qquad
	\begin{array}{c}
	g_2 \\
	1 \to 3 \\
	2 \to 1 \\
	3 \to 2 \\
	4 \to 4
	\end{array}
\]
We can also write these as $g_1 = (12)(34)$ and $g_2 = (132)(4)$. In this notation, how to we multiply things? E.g., what is $g_2g_1$? Well, we can write this naïvely as $(132)(4)(12)(34)$, but we don't want to repeat any numbers. Let's see what happens to $1$:
\[ (132)(4)(12)(34) \cdot 1 = (132)(4)(12) \cdot 1 = (132) \cdot 2 = 1. \] For $2$, we get 
\[ (132)(4)(12)(34) \cdot 2 = 3. \] For $3$, this comes $g_2g_1 \cdot 3 = 4$, and for $4$ we have $g_2g_1 \cdot 4 = 2$. Then $g_2g_1 = (1)(234)$. Unfortunately, doing this sort of element-wise reduction is the fastest way to multiply anything.
\end{example}

\begin{problem}
Someone asked the question ``does order matter?'' E.g., is it true that $(12)(34) = (34)(12)$ always?
\end{problem}
\begin{solution}
No. They are the same. Also, $(abc) = (bca)$; as long as the sign of the permutation of the cycle elements is $+1$, it won't matter how you order the elements of a cycle.
\end{solution}

\begin{problem}
Does order matter when there is a number repeated (when the cycles are not disjoint)? E.g. does $g_1g_2 = g_2g_1$?
\end{problem}
\begin{solution}
Yeah, order does matter. Consider that $(12)(13) \not= (13)(12)$. This means that, in general, $S_n$ is not abelian.
\end{solution}

\begin{problem}
Consider $S_5$, where $g = (123)(45)$ and $h = (12345)$. Find $g^2, g^{-1}, h^{-1}$. Fun fact, it's easy.
\end{problem}

These facts lead us to an interesting and useful conclusion.
\begin{proposition}
For any $g \in S_n$, we can write $g$ as a product of disjoint cycles.
\end{proposition}

This gives us an interesting observation for $S_n$.

\begin{proposition}
Let $g \in S_n$ be written as the product of disjoint cycles. Then the order of $g$ is the least common multiple of the orders of the disjoint cycles.
\end{proposition}

\subsection*{Fields $n$ stuff}

\begin{definition}
A field $k$ is a triple $(F, +, \times)$ where $(F,+)$ and $(F \setminus \{0\},\times)$ are groups where $F^\times = F \setminus \{0\}$ and where multiplication distributes over addition. Some cannonical examples are $\Q$, $\R$, $\C$, $\F_p = \Z/p\Z$ for prime $p$.
\end{definition}

A brief note on finite fields: for a finite field $\F$, we know that $\abs{\F} = p^n$ for some prime $p$ and some $n \geq 1$.

Now that we have fields, we can get matrices for free. Consider the cannonical matrix group $\mathrm{GL}_n(k)$ of invertible matrices with entries in $k$.

\begin{example}
Consider $\mathrm{GL}_2(\F_2)$ where $\F_2 = \{\bar{1}, \bar{2}\}$ (note that this is just $\Z / 2\Z$). What is the order of $\mathrm{GL}_2(\F_2)$?
\end{example}

\begin{proof}
There are six. Any element cannot have three or four zeros in it, nor two zeros in the same row or column. Then just count the total possibilities.
\end{proof}

\end{document}