\documentclass{lnotes}

% Bibliography
\usepackage[
	style = alphabetic ,
]{biblatex}
\addbibresource{references.bib} 

\usepackage{epigraph}
\usepackage{wasysym}

% Document Information
\title{Introduction to Abstract~Algebra}
\course{MATH 350}
\place{Yale University}
\term{Fall}
\year{2018}

\blurb{
	These are lecture notes for MATH 350a, ``Introduction to Abstract Algebra,'' taught by Marketa Havlickova at Yale University during the fall of 2018.
	These notes are not official, and have not been proofread by the instructor for the course.
	These notes live in my lecture notes respository at 
	\[\text{\url{https://github.com/jopetty/lecture-notes/tree/master/MATH-350}.}\] 
	If you find any errors, please open a bug report describing the error, and label it with the course identifier, or open a pull request so I can correct it.
}

\begin{document}

\section*{Syllabus}

\begin{center}
\begin{tabular}{@{}rp{10cm}@{}}
\toprule 
\textbf{Instructor} & Marketa Havlickova, \url{miki.havlickova@yale.edu} \\
\textbf{Lecture} & MWF 10:30--11:20 \textsc{am}, LOM 205 \\
\textbf{Recitation} & TBA \\
\textbf{Textbook} & \fullcite{DF} \\
\textbf{Midterms} & Wednesday, October 10, 2018 \\
& Wednesday, November 14, 2018 \\
\textbf{Final} & Monday, December 17, 2018, 2:00--5:30 \textsc{pm} \\
\bottomrule 
\end{tabular} \\[3ex]
\end{center}

Abstract Algebra is the study of mathematical structures carrying notions of ``multiplication'' and/or ``addition''. Though the rules governing these structures seem familiar from our middle and high school training in algebra, they can manifest themselves in a beautiful variety of different ways. The notion of a group, a structure carrying only multiplication, has its classical origins in the study of roots of polynomial equations and in the study of symmetries of geometrical objects. Today, group theory plays a role in almost all aspects of higher mathematics and has important applications in chemistry, computer science, materials science, physics, and in the modern theory of communications security. The main topics covered will be (finite) group theory, homomorphisms and isomorphism theorems, subgroups and quotient groups, group actions, the Sylow theorems, ring theory, ideals and quotient rings, Euclidean domains, principal ideal domains, unique factorization domains, module theory, and vector space theory. Time permitting, we will investigate other topics. This will be a heavily proof-based course with homework requiring a significant investment of time and thought. The course is essential for all students interested in studying higher mathematics, and it would be helpful for those considering majors such as computer science and theoretical physics.

Your final grade for the course will be determined by
\[ \max\left\{
	\begin{array}{cccc}
		25\%\text{ homework} + 20\%\text{ exam 1} + 20\%\text{ exam 2} + 35\%\text{ final} \\
		25\%\text{ homework} + 10\%\text{ exam 1} + 20\%\text{ exam 2} + 45\%\text{ final} \\
		25\%\text{ homework} + 20\%\text{ exam 1} + 10\%\text{ exam 2} + 45\%\text{ final}
	\end{array}
\right\}. \]

\printbibliography

\section{August 31, 2018}

As always, Miki began class at precicely 10:25 \textsc{am}. She wrote a review of last lecture on the bard, and then posed the following question as a warm up. She also talked about how the DUS department is arguing over whether money should be spent on T-shirts or chocolate (Miki thinks chocolate).

\begin{problem}[Warm Up]
Are these groups?
\begin{parts}
\part[w:1a] $(\Z/n\Z, \times)$;
\part[w:1b] $(\Z/n\Z \setminus \{0\}, \times)$
\end{parts}
\end{problem}

\begin{solution}
The solutions to the warm-up
\begin{subproof}[Solution to~\ref{w:1a}]
No, since $0$ has no inverse.
\end{subproof}

\begin{subproof}[Solution to~\ref{w:1b}]
No, this only works when $n$ is prime. For any factors $a,b$ of $n$, $a \times b = 0$, which isn't in the group. We say that $(\Z/p\Z, \times)$ is a group for all prime $p$.
\end{subproof}
\end{solution}

\begin{theorem}[Fermat's little theorem]
For prime $p$ and composite $a = np$, hen $a^{p-1} \equiv 1 \pmod{p}$. 
\end{theorem}

\begin{lemma}
If $\bar{a} \in \Z/p\Z \setminus \{0\}$, then $\bar{a}$ has an inverse in $(\Z/p\Z \setminus \{0\}, \times)^*$.
\end{lemma}

\begin{definition}[Units]
A unit is something which has an inverse. The units of a group are denoted by putitng an asterisk after teh group, eg $(\Z/p\Z \setminus \{0\}, \times)^*$.
\end{definition}

\begin{example}
For integers modulo 4, $(\Z/4\Z, \times)^* = \{\bar{1}, \bar{3}\}$.
\end{example}

\begin{problem}[On Homework]
What are the conditions for determining the units of a group? We know it must have an inverse, but that's hard to check. Instead, we know that $a$ is a unit if and only if $\gcd(a,n) = 1$. Prove this.
\end{problem}

\subsection*{Symmetries of a regular $n$-gon}
Miki is angry with the book because she doesn't like how it treats symmetries, I think because she wants $D_{2n}$ to be called $D_n$.

Miki drew a triangle on the board, and began talking about the different operations we can preform on that triangle to preserve symmetries. She introduced $s$ to mean a reflection, and $r$ to mean a rotation. For a triangle, there are three distinct reflections,
\[ s = \{s_1, s_2, s_3 \}, \]
where $s_i$ is the reflection across the line $OA_1$. We can also rotate the triangle in two directions.

We know that these are all the symmetries, since we can count the permutations of the triangle. We've exhauseted then, so we know that there can't be any more elements of the triangle-symmetry group $D_6$. In fact, because of the permutation fact, we konw that $\abs{D_{2n}} = 2n$. Some other observations about $D_{2n}$:
\begin{itemize}
\item $s^2 = e \implies s = s^{-1}$;
\item rotating twice clockwise is the same as rotating counterclockwise, so these aren't unique elememnts;
\item $r^n = e$
\item $rs = s_2$, so $s_n$ is just a combination of $r$ and $s$ --- then we can generate the entire group with just $r$ and $s$.
\end{itemize}
These things lead us to discover a new object.

\begin{definition}[Generators]
For a group $G$, the generators of $G$ is a set $ S = \{a,b,\dotsc : a,b,\dotsc \in G\}$ where $G$ is equal to all possible sombinations of elements of $S$. For $D_{2n}$, we could say that $D_{2n}$ is generated by $r$ and $s$.
Usually there isn't a way to guess the generators of a group easily.
\end{definition}

\begin{definition}[Relations]
A relation is a way of writing equivalent elements of groups. For example, in $D_{2n}$,
\[ r^3 \equiv 1, \qquad s^2 \equiv 1, \qquad sr \equiv r^2s. \]
Relations allow us to define how we can commute elements of the group.
\end{definition}

\begin{definition}[Presentation]
A presentation of a group are the generators combined with the relations necessary to create the group. The largest group which is generated from the generators and which satisfies the relations, and has no other relations, is our group. A presentation is written as $\langle a,b \mid \text{relations between $a$ and $b$} \rangle$, where $a$ and $b$ are the generators of the group.
\end{definition}

Now Miki told us that the group of the  symmetires of a regular $n$-gon is the dihedral group of order $2n$, written either as \{$D_{2n}$ or $D_{n}$\}, depending on if you are a representation theorist or not.

\begin{problem}[HW]
Why is the order of $D_{2n}$ always $2n$?
\end{problem}

\subsection*{Symmetric group on $n$ elements}

Miki defined the symmetric group on $n$ elements $S_n$, which is just the permutations of $n$ elemnts. Notice that $D_{2n}$ is a subgroup of $S_n$. We know that the order of $S_n = n!$ and the order of $D_{2n} = 2n$.

[Insert diagrams of different ways to denote permuations, like the cycle notation]

\section{September 4, 2018}

\begin{definition}
For a set $\Omega$, the symmetric group on $\Omega$ is $S_\Omega = \{\text{bijective maps $\Omega \to \Omega$}\}$. For $n \in \N$, we say that $S_n = S_{\{1,\dotsc,n\}}$. This is usually called the symmetric group on $n$ letters.
\end{definition}

Let's consider this example for $S_4$ (warning, there's some cyclic decomposition for $g_1, g_2$?)

\begin{example}
Consider the following maps $g_1, g_2 \in S_4$,
\[
	\begin{array}{c}
	g_1 \\
	1 \to 2 \\
	2 \to 1 \\
	3 \to 4 \\
	4 \to 2
	\end{array} \qquad
	\begin{array}{c}
	g_2 \\
	1 \to 3 \\
	2 \to 1 \\
	3 \to 2 \\
	4 \to 4
	\end{array}
\]
We can also write these as $g_1 = (12)(34)$ and $g_2 = (132)(4)$. In this notation, how to we multiply things? E.g., what is $g_2g_1$? Well, we can write this naïvely as $(132)(4)(12)(34)$, but we don't want to repeat any numbers. Let's see what happens to $1$:
\[ (132)(4)(12)(34) \cdot 1 = (132)(4)(12) \cdot 1 = (132) \cdot 2 = 1. \] For $2$, we get 
\[ (132)(4)(12)(34) \cdot 2 = 3. \] For $3$, this comes $g_2g_1 \cdot 3 = 4$, and for $4$ we have $g_2g_1 \cdot 4 = 2$. Then $g_2g_1 = (1)(234)$. Unfortunately, doing this sort of element-wise reduction is the fastest way to multiply anything.
\end{example}

\begin{problem}
Someone asked the question ``does order matter?'' E.g., is it true that $(12)(34) = (34)(12)$ always?
\end{problem}
\begin{solution}
No. They are the same. Also, $(abc) = (bca)$; as long as the sign of the permutation of the cycle elements is $+1$, it won't matter how you order the elements of a cycle.
\end{solution}

\begin{problem}
Does order matter when there is a number repeated (when the cycles are not disjoint)? E.g. does $g_1g_2 = g_2g_1$?
\end{problem}
\begin{solution}
Yeah, order does matter. Consider that $(12)(13) \not= (13)(12)$. This means that, in general, $S_n$ is not abelian.
\end{solution}

\begin{problem}
Consider $S_5$, where $g = (123)(45)$ and $h = (12345)$. Find $g^2, g^{-1}, h^{-1}$. Fun fact, it's easy.
\end{problem}

These facts lead us to an interesting and useful conclusion.
\begin{proposition}
For any $g \in S_n$, we can write $g$ as a product of disjoint cycles.
\end{proposition}

This gives us an interesting observation for $S_n$.

\begin{proposition}
Let $g \in S_n$ be written as the product of disjoint cycles. Then the order of $g$ is the least common multiple of the orders of the disjoint cycles.
\end{proposition}

\subsection*{Fields $n$ stuff}

\begin{definition}
A field $k$ is a triple $(F, +, \times)$ where $(F,+)$ and $(F \setminus \{0\},\times)$ are groups where $F^\times = F \setminus \{0\}$ and where multiplication distributes over addition. Some cannonical examples are $\Q$, $\R$, $\C$, $\F_p = \Z/p\Z$ for prime $p$.
\end{definition}

A brief note on finite fields: for a finite field $\F$, we know that $\abs{\F} = p^n$ for some prime $p$ and some $n \geq 1$.

Now that we have fields, we can get matrices for free. Consider the cannonical matrix group $\mathrm{GL}_n(k)$ of invertible matrices with entries in $k$.

\begin{example}
Consider $\mathrm{GL}_2(\F_2)$ where $\F_2 = \{\bar{1}, \bar{2}\}$ (note that this is just $\Z / 2\Z$). What is the order of $\mathrm{GL}_2(\F_2)$?
\end{example}

\begin{proof}
There are six. Any element cannot have three or four zeros in it, nor two zeros in the same row or column. Then just count the total possibilities.
\end{proof}
% !TEX root = ../notes.tex

\section{Wednesday, January 23}

Recall the uniqueness of prime factorization, where for all $n \in \N$ we have a unique list of primes $p_1, \dotsc, p_k$ and $a_i, \dotsc, a_k \in \Z_{>0}$ such that $n = \prod_{i=1}^k p_i^{a_i}$.

\subsection{Infinitude of Primes}
\begin{problem}
How many primes are there?
\end{problem}

\begin{theorem}
There are infinitely many primes.
\end{theorem}

\begin{proof}[Euclid's Proof]
Assume by way of contradiction we have a finite list of primes $p_1, \dotsc, p_k$ of all primes. Let $M = \prod p_i$, and consider $M+1$. By the existence of prime factorization, we know that $M+1 = \prod_{i=1}^k p_i^{a_i}$. Without a loss of generality assume that $a_1 \not= 0$. Then $p_1$ divide $M+1$ and since $p_1$ divides $M$ it must be the case that $p_1$ divides $1$ as well which presents a contradiction.
\end{proof}

\textbf{Fact:} Let $p_1, p_2, p_3, \dotsc$ be a list of primes in order. By the uniqueness of prime factorization, there is an injective correspondence between vectors $(a_1, a_2, \dotsc) \in (\Z_{\geq0})^\infty$ with finitely many nonzero entries and $\N$. The correspondence is $n = \prod p_i^{a_i}$ with a lot of $a_i$ being zero.

\textbf{NOTE: THE BELOW IS BY CONTRADCTION and (*) ONLY HOLDS FOR $k=\infty$.}
If we assume that the list of primes is finite, then we would have an injective correspondence between $(a_1, \dotsc, a_k) \in (\Z_{\geq 0})^k$ and $\N$. Therefore
\[ \prod_{i=1}^k\qty(\sum_{j=0}^\infty \frac{1}{p_i}) = \sum_{n=1}^\infty \frac{1}{n}. \tag{*} \]
Then by uniqueness of prime factorization for each $n \in \N$ we know that $1/n$ appears exactly once when you expand this product. This is Euler's product for the $\zeta$ function?

\begin{proof}[Euler's Proof]
Assume by way of contradiction that there are finitely many primes. Then 
\[ \sum_{n=1}^\infty \frac{1}{n} = \prod_{i=1}^k\qty(1 + \frac{1}{p_1} + \cdots) = \prod_{i=1}^k \qty(\frac{1}{1-1/p_i}) < \infty. \]
Yet we know that $\sum 1/n$ diverges, which presents a contradiction.
\end{proof}

\begin{lemma}
For any $n \in \Z$ there exists a unique $a,b \in \Z$ such that $a$ is square free (meaning that no square number divides it) and $n = ab^2$.
\end{lemma}

\begin{proof}[Erdős' Proof]
Assume by way of contradiction that there are finitely many primes. Then any square-free number $n = \prod p_i^{a_i}$ where $a_i \in \{0,1\}$. Thus there are only $2^k$ square-free numbers. Now let's look at all numbers at most $N$ for some $N$. By the above lemma, they can be specified by $(a,b)$ where $a$ is square-free and $b^2$ is square. There are $2^k$ square-free numbers and at most $\sqrt{N}$ square numbers, so $N \leq 2^k\sqrt{N}$ for all $N$, so $2^k \geq \sqrt{N}$ for all $N$, which is very very false if $N > 2^{2k}$.
\end{proof}

\subsection{Congruence Equations \& Modular Arithmetic}

\begin{definition}[Congruence]
We say that $a \equiv b \pmod{m}$ if and only if $m$ divides $b-a$. Alternatively, we say that $a \equiv b \pmod{m}$ if and only if there exists some $k$ such that $a = b + mk$.
\end{definition}

\begin{theorem}[Some quick remarks] \hfill
\begin{enumerate}
\item Congruency is an equivalence relation on the integers (transitive, symmetric, and reflexive);
\item For some fixed $m$, we define the congruence class $\bar{a}$ to be the set $\bar{a} = \{n \in \Z \mid n \equiv a \pmod{m}\}$;
\item Arithmetic on these congruence classes holds; If $a \equiv b \pmod{m}$ and $c \equiv d \pmod{m}$ then $a+c \equiv b + d \pmod{m}$ and $ac \equiv bd \pmod{m}$. Thus $\bar{a} + \bar{b} = \overline{a + b}$ and $\bar{a}\bar{b} = \overline{ab}$. This forms the commutative ring $\Z/m\Z$.
\end{enumerate}
\end{theorem}

\begin{problem}
Let $a,b,m$ be fixed.
When is the congruence $ax \equiv b \pmod{m}$ solvable?
\end{problem}

\begin{enumerate}[label=\textbf{Obs. \arabic*.}]
\item If $(a,m)=1$ then we can use Bezout's theorem. This tells us that tehre exist some $X,Y$ such that $1 = aX + mY$. Then we multiply through by $b$ to get that $b = a(Xb) + m(Yb)$. Then $aXb \equiv b \pmod{m}$.
\end{enumerate}

\begin{lemma}
The congruence $ax \equiv b \pmod{m}$ has solutions if and only if the \textsc{gcd} of $a$ and $m$ divides $b$.
\end{lemma}

\begin{proof}
Let $d$ be the \textsc{gcd} of $a$ and $m$. By Bezout, there exists some $X_0, Y_0 \in \Z$ such that $d = aX_0 + bY_0$. Since $d$ divides $b$ there exists some $k$ such that $b = dk$. Then $b = aX_0k + mY_0k$ so $b \cong aX \pmod{m}$ for $X = X_0k$. In the other direction, just write it out. If there is a solution then $b \cong aX \pmod{m}$ so $b = aX + mY$. Since the \textsc{gcd} divides the right hand side it must divide the left as well, so $d$ divides $b$.
\end{proof}

\begin{problem}
Can there be lots of different solutions? What do solutions look like?
\end{problem}
% !TEX root = ../notes.tex

\section{Monday, January 28}

\begin{multicols}{2}
\begin{enumerate}
\item Degrees of Reality;
\item Formal Reality \& Objective Reality;
\item Mister Ed;
\item Two causal principles \& and the PSR;
\item Whodunit;
\item D.'s finest hour;
\item Not a deceiver;
\item Cartesian Circle;
\item Intellectual Problem of Evil;
\item Belief \& the Will.
\end{enumerate}
\end{multicols}

\subsection{Degrees of Reality}

Descartes' proof of God's existence hinges on different degrees of reality, which in some sense is a measure of independence; in Descartes' view, God (an infinite substance) ought to be independent on anything else while everything else ought to be dependent on God. Finite and/or extended substances (table, mind, Mr.\ Ed, etc.) all depend on God, and in turn the shape of a table depends on the table itself.

\begin{definition}[Formal Reality]
The reality something has by virtue of its existence. Usually a measure of how independent a thing is. Often also derives from the complexity of the thing; a machine would have more formal reality than a rock.
\end{definition}

\begin{definition}[Objective Reality]
The objective reality of an idea is equal to the formal reality of the object of the idea has, or would have if it existed. Only ideas can have objective reality.
\end{definition}

\begin{example}
Consider the idea of Mr.\ Ed. This has relatively low formal reality since it's just a though in a mind, but has relatively high objective reality since Mr.\ Ed himself has a relatively high formal reality. The idea of god has infinite objective reality.
\end{example}

\subsection{Two Causal Principles}

\begin{proposition}
The formal reality of a cause is greater than or equal to the formal reality of the effect.
\end{proposition}

\begin{proposition}
The formal reality of the cause is greater than or equal to the objective reality of the effect.
\end{proposition}

From this, God asks ``What causes my idea of God?''
\section{September 12, 2018}

\epigraph{``Oh, I erased my smiley face. How sad.'' (she did not sound sad)}{Miki}

Today we'll officially state something we covered last time.

\begin{theorem}[Caley's Theorem]
Every finite group $G$ is isomorphic to a subgroup of $S_n$ for some $n$.
\end{theorem}
\begin{proof}
Let $n = \abs{G}$.
\end{proof}

\subsection{Kernels}

Let's discuss formally the idea of a kernel of a homomorphism and a kernel of a group action.

\begin{definition}[Kernel]
Let $\phi : G \to H$ be a homomorphism. Then the kernel of $\phi$, written $\ker \phi$, is the set of all elements in $G$ which are mapped to the identity in $H$; i.e., $\ker \phi = \{g \mid \phi(g) = 1_h \}$.
\end{definition}

\begin{definition}
Suppose $G$ acts on $A$ by $\pi$. Then the kernel of the action is the set of all elements of $g$ which act trivially on $A$; i.e., $\ker \pi = \{g \mid ga = a \text{ for all $a \in A$}\}$.
\end{definition}

\begin{example}
Consider the action $\phi : \mathrm{GL}_2(\R) \to (\R^\times, \times) : A \mapsto \det A$. Then the kernel of $\phi$ are all matricies with determinant $1$, called $\mathrm{SL}_2(\R)$.
\end{example}

\begin{definition}[Stabalizer]
Let $\pi : G \times A \to A$ be a group action, and fix $a \in A$. The \emph{stabalizer} is $G_a = \{g \in G \mid ga = a\}$. By this definition, the kernel is contained within any stabalizer, and in fact is equal to the intersection of all stabalizers.
\end{definition}

\begin{example}
Let $G = \mathrm{GL}_2(\R)$ and let $A = \R^2$ defined with the usual action (vector-matrix multiplication). What is the kernel of this action? Then let $c = (0,1)^\top \in \R^2$. What is the stabalizer of $c$?
\end{example}

\begin{corollary}
The kernel of an action is a subgroup of $G$, and $G_a$ is a subgroup of $G$ for any fixed $a \in A$.
\end{corollary}

\begin{definition}[Orbit]
Fix $a \in A$. The orbit of $a$ is the image of $a$ under the group action; i.e., $O_a = \{ga \mid g \in G\}$. Intuitively, it's everywhere $a$ can go under a specific group action. Notice that the orbits partition $A$, and so are equivalence classes in $A$.
\end{definition}

\begin{example}
Let $G = \mathrm{GL}_2(\R)$ and let $A = \R^2$ defined with the usual action (vector-matrix multiplication). What is the orbit of $(1,0)^\top$?
\end{example}

\begin{definition}[Faithful]
An action is faithful if the kernel is the identity. This means that the base element of the action must be the identity. This tells us that $G$ is injective into $S_A$.
\end{definition}

\begin{example}
Consider $D_8$ acting on a square (technically the set $A = \{1,2,3,4\}$). The orbit $O_1$ is all possible vertices, since you can rotate any vertex to any position. The stabalizer is $\{1, s\}$.
\end{example}

\begin{lemma}
As it turns out, for a fixed $a \in A$, we see that $\abs{O_a}\abs{G_a} = \abs{G}$. We'll prove this later. (Orbit-Stabalizer Theorem I think?)
\end{lemma}

\begin{definition}[Conjugation]
Consider the action $\pi : G \times G \to G : (g,a) \mapsto gag^{-1}$. This action is known as \emph{conjugation}.
\end{definition}

\begin{definition}[Centralizer]
Let $S \subset G$. The \emph{centralizer} of $S$ in $G$, written $C_G(S) = \{g \in G \mid gsg^{-1} = s \text{ for all $s \in S$}\}$. This is the set of things that fix $S$ in $G$ pointwise under conjucation. By definition, this is the set of elements in $G$ which commute with all elements in $S$. In the case that $S = \{s\}$ we see that $C_G(S) = G_S$. 
\end{definition}

\begin{definition}[Normalizer]
Let $S \subset G$. The \emph{normalizer} of $S$ in $G$ is $N_G(S) = \{g \in G \mid gSg^{-1} = S\}$. Essentially, this is just a centralizer on a set, except that it may permute the elements of $S$. Then $C_G(S) \subset N_G(S)$.
\end{definition}

\begin{example}
Suppose that $G$ is abelian. For any $S \subset G$, we see that $C_G(S) = N_G(S) = G$.
\end{example}

\begin{example}
Let $G = S_3$, and let $S = G$. What is the normalizer of $S$? (It's the whole thing since $G$ is closed under its operation.) What is the centralizer of $S$? (It's the identity.)
\end{example}
\section{September 14, 2018}

\epigraph{``Why do we get struck by lightning when we reach a contradiciton? I don't know, it's usually a good thing.'' \lightning}{Miki}

\begin{definition}[Center]
The center of a group $G$ is $Z(G) = \{g \in G \mid gs = sg \text{ for all $s \in G$}\}$; i.e., $Z(G) = C_G(G)$, so it's the centralizer of the whole group.
\end{definition}

Why do we care so much about conjugation? We give all these special names to the sets of conjugation, like the Normalizer, Stabalizer, and Centralizer. We also know that conjugation preserves the order of an element, so $\abs{a} = \abs{gag^{-1}}$.

\begin{problem}
What is the center of $D_8$? We know the identity must be in the center. What about $r^2$? We know it commutes with $s$, and $sr^2 = r^{-2}s = r^2s$, so it commutes with $s$ as well; Since $r$ and $s$ generate the group, we know that it be in the center as well. So $Z(D_8) = \{1, r^2\}$.
\end{problem}

\subsection{Cyclic Groups}

\begin{proposition}
Let $G$ be a group, and let $x \in G$. For $m,n \in \Z$, if $x^n = x^m = 1$ then $x^d = 1$ where $d = \gcd(m,n)$.
\end{proposition}
\begin{proof}
Use the Euclidean Algorithm. We know there are integers $a,b \in \Z$ where $d = am + bn$, so $x^d = x^{am+bn} = (x^a)^m (x^b)^n = 1^a1^b = 1$.
\end{proof}
\begin{corollary}
If $x^m = 1$ then $\abs{x}$ divides $m$ if $m$ is finite.
\end{corollary}
\begin{proof}
If $m = 0$, we are done since everything divies zero. Assume that $1 \leq m < \infty$. Let $n = \abs{x} \leq m < \infty$ be finite. Let $d = \gcd(m,n)$, so $x^d = 1$. We know that $d$ divides $n$, and since $n$ is the smallest power of $x$ to be the identity, we know that $d = n$. \emph{A priori,} we know that $d$ divides $m$ so $d$ must divide $m$ as well.
\end{proof}
\begin{proposition}
Let $x \in G$, and let $a \in \Z \setminus \{0\}$.
\begin{enumerate}
\item If $\abs{x} = \infty$, then $\abs{x^a} = \infty$;
\item If $\abs{x} = n < \infty$, then $\abs{x^a} = n / \gcd(a,n)$.
\end{enumerate}
\end{proposition}
\begin{proof}
The proof of (1) is ommitted, and left as an exercise to the student. For (2), let's focus on the special case that $a$ divides $n$. If $x^n = 1$ then $(x^a)^{n/a} = x^n = 1$. Then $\abs{x^a}$ is at most $n/a$. Suppose by way of contraction that the order $d$ is strictly less than $n / a$. Then $x^{ad} = 1 \implies 1 \leq ad < n$, but $\abs{x} = n$. This is a contradiction, so the order of $x^a$ must be exactly $n/a$. In the case that $a$ does not divide $n$, play around with this to get the more general conclusion (the logic is the same).
\end{proof}

\begin{definition}[Cyclic Group]
A group $G$ is cyclic if there exists an $x \in G$ such that $G = \langle x \rangle$. As a note, it's not always easy to tell since there could be other presentaitons of a group which are not single elements. Always remember that presentations are not unique.
\end{definition}

\begin{problem}
Let $G = \langle a,b \mid a^2 = b^3 = 1, ab = ba \rangle$. Show that $G$ is cyclic.
\end{problem}

\begin{corollary}
All cyclic groups must be abelian, since any $g \in G$ is generated by some $x^a$, and $x$ always commutes with itself.
\end{corollary}

\begin{example}[Infinite Cyclic Groups]
Throughout, let $G = \langle x \rangle$, and assume that $\abs{x} = \infty$.

\begin{proposition}
The order of $G$ is $\infty$. Then $x^m \not= x^n$ for all distinct $m,n \in \Z$.
\end{proposition}

\begin{proof}
Let $m < n$. Suppose by way of contradiction that $x^m = x^n$. Then $x^{n-m} = 1$, which cannot happen since $n-m > 0$ and $\abs{x} = \infty$. Then $\abs{G} = \infty$.
\end{proof}

\begin{proposition}
Such $G$ must be isomorphic to $(\Z, +)$.
\end{proposition}
\begin{proof}
Define a map $\phi : \Z \to G : n \mapsto x^n$. This map is well defined. It also respects multiplication since $m+n \mapsto x^mx^n$. It is injective by Proposition~1, and it is surjective by Proposition~1 since $G$ is generated completely by $x$. Then $\phi$ is an isomorphism.
\end{proof}

\begin{proposition}
Such a group $G$ is generated by $x^n$ if and only if $n = \pm 1$.
\end{proposition}
\begin{proof}
Left as an exercise to the student.
\end{proof}

\begin{proposition}
Every subgroup of $G$ is cyclic of the form $H = \langle x^n \rangle$ for some $n \in \Z$.
\end{proposition}
\begin{proof}
Suppose that $x^n = 1$. Then $H$ is obviously cyclic. On the other hand, if $H \not= \langle 1 \rangle$. Let $n = \min\{k > 0 \mid x^k \in H\}$. This can't be empty, so there is an $n$. Then $\langle x^n \rangle \subset H$. Take some other element $x^m \in H$, and let $d = \gcd(m,n) = am + bn$. Then $x^d = (x^m)^a(x^n)^b \in H$ but $1 \leq d \leq n$, so $d = n$. Then $n$ divides $m$, and so $x^m \in \langle x^n \rangle$. The $\langle x^n \rangle = H$, and so $H$ is cyclic.
\end{proof}
\end{example}
\begin{corollary}
Every non-trivial subgroup of $\Z$ is isomorphic to $\Z$.
\end{corollary}
\begin{corollary}
For some cyclic $G$, we know that $\langle x^n \rangle = \langle x^{-n} \rangle \subset G$. Then all non-trivial subgroups correspond to $\Z_{>0}$.
\end{corollary}

\section{September 17, 2018}

\epigraph{``The only thing I learned for years was how to count hedgehogs in a field.'' (In the midst of a wonderful and inspirational talk about being a mathematician.)}{Miki}

\subsection{Finite Cyclic Groups}

Today, we'll cover finite cyclic groups. This will be very similar to the previous lecture on infinite cyclic groups. As a reminder, here are the propositions for infinite cyclic groups:

\begin{proposition}[Infinite Cyclic Groups]
Let $G$ be an infinite cyclic group.
\begin{enumerate}
\item The order of $G$ is infinite, with $G = \{\dotsc,x^{-1},1,x,x^2,\dotsc\}$ all distinct.
\item The group $G$ is isomorphic to $\Z$.
\item The group $G$ is generated by $x^n$ if and only if $n = \pm 1$.
\item Every subgroup $G$ is cyclic.
\end{enumerate}
\end{proposition}

Now for the finite case. 

\begin{proposition}[Finite Cyclic Groups]
 Let $G = \langle x \rangle$ with $\abs{G} = n < \infty$.
 \begin{enumerate}[label=P\arabic*.]
 \item The group $G$ is exactly $\{1,x,\dotsc,x^{n-1}\}$.
 \item The group $G$ is isomorphic to $\Z/n\Z$.
 \item The group $G$ is generated by $x^k$ if and only if $\gcd(k,n) = 1$.
 \item Every subgroup of $G$ is also cyclic. That is, for all $k > 0$ where $k$ divides $n$ we get a subgroup $H$ of order $k$ generated by $x^{n/k}$
 \end{enumerate}
\end{proposition}
\begin{proof}[Proof of~P1]
We know that $1,\dotsc,x^{n-1}$ are all in $G$. Suppose that $x^a = x^b$ for some distinct $a,b$. Then $x^{b-a} = 1$ for $0 < b-a < n$, which is a contradiciton since $\abs{x} = n$. In fact, this set enumerates $G$. Suppose that $x^k \in G$ for some $k \in \Z$. We use the divison algorithm to write that $k = an + r$ for some $a,r \in \Z$. Then $x^k = x^{an+r} = (x^n)^ax^r = x^r$, so $x^k$ is in $G$.
\end{proof}
\begin{proof}[Proof of~P2]
Let $\phi : \Z/n\Z \to G : \bar{k} \mapsto x^k$ where $k$ is any representative of $\bar{k} \in \Z$. To show that $\phi$ is well-defined, consider another representative $\ell$ of $\bar{k} \in \Z$. Then $\ell = k + an$, so $x^\ell = x^{k+an} = x^k(x^n)^a = x^k$. To show that $\phi$ is a homomorphism, consider that $\phi(\bar{m}+\bar{n}) = x^{m+n} = x^{m}x^n$, so $\phi$ is multiplicative. Finally, we know that $\phi$ is surjective and injective by P1. This tells us that, up to an isomorphism, there are only really two cyclic groups; $\Z$ if the group is of infinite order, or $\Z/n\Z$ if it is finite.
\end{proof}
\begin{proof}[Proof of~P3]
This is more of a sketch. Recall that $\langle \abs{x^k}\rangle = \abs{x^k}$, and this is $n$ if and only if $\gcd(k,n) = 1$. In general, $\abs{x^k} = n / \gcd(k,n)$.
\end{proof}
\begin{proof}[Proof of~P4]
Exactly the same as the infinite case.
\end{proof}

Now that we've covered cyclic groups, it's helpful to introduce some notation to represent them.
\begin{notation}[$\Z_n, C_n$]
We write the multiplicative cyclic group of order $n$ as $\Z_n$. The additive cyclic group of order $n$, which we've been writing as $\Z/n\Z$, is commonly written as $C_n$.
\end{notation}

\subsection{Subgroups}

\cite{DF} uses the notation $S \subset G$ to mean that $S$ is a subset of $G$, and $H \leq G$ to mean that $H$ is a subgroup of $G$.

\begin{definition}[Subgroup]
Let $S \subset G$ be nonempty. Let $H = \{a_1^{\varepsilon_1}\cdots a_k^{\varepsilon_k}\}$ where $a_i \in S$ and $\varepsilon_i = \pm 1$ for $k \in \Z_{\geq 0}$. This sequence of $a_i^{\varepsilon_i}$ is called a \emph{word}. Note that $a_i$ need not be distinct. Then $H$ is a subgroup of $G$.
\end{definition}

\begin{proof}
First, we know that $H \subset G$. We know that $1 \in H$. Since any concatenation of words is also a word, we know that $H$ is closed under multiplication. Finally, since $((ab)^n)^{-1} = b^{-n}a^{-n}$ we know that the inverses of a word are words themselves, and so $H$ is closed under inversion.
\end{proof}
\section{September 19, 2018}

\epigraph{``Funny things happen with groups, which is why they're fun!''}{Miki}

Recall from last time how we defined a subgroup $H$ of $G$ in terms of words where the powers of each element was $\pm 1$. If $G$ is abelian we can combine elements of like bases to get powers which can be any integral value. If we assume that $\abs{a_i} = d_i$ is finite for all $a_i \in H$, then we know that $\abs{H} \leq d_1 \cdots d_k$. This gives us a limit on the order of a subgroup; if $G$ is abelian then the order of a subgroup is bounded above by the product of the orders of the generating elements. On the other hand, if $G$ is not abelian then this does not always hold. Consider $G = \langle a,b \mid a^2 = b^2 = 1 \rangle$. If $G$ isn't commutative, then $(ab)^n \not= a^nb^n$ for all $n$ and so we can just create infinitely many words by appending $ab$ to one another and so the order is infinite.

\begin{lemma}
Let $G = \{a_1^{n_1} \cdots a_k^{n_k}\}$ be abelian, and let each $a_i$ have finite order $d_i$. Then $\abs{G} \leq d_1 \cdots d_i$.
\end{lemma}

\begin{proposition}
Let $G$ be a group and let $\mathcal{L}$ be a collection of subgroups of $G$. Then \[ K = \bigcap_{L \in \mathcal{L}} L \]
is a subgroup of $G$.
\end{proposition}

\begin{definition}[Subgroup]
Let $S \subset G$ and let $\mathcal{L} = \{L \leq G \mid s \subset L\}$. Then the subgroup generated by $S$ is \[ K = \bigcap_{L \in \mathcal{L}} L. \]
\end{definition}

What do we know from this definition? Well, $S \subset K$ and $K \leq G$. We want to say that $K \in \mathcal{L}$ is the minimal element, so $K = L_i$ for some $i$.

\begin{definition}[Minimal Element]
Let $\mathcal{M}$ be a collection of subsets of $G$. A minimal element is an element $M$ of $\mathcal{M}$ such that if $M' \in \mathcal{M}$ and $M' \subset M$ then $M = M'$. It's like ``the smallest element'' except there could be multiple minimal elements.
\end{definition}

We want to show that $K$ is \emph{the} minimal element of $\mathcal{L}$.

\begin{proof}[Proof $K$ is minimal]
Let $L \in \mathcal{L}$. Then $K \subset L$. Then either $K = L$ or $L$ is not minimal.
\end{proof}

\begin{proof}[Proof $K$ is \emph{the} minimal element]
Suppose there is another minimal $M$ in $\mathcal{L}$. By definition $K \subset M$ so by minimality $M = K$.
\end{proof}

\begin{proposition}
Our two definitions for subgroup (generated by words $H$ and minimal element of collection $K$) containing elements of $S \subset G$ are equivalent.
\end{proposition}

\begin{proof}
$H \leq G$ and $S \subset H$ by the construction of $1$-letter words. Then $H \in \mathcal{L}$ so $K \subset H$. On the other hand, $S \subset K$ and $K$ is a group. Then $K$ contains all inverses and products of elements in $S$, so it contains all words and therefore contains $H$. Then $H \subset K$. Putting this together, we have that $K = H$.
\end{proof}

\begin{definition}[Lattice]
Given a group $G$, a lattice is a diagram showing all subgroups of $G$ which shows containment between the subgroups.
\end{definition}

\begin{figure}[h]
	\caption{Lattice Diagram of $C_2$}
\end{figure}

Recall from last time that for $C_n$ the subgroups are paired with the divisors $k$ of $n$; then $\langle k \rangle$ generates subgroup of order $n / k$.

\begin{figure}[ht]
	\caption{Lattice Diagram of $C_4$}
\end{figure}

\begin{figure}[ht]
	\caption{Lattice Diagram of $C_8$}
\end{figure}

\begin{figure}[ht]
	\caption{Lattice Diagram of $C_6$}
\end{figure}

\begin{figure}[ht]
	\caption{Lattice Diagram of $S_3$}
\end{figure}

\end{document}