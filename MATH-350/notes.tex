\documentclass{notes}

% Bibliography
\usepackage[
	style = alphabetic ,
]{biblatex}
\addbibresource{references.bib}

% Document Information
\title{Introduction to Abstract~Algebra}
\courseid{MATH 350}
\place{Yale University}
\term{Fall}
\year{2018}

\blurb{
	These are lecture notes for MATH 350a, ``Introduction to Abstract Algebra,'' taught by Marketa Havlickova at Yale University during the fall of 2018.
	These notes are not official, and have not been proofread by the instructor for the course.
	These notes live in my lecture notes respository at 
	\[\text{\url{https://github.com/jopetty/lecture-notes/tree/master/MATH-350}.}\] 
	If you find any errors, please open a bug report describing the error, and label it with the course identifier, or open a pull request so I can correct it.
}

\begin{document}

\section*{Syllabus}

\begin{center}
\begin{tabular}{@{}rp{10cm}@{}}
\toprule 
\textbf{Instructor} & Marketa Havlickova, \url{miki.havlickova@yale.edu} \\
\textbf{Lecture} & MWF 10:30--11:20 \textsc{am}, LOM 205 \\
\textbf{Recitation} & TBA \\
\textbf{Textbook} & \fullcite{DF} \\
\textbf{Midterms} & Wednesday, October 10, 2018 \\
& Wednesday, November 14, 2018 \\
\textbf{Final} & Monday, December 17, 2018, 2:00--5:30 \textsc{pm} \\
\bottomrule
\end{tabular} \\[3ex]
\end{center}

Abstract Algebra is the study of mathematical structures carrying notions of ``multiplication'' and/or ``addition''. Though the rules governing these structures seem familiar from our middle and high school training in algebra, they can manifest themselves in a beautiful variety of different ways. The notion of a group, a structure carrying only multiplication, has its classical origins in the study of roots of polynomial equations and in the study of symmetries of geometrical objects. Today, group theory plays a role in almost all aspects of higher mathematics and has important applications in chemistry, computer science, materials science, physics, and in the modern theory of communications security. The main topics covered will be (finite) group theory, homomorphisms and isomorphism theorems, subgroups and quotient groups, group actions, the Sylow theorems, ring theory, ideals and quotient rings, Euclidean domains, principal ideal domains, unique factorization domains, module theory, and vector space theory. Time permitting, we will investigate other topics. This will be a heavily proof-based course with homework requiring a significant investment of time and thought. The course is essential for all students interested in studying higher mathematics, and it would be helpful for those considering majors such as computer science and theoretical physics.

Your final grade for the course will be determined by
\[ \max\left\{
	\begin{array}{cccc}
		25\%\text{ homework} + 20\%\text{ exam 1} + 20\%\text{ exam 2} + 35\%\text{ final} \\
		25\%\text{ homework} + 10\%\text{ exam 1} + 20\%\text{ exam 2} + 45\%\text{ final} \\
		25\%\text{ homework} + 20\%\text{ exam 1} + 10\%\text{ exam 2} + 45\%\text{ final}
	\end{array}
\right\}. \]

\printbibliography

\end{document}