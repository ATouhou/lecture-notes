\documentclass{lnotes}

\usepackage{tikz-cd}

\DeclareMathOperator{\lcm}{lcm}

% Bibliography
\usepackage[
	style = alphabetic ,
]{biblatex}
\addbibresource{references.bib} 

\usepackage{epigraph}
\usepackage{wasysym}
\usepackage[osf]{libertineRoman}

% Document Information
\title{Introduction to Abstract~Algebra}
\course{\textsc{math} 350}
\place{Yale University}
\term{Fall}
\year{2018}

\blurb{
	These are lecture notes for \textsc{math} 350a, ``Introduction to Abstract Algebra,'' taught by Marketa Havlickova at Yale University during the fall of 2018.
	These notes are not official, and have not been proofread by the instructor for the course.
	These notes live in my lecture notes respository at 
	\[\text{\url{https://github.com/jopetty/lecture-notes/tree/master/MATH-350}.}\] 
	If you find any errors, please open a bug report describing the error, and label it with the course identifier, or open a pull request so I can correct it.
}

\begin{document}

\section*{Syllabus}

\begin{center}
\begin{tabular}{@{}rp{10cm}@{}}
\toprule 
\textbf{Instructor} & Marketa Havlickova, \url{miki.havlickova@yale.edu} \\
\textbf{Lecture} & MWF 10:30--11:20 \textsc{am}, LOM 205 \\
\textbf{Recitation} & TBA \\
\textbf{Textbook} & \fullcite{DF} \\
\textbf{Midterms} & Wednesday, October 10, 2018 \\
& Wednesday, November 14, 2018 \\
\textbf{Final} & Monday, December 17, 2018, 2:00--5:30 \textsc{pm} \\
\bottomrule 
\end{tabular} \\[3ex]
\end{center}

Abstract Algebra is the study of mathematical structures carrying notions of ``multiplication'' and/or ``addition''. Though the rules governing these structures seem familiar from our middle and high school training in algebra, they can manifest themselves in a beautiful variety of different ways. The notion of a group, a structure carrying only multiplication, has its classical origins in the study of roots of polynomial equations and in the study of symmetries of geometrical objects. Today, group theory plays a role in almost all aspects of higher mathematics and has important applications in chemistry, computer science, materials science, physics, and in the modern theory of communications security. The main topics covered will be (finite) group theory, homomorphisms and isomorphism theorems, subgroups and quotient groups, group actions, the Sylow theorems, ring theory, ideals and quotient rings, Euclidean domains, principal ideal domains, unique factorization domains, module theory, and vector space theory. Time permitting, we will investigate other topics. This will be a heavily proof-based course with homework requiring a significant investment of time and thought. The course is essential for all students interested in studying higher mathematics, and it would be helpful for those considering majors such as computer science and theoretical physics.

Your final grade for the course will be determined by
\[ \max\left\{
	\begin{array}{cccc}
		25\%\text{ homework} + 20\%\text{ exam 1} + 20\%\text{ exam 2} + 35\%\text{ final} \\
		25\%\text{ homework} + 10\%\text{ exam 1} + 20\%\text{ exam 2} + 45\%\text{ final} \\
		25\%\text{ homework} + 20\%\text{ exam 1} + 10\%\text{ exam 2} + 45\%\text{ final}
	\end{array}
\right\}. \]

\printbibliography

%% Sanskrit
\section{August 30, 2018}

Linguistics was the model science in ancient india

Today, we'll talk about phonology --- range of sounds, how they're identified.
Phonology is the sound structure of a language, as opposed to syntax or semantics. For the time being we'll transliterate Sanskrit into the Latin script; tomorrow, we'll move on to devanagari. Sanskrit has been written in a variety of scripts, but devanagari is the most commonly used.
Much of the historic writings in Sanskrit are written in Devanagari or one of its predecessors.

Sanskrit's phonological system is very ``scientific'' --- it's very systematic and structured. Grammarians divide sounds in Sanskrit into two groups: the \emph{svara}, or vowels; and the \emph{vyañjana}, or consonants.

\subsection*{Svara}
In Sanskrit, svara (meaning sound, tone, or accent) are sounds which can be pronounced independently, contrasting with consonants which have an implied vowel associated with them.
There are 14 vowels in Sanskrit (really only 13) which are organized by two orthogonal qualities: the place of articulation and the length.

\begin{wrapfigure}{l}{0.25\textwidth}
\begin{tabular}{@{}cc@{}}
\toprule
Hrasva & Dīrga \\
\midrule
a & ā \\
i & ī \\
u & ū \\
ṛ & ṝ \\
ḷ & \emph{ḹ}\footnote{This vowel is never attested, and only exists to complete the chart} \\
\midrule
e & ai \\
o & au \\
\bottomrule
\end{tabular}
\end{wrapfigure}

Long and short vowels are distinguished by the number of \emph{mātrā} they consume.
You can think of mātrā as a kind of tempo marking for the language; short vowels, or \emph{hrvasa}, last for ``one beat'' of time, while long vowels, or \emph{dīrga}, last for two. In ancienct Vedic Sanskrit, there was a third category of vowels, called protracted vowels or \emph{pluta}, which lasted for three beats, but these were lost in Classical Sanskrit.
Vowels can also be simple (\emph{śuddha}, or ``pure''), or complex (\emph{samyukta}, or ``joined''). Samyukta are known in linguistics as diphthongs, and are made from the combination of two or more śuddha. In the table on the left, the samyukta are the bottom four.

Here are some examples of the vowels in Sanskrit words:

\begin{figure}[h]
\begin{center}
\begin{tabular}{@{}ll@{}}
\toprule
Abhava (absense, destruction) & Ātman (self) \\
Idam (this) & Īśvara (lord) \\
Ubhaya (both) & Abhūt (was) \\
Kṛṣṇa (Krishna) & Nṝṇām\footnote{The consonant ṝ is very rare, appearing only in nominative and genetive nouns} (Of men) \\
Kḷpti (arrangement) & --- \\
Etad (this) & Bhojana (food) \\
Chaitanya (a saint) & Saubhagya (good fortune) \\
\bottomrule
\end{tabular}
\end{center}
\end{figure}

\subsection*{Consonants}

 There are three kinds of consonants in SKRT. Consontants are classified by the place of articulation (sthāna):
 	- Velar or kan.t.hya
 	- soft pallette, palatal (talva)
 	- hard pallatte, retroflex (murtanyo)
 	- dental (danta)
 	- labial (os.t.ha)

 Also classified by the closure of the lips
 	- full contact, stops (sparśa)

 Voiced/Unvoiced
 	- ghośavat (voiced)
 	- aghośa 	(unvoiced)
 
 Apsiration
 	- in english, spin vs pin, cut vs king
 	- Alpha prana 	(unaspirated)
 	- maha prana	(aspirated)


\section{August 31, 2018}

As always, Miki began class at precicely 10:25 \textsc{am}. She wrote a review of last lecture on the bard, and then posed the following question as a warm up. She also talked about how the DUS department is arguing over whether money should be spent on T-shirts or chocolate (Miki thinks chocolate).

\begin{problem}[Warm Up]
Are these groups?
\begin{parts}
\part[w:1a] $(\Z/n\Z, \times)$;
\part[w:1b] $(\Z/n\Z \setminus \{0\}, \times)$
\end{parts}
\end{problem}

\begin{solution}
The solutions to the warm-up
\begin{subproof}[Solution to~\ref{w:1a}]
No, since $0$ has no inverse.
\end{subproof}

\begin{subproof}[Solution to~\ref{w:1b}]
No, this only works when $n$ is prime. For any factors $a,b$ of $n$, $a \times b = 0$, which isn't in the group. We say that $(\Z/p\Z, \times)$ is a group for all prime $p$.
\end{subproof}
\end{solution}

\begin{theorem}[Fermat's little theorem]
For prime $p$ and composite $a = np$, hen $a^{p-1} \equiv 1 \pmod{p}$. 
\end{theorem}

\begin{lemma}
If $\bar{a} \in \Z/p\Z \setminus \{0\}$, then $\bar{a}$ has an inverse in $(\Z/p\Z \setminus \{0\}, \times)^*$.
\end{lemma}

\begin{definition}[Units]
A unit is something which has an inverse. The units of a group are denoted by putitng an asterisk after teh group, eg $(\Z/p\Z \setminus \{0\}, \times)^*$.
\end{definition}

\begin{example}
For integers modulo 4, $(\Z/4\Z, \times)^* = \{\bar{1}, \bar{3}\}$.
\end{example}

\begin{problem}[On Homework]
What are the conditions for determining the units of a group? We know it must have an inverse, but that's hard to check. Instead, we know that $a$ is a unit if and only if $\gcd(a,n) = 1$. Prove this.
\end{problem}

\subsection*{Symmetries of a regular $n$-gon}
Miki is angry with the book because she doesn't like how it treats symmetries, I think because she wants $D_{2n}$ to be called $D_n$.

Miki drew a triangle on the board, and began talking about the different operations we can preform on that triangle to preserve symmetries. She introduced $s$ to mean a reflection, and $r$ to mean a rotation. For a triangle, there are three distinct reflections,
\[ s = \{s_1, s_2, s_3 \}, \]
where $s_i$ is the reflection across the line $OA_1$. We can also rotate the triangle in two directions.

We know that these are all the symmetries, since we can count the permutations of the triangle. We've exhauseted then, so we know that there can't be any more elements of the triangle-symmetry group $D_6$. In fact, because of the permutation fact, we konw that $\abs{D_{2n}} = 2n$. Some other observations about $D_{2n}$:
\begin{itemize}
\item $s^2 = e \implies s = s^{-1}$;
\item rotating twice clockwise is the same as rotating counterclockwise, so these aren't unique elememnts;
\item $r^n = e$
\item $rs = s_2$, so $s_n$ is just a combination of $r$ and $s$ --- then we can generate the entire group with just $r$ and $s$.
\end{itemize}
These things lead us to discover a new object.

\begin{definition}[Generators]
For a group $G$, the generators of $G$ is a set $ S = \{a,b,\dotsc : a,b,\dotsc \in G\}$ where $G$ is equal to all possible sombinations of elements of $S$. For $D_{2n}$, we could say that $D_{2n}$ is generated by $r$ and $s$.
Usually there isn't a way to guess the generators of a group easily.
\end{definition}

\begin{definition}[Relations]
A relation is a way of writing equivalent elements of groups. For example, in $D_{2n}$,
\[ r^3 \equiv 1, \qquad s^2 \equiv 1, \qquad sr \equiv r^2s. \]
Relations allow us to define how we can commute elements of the group.
\end{definition}

\begin{definition}[Presentation]
A presentation of a group are the generators combined with the relations necessary to create the group. The largest group which is generated from the generators and which satisfies the relations, and has no other relations, is our group. A presentation is written as $\langle a,b \mid \text{relations between $a$ and $b$} \rangle$, where $a$ and $b$ are the generators of the group.
\end{definition}

Now Miki told us that the group of the  symmetires of a regular $n$-gon is the dihedral group of order $2n$, written either as \{$D_{2n}$ or $D_{n}$\}, depending on if you are a representation theorist or not.

\begin{problem}[HW]
Why is the order of $D_{2n}$ always $2n$?
\end{problem}

\subsection*{Symmetric group on $n$ elements}

Miki defined the symmetric group on $n$ elements $S_n$, which is just the permutations of $n$ elemnts. Notice that $D_{2n}$ is a subgroup of $S_n$. We know that the order of $S_n = n!$ and the order of $D_{2n} = 2n$.

[Insert diagrams of different ways to denote permuations, like the cycle notation]

\section{September 4, 2018}

\begin{definition}
For a set $\Omega$, the symmetric group on $\Omega$ is $S_\Omega = \{\text{bijective maps $\Omega \to \Omega$}\}$. For $n \in \N$, we say that $S_n = S_{\{1,\dotsc,n\}}$. This is usually called the symmetric group on $n$ letters.
\end{definition}

Let's consider this example for $S_4$ (warning, there's some cyclic decomposition for $g_1, g_2$?)

\begin{example}
Consider the following maps $g_1, g_2 \in S_4$,
\[
	\begin{array}{c}
	g_1 \\
	1 \to 2 \\
	2 \to 1 \\
	3 \to 4 \\
	4 \to 2
	\end{array} \qquad
	\begin{array}{c}
	g_2 \\
	1 \to 3 \\
	2 \to 1 \\
	3 \to 2 \\
	4 \to 4
	\end{array}
\]
We can also write these as $g_1 = (12)(34)$ and $g_2 = (132)(4)$. In this notation, how to we multiply things? E.g., what is $g_2g_1$? Well, we can write this naïvely as $(132)(4)(12)(34)$, but we don't want to repeat any numbers. Let's see what happens to $1$:
\[ (132)(4)(12)(34) \cdot 1 = (132)(4)(12) \cdot 1 = (132) \cdot 2 = 1. \] For $2$, we get 
\[ (132)(4)(12)(34) \cdot 2 = 3. \] For $3$, this comes $g_2g_1 \cdot 3 = 4$, and for $4$ we have $g_2g_1 \cdot 4 = 2$. Then $g_2g_1 = (1)(234)$. Unfortunately, doing this sort of element-wise reduction is the fastest way to multiply anything.
\end{example}

\begin{problem}
Someone asked the question ``does order matter?'' E.g., is it true that $(12)(34) = (34)(12)$ always?
\end{problem}
\begin{solution}
No. They are the same. Also, $(abc) = (bca)$; as long as the sign of the permutation of the cycle elements is $+1$, it won't matter how you order the elements of a cycle.
\end{solution}

\begin{problem}
Does order matter when there is a number repeated (when the cycles are not disjoint)? E.g. does $g_1g_2 = g_2g_1$?
\end{problem}
\begin{solution}
Yeah, order does matter. Consider that $(12)(13) \not= (13)(12)$. This means that, in general, $S_n$ is not abelian.
\end{solution}

\begin{problem}
Consider $S_5$, where $g = (123)(45)$ and $h = (12345)$. Find $g^2, g^{-1}, h^{-1}$. Fun fact, it's easy.
\end{problem}

These facts lead us to an interesting and useful conclusion.
\begin{proposition}
For any $g \in S_n$, we can write $g$ as a product of disjoint cycles.
\end{proposition}

This gives us an interesting observation for $S_n$.

\begin{proposition}
Let $g \in S_n$ be written as the product of disjoint cycles. Then the order of $g$ is the least common multiple of the orders of the disjoint cycles.
\end{proposition}

\subsection*{Fields $n$ stuff}

\begin{definition}
A field $k$ is a triple $(F, +, \times)$ where $(F,+)$ and $(F \setminus \{0\},\times)$ are groups where $F^\times = F \setminus \{0\}$ and where multiplication distributes over addition. Some cannonical examples are $\Q$, $\R$, $\C$, $\F_p = \Z/p\Z$ for prime $p$.
\end{definition}

A brief note on finite fields: for a finite field $\F$, we know that $\abs{\F} = p^n$ for some prime $p$ and some $n \geq 1$.

Now that we have fields, we can get matrices for free. Consider the cannonical matrix group $\mathrm{GL}_n(k)$ of invertible matrices with entries in $k$.

\begin{example}
Consider $\mathrm{GL}_2(\F_2)$ where $\F_2 = \{\bar{1}, \bar{2}\}$ (note that this is just $\Z / 2\Z$). What is the order of $\mathrm{GL}_2(\F_2)$?
\end{example}

\begin{proof}
There are six. Any element cannot have three or four zeros in it, nor two zeros in the same row or column. Then just count the total possibilities.
\end{proof}
% !TEX root = ../notes.tex

\section{Wednesday, January 23}

Recall the uniqueness of prime factorization, where for all $n \in \N$ we have a unique list of primes $p_1, \dotsc, p_k$ and $a_i, \dotsc, a_k \in \Z_{>0}$ such that $n = \prod_{i=1}^k p_i^{a_i}$.

\subsection{Infinitude of Primes}
\begin{problem}
How many primes are there?
\end{problem}

\begin{theorem}
There are infinitely many primes.
\end{theorem}

\begin{proof}[Euclid's Proof]
Assume by way of contradiction we have a finite list of primes $p_1, \dotsc, p_k$ of all primes. Let $M = \prod p_i$, and consider $M+1$. By the existence of prime factorization, we know that $M+1 = \prod_{i=1}^k p_i^{a_i}$. Without a loss of generality assume that $a_1 \not= 0$. Then $p_1$ divide $M+1$ and since $p_1$ divides $M$ it must be the case that $p_1$ divides $1$ as well which presents a contradiction.
\end{proof}

\textbf{Fact:} Let $p_1, p_2, p_3, \dotsc$ be a list of primes in order. By the uniqueness of prime factorization, there is an injective correspondence between vectors $(a_1, a_2, \dotsc) \in (\Z_{\geq0})^\infty$ with finitely many nonzero entries and $\N$. The correspondence is $n = \prod p_i^{a_i}$ with a lot of $a_i$ being zero.

\textbf{NOTE: THE BELOW IS BY CONTRADCTION and (*) ONLY HOLDS FOR $k=\infty$.}
If we assume that the list of primes is finite, then we would have an injective correspondence between $(a_1, \dotsc, a_k) \in (\Z_{\geq 0})^k$ and $\N$. Therefore
\[ \prod_{i=1}^k\qty(\sum_{j=0}^\infty \frac{1}{p_i}) = \sum_{n=1}^\infty \frac{1}{n}. \tag{*} \]
Then by uniqueness of prime factorization for each $n \in \N$ we know that $1/n$ appears exactly once when you expand this product. This is Euler's product for the $\zeta$ function?

\begin{proof}[Euler's Proof]
Assume by way of contradiction that there are finitely many primes. Then 
\[ \sum_{n=1}^\infty \frac{1}{n} = \prod_{i=1}^k\qty(1 + \frac{1}{p_1} + \cdots) = \prod_{i=1}^k \qty(\frac{1}{1-1/p_i}) < \infty. \]
Yet we know that $\sum 1/n$ diverges, which presents a contradiction.
\end{proof}

\begin{lemma}
For any $n \in \Z$ there exists a unique $a,b \in \Z$ such that $a$ is square free (meaning that no square number divides it) and $n = ab^2$.
\end{lemma}

\begin{proof}[Erdős' Proof]
Assume by way of contradiction that there are finitely many primes. Then any square-free number $n = \prod p_i^{a_i}$ where $a_i \in \{0,1\}$. Thus there are only $2^k$ square-free numbers. Now let's look at all numbers at most $N$ for some $N$. By the above lemma, they can be specified by $(a,b)$ where $a$ is square-free and $b^2$ is square. There are $2^k$ square-free numbers and at most $\sqrt{N}$ square numbers, so $N \leq 2^k\sqrt{N}$ for all $N$, so $2^k \geq \sqrt{N}$ for all $N$, which is very very false if $N > 2^{2k}$.
\end{proof}

\subsection{Congruence Equations \& Modular Arithmetic}

\begin{definition}[Congruence]
We say that $a \equiv b \pmod{m}$ if and only if $m$ divides $b-a$. Alternatively, we say that $a \equiv b \pmod{m}$ if and only if there exists some $k$ such that $a = b + mk$.
\end{definition}

\begin{theorem}[Some quick remarks] \hfill
\begin{enumerate}
\item Congruency is an equivalence relation on the integers (transitive, symmetric, and reflexive);
\item For some fixed $m$, we define the congruence class $\bar{a}$ to be the set $\bar{a} = \{n \in \Z \mid n \equiv a \pmod{m}\}$;
\item Arithmetic on these congruence classes holds; If $a \equiv b \pmod{m}$ and $c \equiv d \pmod{m}$ then $a+c \equiv b + d \pmod{m}$ and $ac \equiv bd \pmod{m}$. Thus $\bar{a} + \bar{b} = \overline{a + b}$ and $\bar{a}\bar{b} = \overline{ab}$. This forms the commutative ring $\Z/m\Z$.
\end{enumerate}
\end{theorem}

\begin{problem}
Let $a,b,m$ be fixed.
When is the congruence $ax \equiv b \pmod{m}$ solvable?
\end{problem}

\begin{enumerate}[label=\textbf{Obs. \arabic*.}]
\item If $(a,m)=1$ then we can use Bezout's theorem. This tells us that tehre exist some $X,Y$ such that $1 = aX + mY$. Then we multiply through by $b$ to get that $b = a(Xb) + m(Yb)$. Then $aXb \equiv b \pmod{m}$.
\end{enumerate}

\begin{lemma}
The congruence $ax \equiv b \pmod{m}$ has solutions if and only if the \textsc{gcd} of $a$ and $m$ divides $b$.
\end{lemma}

\begin{proof}
Let $d$ be the \textsc{gcd} of $a$ and $m$. By Bezout, there exists some $X_0, Y_0 \in \Z$ such that $d = aX_0 + bY_0$. Since $d$ divides $b$ there exists some $k$ such that $b = dk$. Then $b = aX_0k + mY_0k$ so $b \cong aX \pmod{m}$ for $X = X_0k$. In the other direction, just write it out. If there is a solution then $b \cong aX \pmod{m}$ so $b = aX + mY$. Since the \textsc{gcd} divides the right hand side it must divide the left as well, so $d$ divides $b$.
\end{proof}

\begin{problem}
Can there be lots of different solutions? What do solutions look like?
\end{problem}
% !TEX root = ../notes.tex

\section{Monday, January 28}

\begin{multicols}{2}
\begin{enumerate}
\item Degrees of Reality;
\item Formal Reality \& Objective Reality;
\item Mister Ed;
\item Two causal principles \& and the PSR;
\item Whodunit;
\item D.'s finest hour;
\item Not a deceiver;
\item Cartesian Circle;
\item Intellectual Problem of Evil;
\item Belief \& the Will.
\end{enumerate}
\end{multicols}

\subsection{Degrees of Reality}

Descartes' proof of God's existence hinges on different degrees of reality, which in some sense is a measure of independence; in Descartes' view, God (an infinite substance) ought to be independent on anything else while everything else ought to be dependent on God. Finite and/or extended substances (table, mind, Mr.\ Ed, etc.) all depend on God, and in turn the shape of a table depends on the table itself.

\begin{definition}[Formal Reality]
The reality something has by virtue of its existence. Usually a measure of how independent a thing is. Often also derives from the complexity of the thing; a machine would have more formal reality than a rock.
\end{definition}

\begin{definition}[Objective Reality]
The objective reality of an idea is equal to the formal reality of the object of the idea has, or would have if it existed. Only ideas can have objective reality.
\end{definition}

\begin{example}
Consider the idea of Mr.\ Ed. This has relatively low formal reality since it's just a though in a mind, but has relatively high objective reality since Mr.\ Ed himself has a relatively high formal reality. The idea of god has infinite objective reality.
\end{example}

\subsection{Two Causal Principles}

\begin{proposition}
The formal reality of a cause is greater than or equal to the formal reality of the effect.
\end{proposition}

\begin{proposition}
The formal reality of the cause is greater than or equal to the objective reality of the effect.
\end{proposition}

From this, God asks ``What causes my idea of God?''
\section{September 12, 2018}

\epigraph{``Oh, I erased my smiley face. How sad.'' (she did not sound sad)}{Miki}

Today we'll officially state something we covered last time.

\begin{theorem}[Caley's Theorem]
Every finite group $G$ is isomorphic to a subgroup of $S_n$ for some $n$.
\end{theorem}
\begin{proof}
Let $n = \abs{G}$.
\end{proof}

\subsection{Kernels}

Let's discuss formally the idea of a kernel of a homomorphism and a kernel of a group action.

\begin{definition}[Kernel]
Let $\phi : G \to H$ be a homomorphism. Then the kernel of $\phi$, written $\ker \phi$, is the set of all elements in $G$ which are mapped to the identity in $H$; i.e., $\ker \phi = \{g \mid \phi(g) = 1_h \}$.
\end{definition}

\begin{definition}
Suppose $G$ acts on $A$ by $\pi$. Then the kernel of the action is the set of all elements of $g$ which act trivially on $A$; i.e., $\ker \pi = \{g \mid ga = a \text{ for all $a \in A$}\}$.
\end{definition}

\begin{example}
Consider the action $\phi : \mathrm{GL}_2(\R) \to (\R^\times, \times) : A \mapsto \det A$. Then the kernel of $\phi$ are all matricies with determinant $1$, called $\mathrm{SL}_2(\R)$.
\end{example}

\begin{definition}[Stabalizer]
Let $\pi : G \times A \to A$ be a group action, and fix $a \in A$. The \emph{stabalizer} is $G_a = \{g \in G \mid ga = a\}$. By this definition, the kernel is contained within any stabalizer, and in fact is equal to the intersection of all stabalizers.
\end{definition}

\begin{example}
Let $G = \mathrm{GL}_2(\R)$ and let $A = \R^2$ defined with the usual action (vector-matrix multiplication). What is the kernel of this action? Then let $c = (0,1)^\top \in \R^2$. What is the stabalizer of $c$?
\end{example}

\begin{corollary}
The kernel of an action is a subgroup of $G$, and $G_a$ is a subgroup of $G$ for any fixed $a \in A$.
\end{corollary}

\begin{definition}[Orbit]
Fix $a \in A$. The orbit of $a$ is the image of $a$ under the group action; i.e., $O_a = \{ga \mid g \in G\}$. Intuitively, it's everywhere $a$ can go under a specific group action. Notice that the orbits partition $A$, and so are equivalence classes in $A$.
\end{definition}

\begin{example}
Let $G = \mathrm{GL}_2(\R)$ and let $A = \R^2$ defined with the usual action (vector-matrix multiplication). What is the orbit of $(1,0)^\top$?
\end{example}

\begin{definition}[Faithful]
An action is faithful if the kernel is the identity. This means that the base element of the action must be the identity. This tells us that $G$ is injective into $S_A$.
\end{definition}

\begin{example}
Consider $D_8$ acting on a square (technically the set $A = \{1,2,3,4\}$). The orbit $O_1$ is all possible vertices, since you can rotate any vertex to any position. The stabalizer is $\{1, s\}$.
\end{example}

\begin{lemma}
As it turns out, for a fixed $a \in A$, we see that $\abs{O_a}\abs{G_a} = \abs{G}$. We'll prove this later. (Orbit-Stabalizer Theorem I think?)
\end{lemma}

\begin{definition}[Conjugation]
Consider the action $\pi : G \times G \to G : (g,a) \mapsto gag^{-1}$. This action is known as \emph{conjugation}.
\end{definition}

\begin{definition}[Centralizer]
Let $S \subset G$. The \emph{centralizer} of $S$ in $G$, written $C_G(S) = \{g \in G \mid gsg^{-1} = s \text{ for all $s \in S$}\}$. This is the set of things that fix $S$ in $G$ pointwise under conjucation. By definition, this is the set of elements in $G$ which commute with all elements in $S$. In the case that $S = \{s\}$ we see that $C_G(S) = G_S$. 
\end{definition}

\begin{definition}[Normalizer]
Let $S \subset G$. The \emph{normalizer} of $S$ in $G$ is $N_G(S) = \{g \in G \mid gSg^{-1} = S\}$. Essentially, this is just a centralizer on a set, except that it may permute the elements of $S$. Then $C_G(S) \subset N_G(S)$.
\end{definition}

\begin{example}
Suppose that $G$ is abelian. For any $S \subset G$, we see that $C_G(S) = N_G(S) = G$.
\end{example}

\begin{example}
Let $G = S_3$, and let $S = G$. What is the normalizer of $S$? (It's the whole thing since $G$ is closed under its operation.) What is the centralizer of $S$? (It's the identity.)
\end{example}
\section{September 14, 2018}

\epigraph{``Why do we get struck by lightning when we reach a contradiciton? I don't know, it's usually a good thing.'' \lightning}{Miki}

\begin{definition}[Center]
The center of a group $G$ is $Z(G) = \{g \in G \mid gs = sg \text{ for all $s \in G$}\}$; i.e., $Z(G) = C_G(G)$, so it's the centralizer of the whole group.
\end{definition}

Why do we care so much about conjugation? We give all these special names to the sets of conjugation, like the Normalizer, Stabalizer, and Centralizer. We also know that conjugation preserves the order of an element, so $\abs{a} = \abs{gag^{-1}}$.

\begin{problem}
What is the center of $D_8$? We know the identity must be in the center. What about $r^2$? We know it commutes with $s$, and $sr^2 = r^{-2}s = r^2s$, so it commutes with $s$ as well; Since $r$ and $s$ generate the group, we know that it be in the center as well. So $Z(D_8) = \{1, r^2\}$.
\end{problem}

\subsection{Cyclic Groups}

\begin{proposition}
Let $G$ be a group, and let $x \in G$. For $m,n \in \Z$, if $x^n = x^m = 1$ then $x^d = 1$ where $d = \gcd(m,n)$.
\end{proposition}
\begin{proof}
Use the Euclidean Algorithm. We know there are integers $a,b \in \Z$ where $d = am + bn$, so $x^d = x^{am+bn} = (x^a)^m (x^b)^n = 1^a1^b = 1$.
\end{proof}
\begin{corollary}
If $x^m = 1$ then $\abs{x}$ divides $m$ if $m$ is finite.
\end{corollary}
\begin{proof}
If $m = 0$, we are done since everything divies zero. Assume that $1 \leq m < \infty$. Let $n = \abs{x} \leq m < \infty$ be finite. Let $d = \gcd(m,n)$, so $x^d = 1$. We know that $d$ divides $n$, and since $n$ is the smallest power of $x$ to be the identity, we know that $d = n$. \emph{A priori,} we know that $d$ divides $m$ so $d$ must divide $m$ as well.
\end{proof}
\begin{proposition}
Let $x \in G$, and let $a \in \Z \setminus \{0\}$.
\begin{enumerate}
\item If $\abs{x} = \infty$, then $\abs{x^a} = \infty$;
\item If $\abs{x} = n < \infty$, then $\abs{x^a} = n / \gcd(a,n)$.
\end{enumerate}
\end{proposition}
\begin{proof}
The proof of (1) is ommitted, and left as an exercise to the student. For (2), let's focus on the special case that $a$ divides $n$. If $x^n = 1$ then $(x^a)^{n/a} = x^n = 1$. Then $\abs{x^a}$ is at most $n/a$. Suppose by way of contraction that the order $d$ is strictly less than $n / a$. Then $x^{ad} = 1 \implies 1 \leq ad < n$, but $\abs{x} = n$. This is a contradiction, so the order of $x^a$ must be exactly $n/a$. In the case that $a$ does not divide $n$, play around with this to get the more general conclusion (the logic is the same).
\end{proof}

\begin{definition}[Cyclic Group]
A group $G$ is cyclic if there exists an $x \in G$ such that $G = \langle x \rangle$. As a note, it's not always easy to tell since there could be other presentaitons of a group which are not single elements. Always remember that presentations are not unique.
\end{definition}

\begin{problem}
Let $G = \langle a,b \mid a^2 = b^3 = 1, ab = ba \rangle$. Show that $G$ is cyclic.
\end{problem}

\begin{corollary}
All cyclic groups must be abelian, since any $g \in G$ is generated by some $x^a$, and $x$ always commutes with itself.
\end{corollary}

\begin{example}[Infinite Cyclic Groups]
Throughout, let $G = \langle x \rangle$, and assume that $\abs{x} = \infty$.

\begin{proposition}
The order of $G$ is $\infty$. Then $x^m \not= x^n$ for all distinct $m,n \in \Z$.
\end{proposition}

\begin{proof}
Let $m < n$. Suppose by way of contradiction that $x^m = x^n$. Then $x^{n-m} = 1$, which cannot happen since $n-m > 0$ and $\abs{x} = \infty$. Then $\abs{G} = \infty$.
\end{proof}

\begin{proposition}
Such $G$ must be isomorphic to $(\Z, +)$.
\end{proposition}
\begin{proof}
Define a map $\phi : \Z \to G : n \mapsto x^n$. This map is well defined. It also respects multiplication since $m+n \mapsto x^mx^n$. It is injective by Proposition~1, and it is surjective by Proposition~1 since $G$ is generated completely by $x$. Then $\phi$ is an isomorphism.
\end{proof}

\begin{proposition}
Such a group $G$ is generated by $x^n$ if and only if $n = \pm 1$.
\end{proposition}
\begin{proof}
Left as an exercise to the student.
\end{proof}

\begin{proposition}
Every subgroup of $G$ is cyclic of the form $H = \langle x^n \rangle$ for some $n \in \Z$.
\end{proposition}
\begin{proof}
Suppose that $x^n = 1$. Then $H$ is obviously cyclic. On the other hand, if $H \not= \langle 1 \rangle$. Let $n = \min\{k > 0 \mid x^k \in H\}$. This can't be empty, so there is an $n$. Then $\langle x^n \rangle \subset H$. Take some other element $x^m \in H$, and let $d = \gcd(m,n) = am + bn$. Then $x^d = (x^m)^a(x^n)^b \in H$ but $1 \leq d \leq n$, so $d = n$. Then $n$ divides $m$, and so $x^m \in \langle x^n \rangle$. The $\langle x^n \rangle = H$, and so $H$ is cyclic.
\end{proof}
\end{example}
\begin{corollary}
Every non-trivial subgroup of $\Z$ is isomorphic to $\Z$.
\end{corollary}
\begin{corollary}
For some cyclic $G$, we know that $\langle x^n \rangle = \langle x^{-n} \rangle \subset G$. Then all non-trivial subgroups correspond to $\Z_{>0}$.
\end{corollary}

\section{September 17, 2018}

\epigraph{``The only thing I learned for years was how to count hedgehogs in a field.'' (In the midst of a wonderful and inspirational talk about being a mathematician.)}{Miki}

\subsection{Finite Cyclic Groups}

Today, we'll cover finite cyclic groups. This will be very similar to the previous lecture on infinite cyclic groups. As a reminder, here are the propositions for infinite cyclic groups:

\begin{proposition}[Infinite Cyclic Groups]
Let $G$ be an infinite cyclic group.
\begin{enumerate}
\item The order of $G$ is infinite, with $G = \{\dotsc,x^{-1},1,x,x^2,\dotsc\}$ all distinct.
\item The group $G$ is isomorphic to $\Z$.
\item The group $G$ is generated by $x^n$ if and only if $n = \pm 1$.
\item Every subgroup $G$ is cyclic.
\end{enumerate}
\end{proposition}

Now for the finite case. 

\begin{proposition}[Finite Cyclic Groups]
 Let $G = \langle x \rangle$ with $\abs{G} = n < \infty$.
 \begin{enumerate}[label=P\arabic*.]
 \item The group $G$ is exactly $\{1,x,\dotsc,x^{n-1}\}$.
 \item The group $G$ is isomorphic to $\Z/n\Z$.
 \item The group $G$ is generated by $x^k$ if and only if $\gcd(k,n) = 1$.
 \item Every subgroup of $G$ is also cyclic. That is, for all $k > 0$ where $k$ divides $n$ we get a subgroup $H$ of order $k$ generated by $x^{n/k}$
 \end{enumerate}
\end{proposition}
\begin{proof}[Proof of~P1]
We know that $1,\dotsc,x^{n-1}$ are all in $G$. Suppose that $x^a = x^b$ for some distinct $a,b$. Then $x^{b-a} = 1$ for $0 < b-a < n$, which is a contradiciton since $\abs{x} = n$. In fact, this set enumerates $G$. Suppose that $x^k \in G$ for some $k \in \Z$. We use the divison algorithm to write that $k = an + r$ for some $a,r \in \Z$. Then $x^k = x^{an+r} = (x^n)^ax^r = x^r$, so $x^k$ is in $G$.
\end{proof}
\begin{proof}[Proof of~P2]
Let $\phi : \Z/n\Z \to G : \bar{k} \mapsto x^k$ where $k$ is any representative of $\bar{k} \in \Z$. To show that $\phi$ is well-defined, consider another representative $\ell$ of $\bar{k} \in \Z$. Then $\ell = k + an$, so $x^\ell = x^{k+an} = x^k(x^n)^a = x^k$. To show that $\phi$ is a homomorphism, consider that $\phi(\bar{m}+\bar{n}) = x^{m+n} = x^{m}x^n$, so $\phi$ is multiplicative. Finally, we know that $\phi$ is surjective and injective by P1. This tells us that, up to an isomorphism, there are only really two cyclic groups; $\Z$ if the group is of infinite order, or $\Z/n\Z$ if it is finite.
\end{proof}
\begin{proof}[Proof of~P3]
This is more of a sketch. Recall that $\langle \abs{x^k}\rangle = \abs{x^k}$, and this is $n$ if and only if $\gcd(k,n) = 1$. In general, $\abs{x^k} = n / \gcd(k,n)$.
\end{proof}
\begin{proof}[Proof of~P4]
Exactly the same as the infinite case.
\end{proof}

Now that we've covered cyclic groups, it's helpful to introduce some notation to represent them.
\begin{notation}[$\Z_n, C_n$]
We write the multiplicative cyclic group of order $n$ as $\Z_n$. The additive cyclic group of order $n$, which we've been writing as $\Z/n\Z$, is commonly written as $C_n$.
\end{notation}

\subsection{Subgroups}

\cite{DF} uses the notation $S \subset G$ to mean that $S$ is a subset of $G$, and $H \leq G$ to mean that $H$ is a subgroup of $G$.

\begin{definition}[Subgroup]
Let $S \subset G$ be nonempty. Let $H = \{a_1^{\varepsilon_1}\cdots a_k^{\varepsilon_k}\}$ where $a_i \in S$ and $\varepsilon_i = \pm 1$ for $k \in \Z_{\geq 0}$. This sequence of $a_i^{\varepsilon_i}$ is called a \emph{word}. Note that $a_i$ need not be distinct. Then $H$ is a subgroup of $G$.
\end{definition}

\begin{proof}
First, we know that $H \subset G$. We know that $1 \in H$. Since any concatenation of words is also a word, we know that $H$ is closed under multiplication. Finally, since $((ab)^n)^{-1} = b^{-n}a^{-n}$ we know that the inverses of a word are words themselves, and so $H$ is closed under inversion.
\end{proof}
\section{September 19, 2018}

\epigraph{``Funny things happen with groups, which is why they're fun!''}{Miki}

Recall from last time how we defined a subgroup $H$ of $G$ in terms of words where the powers of each element was $\pm 1$. If $G$ is abelian we can combine elements of like bases to get powers which can be any integral value. If we assume that $\abs{a_i} = d_i$ is finite for all $a_i \in H$, then we know that $\abs{H} \leq d_1 \cdots d_k$. This gives us a limit on the order of a subgroup; if $G$ is abelian then the order of a subgroup is bounded above by the product of the orders of the generating elements. On the other hand, if $G$ is not abelian then this does not always hold. Consider $G = \langle a,b \mid a^2 = b^2 = 1 \rangle$. If $G$ isn't commutative, then $(ab)^n \not= a^nb^n$ for all $n$ and so we can just create infinitely many words by appending $ab$ to one another and so the order is infinite.

\begin{lemma}
Let $G = \{a_1^{n_1} \cdots a_k^{n_k}\}$ be abelian, and let each $a_i$ have finite order $d_i$. Then $\abs{G} \leq d_1 \cdots d_i$.
\end{lemma}

\begin{proposition}
Let $G$ be a group and let $\mathcal{L}$ be a collection of subgroups of $G$. Then \[ K = \bigcap_{L \in \mathcal{L}} L \]
is a subgroup of $G$.
\end{proposition}

\begin{definition}[Subgroup]
Let $S \subset G$ and let $\mathcal{L} = \{L \leq G \mid s \subset L\}$. Then the subgroup generated by $S$ is \[ K = \bigcap_{L \in \mathcal{L}} L. \]
\end{definition}

What do we know from this definition? Well, $S \subset K$ and $K \leq G$. We want to say that $K \in \mathcal{L}$ is the minimal element, so $K = L_i$ for some $i$.

\begin{definition}[Minimal Element]
Let $\mathcal{M}$ be a collection of subsets of $G$. A minimal element is an element $M$ of $\mathcal{M}$ such that if $M' \in \mathcal{M}$ and $M' \subset M$ then $M = M'$. It's like ``the smallest element'' except there could be multiple minimal elements.
\end{definition}

We want to show that $K$ is \emph{the} minimal element of $\mathcal{L}$.

\begin{proof}[Proof $K$ is minimal]
Let $L \in \mathcal{L}$. Then $K \subset L$. Then either $K = L$ or $L$ is not minimal.
\end{proof}

\begin{proof}[Proof $K$ is \emph{the} minimal element]
Suppose there is another minimal $M$ in $\mathcal{L}$. By definition $K \subset M$ so by minimality $M = K$.
\end{proof}

\begin{proposition}
Our two definitions for subgroup (generated by words $H$ and minimal element of collection $K$) containing elements of $S \subset G$ are equivalent.
\end{proposition}

\begin{proof}
$H \leq G$ and $S \subset H$ by the construction of $1$-letter words. Then $H \in \mathcal{L}$ so $K \subset H$. On the other hand, $S \subset K$ and $K$ is a group. Then $K$ contains all inverses and products of elements in $S$, so it contains all words and therefore contains $H$. Then $H \subset K$. Putting this together, we have that $K = H$.
\end{proof}

\begin{definition}[Lattice]
Given a group $G$, a lattice is a diagram showing all subgroups of $G$ which shows containment between the subgroups.
\end{definition}

\begin{figure}[h]
	\caption{Lattice Diagram of $C_2$}
\end{figure}

Recall from last time that for $C_n$ the subgroups are paired with the divisors $k$ of $n$; then $\langle k \rangle$ generates subgroup of order $n / k$.

\begin{figure}[ht]
	\caption{Lattice Diagram of $C_4$}
\end{figure}

\begin{figure}[ht]
	\caption{Lattice Diagram of $C_8$}
\end{figure}

\begin{figure}[ht]
	\caption{Lattice Diagram of $C_6$}
\end{figure}

\begin{figure}[ht]
	\caption{Lattice Diagram of $S_3$}
\end{figure}
\section{September 21, 2018}

\epigraph{``Oh I erased my smiley face again. How sad.'' (She did not sound sad.)}{Miki}

\begin{problem}[Warm up]
Draw the lattice diagram for $C_{12}$.
\end{problem}

\begin{figure}[h]
	\begin{center}
	\end{center}
	\caption{Lattice for $C_{12}$}\note{Finish this}
\end{figure}
\section{September 24, 2018}

\epigraph{``This is where the fun begins.'' (slightly paraphrased)}{Miki}

\subsection{Quotient Groups}

For the rest of this section, keep in mind the example of $\Z/n\Z$. This is kind of like the prototypical example for quotient groups.

\begin{definition}[Coset]
Let $H \leq G$. The left coset of $H$ in $G$ is a set of the form $aH = \{ah \mid h \in H\} \subset G$ for a fixed $a \in G$. The right coset of $H$ in $G$ is a set of the form $Hb = \{hb \mid h \in H\} \subset G$ for a fixed $b \in G$.
\end{definition}

We said previously that left multiplication permutes the elements of $H$ (this was called the left regular action), and in particular we know that $\abs{aH} = \abs{H}$. We can see this trivially by simply multiplying each $ah$ by $a^{-1}$. Note that this coset is usually \emph{not} a subgroup; if $a^{-1} \notin H$ then $e \notin aH$.

\begin{example}
Let $G = \Z$, and let $H = 2\Z$. Consider the cosets $0 + H$ and $1 + H$ (these are just the even integers and the odd integers). In particular, $0 \notin 1 + H$ and so $1 + H \not\leq G$.\note{The Right and Left cosets here are equal, which is always true of $G$ is abelian.}
\end{example}

Notice that in the above example, the cosets are disjoint and partition the group into equivalence classes. In general this is a true statement.

\begin{lemma}
The costs of $H$ partition $G$ into equivalence classes, with the relation $a \sim b$ if and only if $a = bh$ for some $h \in H$. In particular, $a \sim b$ if and only if $aH = bH$, and so the cosets defined by those elements are identical.
\end{lemma}

\begin{corollary}
The order of the cosets divides the order of $G$. In particular, $\abs{G} = \abs{H} \cdot [G : H]$ where $[G : H]$ is the \emph{index} of $H$ in $G$ and is the number of (left OR right) cosets of $H$ in $G$.
\end{corollary}

In the example with $\Z$ and $2\Z$, lets try to make these cosets behave like groups. Consider that $(0 + H) + (1 + H) = (1 + H)$ (which just says that and even plus an odd equals an odd). We also have a homomorphism $\pi : \Z \to 2\Z : n \mapsto \bar{n} = n+H$. This maps integers to cosets. Note that $\pi$ respects the operations in each group! This is kind of what defines ``adding even and odd integers'' in the languages of sets.

\begin{notation}
Let $0+H = \pi^{-1}(\bar{0}) = \{n \in \Z \mid \pi(n) = \bar{0}\}$; this is the preimage of $\pi$ or the fiber of $\pi$ above $0$. Yes this is overloaded notaiton, and no $\pi$ does not have an inverse (it's pretty clearly \emph{not} injective.)
\end{notation}

Note that it doesn't really matter which elements we send into $\pi$ as long as they are both of the same coset, so $\pi(a) = \pi(b)$ if and only if $a \sim b$. Additionally, note that since $\pi$ is a homomorphism we can say that $\pi(\bar{1} + \bar{20}) = \pi(\bar{1}) + \pi(\bar{20})$.

Now, in making these cosets into groups we want them to inherit their operation from the parent group (so we can't just make up multiplications to suit our needs).

\begin{definition}
Let $A,B \subset G$. Then $AB = \{ab \mid a \in A, b \in B\} \subset G$. In particular, note that $HH = H$ and $(1H \cdot 1H = 1H)$.
\end{definition}

\begin{example}[Things Go Wrong]
Let $G = S_3$ and let $H = \langle (23) \rangle = \{1, (23)\}$. The cosets of $H$ are $1H = (23)H$, $(12)H = (123)H = \{(12), (123)\}$, and $(13)H = (132)H = \{(13), (132)\}$. Now consider $1H \cdot (12)H = \{(12), (123), (132), (13)\}$. In particular, note that this isn't a coset (it has too many elements!). We would have wanted that $1H \cdot (12)H = (12)H$ but this doesn't happen. Then there is not quotient group $G / H$.
\end{example}

What just happened? Why can't we create a group out of the cosets of $S_3$? We wanted that $aH \cdot bH = abH$ but this didn't happen; essentially, we want $b$ and $H$ to commute, so we want that the left and right cosets to be equal to one another.

\begin{example}
Let $G = S_3$ and let $H = \langle (123) \rangle = A_3$. This is the alternating group on three letters. As always, $1H = H1 = H$. Note that $(12)H$ contains \emph{the only other elements of $G$} which aren't in $1H$, and so $(12)H = H(12) = G \setminus 1H = G \setminus H1$. This happens when $[G : H] = 2$ even though $G$ is not abelian. Then $S_3 \setminus A_3 = G \setminus H$ and $\bar{a}\bar{b} = \bar{ab}$ so multiplication is well defined.
\end{example}


\section{September 26, 2018}

\epigraph{Let $N$ be a group...I'll call it $N$ suggestively}{Miki}

Recall from last class that we found and example of a non-abelian group and a subgroup for which the left and right cosets of the group were the same; in this case, it was $G = S_3$ and $H = A_3$.

\begin{definition}[Quotient Group]
Let $H \leq G$. The \emph{quotient group} $G/H$ is a group whose elements are the left cosets of $H$. The set for this group is known as the quotient set, and the operation for the group is inherited from $G$ such that $gH \star kH = gk H$. Note that $(gH, \star)$ does not always form a group, so it isn't guaranteed that $G/H$ exists for any $G,H$. 
\end{definition}

\subsection{Mapping from \texorpdfstring{$G$}{G} to \texorpdfstring{$G/H$}{G/H}}

Given a group $G$ and a quotient group $G/H$ we can find a very natural mapping $\pi : G \to G/H$ where $g \mapsto gH$. This map sends elements to their coset, and $\pi(a) = \pi(b)$ if and only if $aH = bH$; thenthe fibers of $\pi$ are the left cosets of $H$, and $\ker \pi = H$. This is why we call it the quotient group --- it's like we're dividing out by $H$. Note that this homomorphism is always going to be surjective since there's no member of $G$ which isn't in some coset of $H$ as they partition $G$.

\begin{definition}[Normal Subgroup]
Let $N \leq G$. Then $N$ is normal if and only if the left and right cosets are the same, so $gN = Ng$. If $N$ is normal then $G/N$ forms a quotient group. Note that this does not mean that $gn = ng$ so $g$ and $n$ do not commute necessarily, but the cosets are preserved. This is equivalent to saying that $\mathrm{N}_N(G) = G$ but $\mathrm{C}_N(G)$ is not necessarily $G$.
\end{definition}

\begin{notation}[$\normal$]
We write $N \normal G$ to mean that $N$ is a normal subgroup of $G$.
\end{notation}

\begin{theorem}
The quotient group $G / N$ exists if and only if $N \normal G$.
\end{theorem}

\begin{proof}[Proof that $N \normal G$ is sufficient]
Observe that $(aN)(bN) = abN$ if $N$ is normal. Then group multiplication is well defined. Observe also that $(aN)^{-1} = a^{-1}N$, so the group is closed under inversion, and by definition our multiplication is associative. Then $G/N$ forms a group if $N$ is normal in $G$.
\end{proof}

\begin{proof}[Proof that $N \normal G$ is necessary]
Suppose $H \leq G$ is not normal. Then there is some $g \in G$ for which $gH \not= Hg$. Then we know that $1HgH \not= gH$, and our group operation $\star$ cannot hold.
\end{proof}

Not that $\abs{G/N} = \abs{G} / \abs{N} = [G : N]$ if $G$ is finite, which we already knew but it's worth remembering it.

\subsection{Testing Normality}

\begin{proposition}
The following are equivalent:
\begin{itemize}
\item $N \normal G$;
\item $gNg^{-1} \subset N$ for all $g \in G$ (note this implies they are equal since conjugation is injective);
\item $N$ is the kernel of some homomorphism $\pi : G \to H$ for some $H \leq G$.
\end{itemize}
\end{proposition}

\begin{proof}[Proof that 1 $\implies$ 2]
Let $g \in G$ and $n \in N$. We know that $gN = Ng$, so there exists $n' \in N$ such that $ng = n'g$. Multiply on the right by $g^{-1}$ and we see that $gng^{-1} = n'$, and so $gng^{-1} \in N$ for all $g,n$.
\end{proof}

\begin{proof}[Proof that 2 $\implies$ 1]
Literally just reverse the above procedure.
\end{proof}

\begin{proof}[Proof that 1 $\implies$ 3]
Let $H = G / N$. Then we know that $\ker \pi = N$ where $\pi : G \to G/N : g \mapsto gN$. Then, rather trivially, we know $N$ is the kernel for some homomorhpism if $N \normal G$.
\end{proof}

\begin{proof}[Proof that 3 $\implies$ 2]
We know that $N = \ker \pi$ for some $\pi : G \to H$. Then take any $g \in G$ and $n \in N$, and consider that $\pi(gng^{-1}) = \pi(g)\pi(n)\pi(g^{-1}) = \pi(g)\pi(g^{-1})$ since $n \in \ker \pi$, and then we conclude that $\pi(g)\pi(g^{-1}) = 1$ and so we know that $gng^{-1} \in \ker\pi$ so $gng^{-1} \in N$ for all $n \in N$ and for all $g \in G$.
\end{proof}
\section{Friday, 28 September 2018}

\epigraph{``I'll leave the cosets for later, where later means 15 seconds from now.''}{Miki}
\epigraph{``Continuous math is not allowed...don't tell anyone I said that.''}{Miki}

Recall Lagrange's Theorem, where if $G$ is a finite group and $H \leq G$ then $\abs{H}$ divides $\abs{G}$; in fact, $\abs{G} / \abs{H} = [G : H]$.

\begin{corollary}
Let $G$ be a finite group and let $x \in G$. Then $\abs{x}$ divides $\abs{G}$ since $x$ generates a cyclic subgroup of order $\abs{x}$, so $\abs{x} = \abs{\langle x \rangle}$ which must divide $\abs{G}$ by Lagrange.
\end{corollary}

\begin{corollary}
If $\abs{G} = p$ is prime, then $\abs{G} \cong Z_p$.
\end{corollary}

\begin{proof}
Since $\abs{G} \not= 1$ there exists $x \in G$ which is not the identity. Then consider $\langle x \rangle$. The order of this cyclic group must divide $p$, and since $p$ is prime it must equal $p$, and so $G = \langle x \rangle$ which means it is isomorphic to $Z_p$.
\end{proof}

If we have some $n \in \Z_{>0}$ where $n$ divides $\abs{G}$ for some $G$, it isn't guaranteed that there exists some $H \leq G$ where $\abs{H} = n$, and/or there isn't always an $x \in G$ where $\abs{x} = n$. For example, consider $G = S_3$ and $n = 6$. However, if prime $p$ divides $\abs{G}$ then there exists an $x \in G$ where $\abs{x} = p$ --- Miki says she will prove this later.

\subsection{Product Subgroups}

Let $G$ be a group and let $H,K \leq K$. Let's consider the product of $HK$, which we recall is defined as \[ HK = \{ hk \mid h \in H, k \in K \}. \]
This may or may not be a subgroup. In general it is not.
\begin{example}
Let $G = S_3$, and let $H = \langle (12) \rangle$ and let $K = \langle (13) \rangle$. Then $HK = \{1, (12), (13), (132)\}$ which is not a subgroup of $S_3$ since $4$ does not divide $6$.
\end{example}
What can we say about $HK$ anyways?
\begin{proposition}
The order of $HK$ is at most $\abs{H}\abs{K}$. In fact, 
\[ \abs{HK} = \frac{\abs{K}\abs{K}}{\abs{H \cap K}}. \]
\end{proposition}

\begin{proof}
We know that $HK$ is the union of left cosets of $K$ where 
\[ HK = \bigcup_{h \in H} hK. \]
Consider $a,b \in H$. We know that $aK = bK$ if and only if $a^{-1}b \in K$ which is true if and only if $a^{-1}b \in K \cap H$. This means that $a K \cap H = b K \cap H$. Then we've reduce the problem to counting the number of distinct cosets $hK$ which is just the index, so it is $\abs{H} / \abs{K \cap H}$. Multiplying through by the size of $K$, we find that 
\[ \abs{HK} = \frac{\abs{H}\abs{K}}{\abs{K \cap H}}.\qedhere \]
\end{proof}

Now we can answer when $HK$ is a subgroup; it happens if and only if $HK = KH$. Intuitively, this happens only when $hkh'k' \in HK$ which can happen if and only if we can commute the $h$ and $k$ elements. It is sufficient to say that $H$ is in the normalizer of $K$ or vice-versa. Another sufficient condition is to say that $K \normal G$, or the other way around. Note that neither of these conditions is necessary. 
\[ H \leq N_G H \implies hK = Kh \implies hk = k'h, \]
but we only need that $hk = k'h'$. That is, we only need that $hK = Kh'$ which is a weaker condition than being in the normalizer.

\subsection{Isomorphism Theorems}

\begin{theorem}[First Isomorphism Theorem]
Given a surjective homomorphism $\phi : G \to H$, we know that $H \cong G/\ker\phi$.
\end{theorem}
\begin{proof}
This was the definition of $G / \ker\phi$, since $\ker\phi \normal G$. See the previous lecture notes for a more in-depth explanation.
\end{proof}

\begin{example}
Consider $\mathrm{GL}_2(\F_3)$ and let $\phi = \det : G \to \F_3^\times$. Then $\ker \phi = \mathrm{SL}_2(\F_3)$, and $\mathrm{GL}_2(\F_3) / \mathrm{SL}_2(\F_3) \cong \F_3^\times$. Since $\mathrm{GL}_2(\F_3)$ has $48$ and $\F_3^\times$ has $2$ elements then we know that $\mathrm{SL}_2(\F_3)$ is of order $2$.
\end{example}

\begin{theorem}[Second Isomorphism Theorem]
Let $G$ be a group with $H,K \leq G$ and let $H \leq N_GK$. Then $HK / L \cong H/H\cap K$.
\end{theorem}

\begin{proof}
We know several things.
\begin{itemize}
\item $HK \leq H$ since $H \leq N_G K$;
\item $K \leq HK$, since we know that $H \leq N_G K$ and $K \leq N_G K$ so $K \leq HK$;
\item Now we can take the quotient $HK/K$, which is the left cosets of $K$ in $HK$. We have shown that $hK = h'K$ if and only if $h H \cap K = h' H \cap K$. Then define the map $\pi : H \to HK/K$ defined by $h \mapsto HK$. This is a homomorphism since $hKh'K = hh'K$ since that's how we defined multiplication. Then $\ker\pi$ is all elements $h$ of $H$ which map to the identity coset which happens if and only if $h \in K$, so $\ker \pi = \{h \in H \cap K\}$. Then by the First Isomorphism Theorem, $H/H \cap K \cong HK/K$. \qedhere
\end{itemize}
\end{proof}

\begin{example}
Let $G = S_3$, let $K = A_3$, and let $H = \langle (12) \rangle$. We know that $HK = S_3$ and $H \cap K = \{e\}$. Then we know that $HK/K = S_3/A_3 \cong \langle (12) \rangle / 1 \cong Z_2$.
\end{example}
\section{Monday, 1 October 2018}

\epigraph{``I've got H's on the brain.''}{}

\epigraph{``That's the third isomorphism theorem, I knew you wouldn't like it. It should take you anywhere from a day to seven years to become comfortable with it.''}{Miki}

\epigraph{``It's math....it keeps doing things like that.''}{Miki}

\subsection{Isomorphism Theorems Continued}

Recall from last lecture we developed the first two isomorphism theorems. Today, we'll cover the last two (or one, depending on your perspective).

\begin{theorem}[Third Isomorphism Theorem]
Let $G$ be a group and let $H,N$ be normal subgroups of $G$ with $N \subseteq H$. THen $G/N \big/ H/N \cong G/H$.
\end{theorem}

\begin{proof}
Consider a map $\phi : G/N \to G/H : gN \mapsto gH$. We need this map to be well-defined. Suppose that $g_1N = g_2N$. Then $g_1^{-1}g_2 \in N$, but $N \subseteq H$, and so $g_1H = g_2H$ and $\phi$ is well defined. We also need to know that this is a homomorphism. Consider $\phi(g_1N)\phi(g_2H) = g_1g_2H = \phi(g_1g_2N)$, and in fact we also know that $\phi$ is surjective. Consider $gH \in G/H$ and suppose that $gH = \phi(gN)$. Since $N \subset H$ this is well defined. Consider then that $\ker \phi : \phi(gN) = gH$. This happens if and only if $g \in H$ so $gN \subset H$ is a coset of $N$ in $H$ and $gN \in H/N$, so $\ker\phi = H/N$. Then by the First Isomorphism Theorem, we know that $G/N \big/ \ker\phi \cong G/H$.
\end{proof}

\begin{example}
Let $G = \Z$ with $N = \langle 10 \rangle$ and $H = \langle 2 \rangle$. Then $G/N = \{0 + N,\dotsc,9+N\}$ and $G/H = \{0 + H, 1 + H\}$. Then $H / N = \{0 + N, 2 + N,\dotsc,8 + N\}$. The idea here is that if you take $\Z \pmod{10}$, and then modulo the result by $2$, then it didn't really matter than we modded out by $10$ to begin with.
\end{example}

\begin{theorem}{The Totally not fourth isomorphism theorem}
Let $N \normal G$. THere is a correspondence (bijection) between subgroups of $G$ which contain $N$ and subgroups of $G/N$. That is,
\[ \pi : H \mapsto \pi(H), \quad \bar{H} \mapsto \pi^{-1}(\bar{H}). \]
Note that for any $H \leq G$ we know that $\pi(H) \leq G/N$. We require normality to ensure that $\pi$ is injective.
\end{theorem}

\begin{example}
Consider $G = S_3$ with $N = A_3$. Then $\pi(S_3) = G/N$ and $\pi(A_3) = N$. What is $\pi(\langle (12) \rangle )$? It's all of $G/N$.
\end{example}

\subsection{Why do people care about normal groups?}

\begin{definition}[Simple]
A group $G$  is simple if $\abs{G} > 1$ and $G$ contains no proper normal subgroups.
\end{definition}

\begin{definition}[Composition Series]
Consdier something like $1 = N_0 \normal N_1 \normal \cdots \normal N_r = G$ where $N_{i+1}/N_i$ is simple for all $0 \leq i \leq r-1$. As an example, $1 \normal A_3 \normal S_3$. Then $S_3/A_3 \cong Z_2$ and $A_3 / 1 \cong Z_3$. These series allow us to construct large groups whose multiplication is unknown, since normal subgroups multiply to form subgroups of something larger. For more information on this, see the \emph{Holder Program}, which was started in 1890 to classify simple groups and it took 103 years to actually classify them all. These series are \emph{almost} unique, where the quotient groups are unique up to a permutaiton, so the set of quotient groups are unique.
\end{definition}

\begin{definition}[Solvable groups]
A group $G$ is solvalble is $1 = N_0 \normal \cdots N_r = G$ and $N_{i+1}/N$ is abelian. This kind of object shows up a lot in Galois Theory. As it turns out, $A_1$ through $A_4$ are solvalble but $A_5$ and higher is not solvable, which is why we can't solve arbitrary quintics.
\end{definition}
\section{Wednesday, 3 October 2018}

\epigraph{``I will not try to decide whether that was happy or sad.''}{Miki}

\epigraph{``Try it if you don't believe me.''}{Miki}

\epigraph{``If you don't have surjectivity, you have nothing.''}{Miki}

Recall from last time that we defined a simple group to be a non-trivial group which has no proper normal subgroups. Observe that if $G$ is abelian and simple then it has no proper subgroups at all, since all subgroups would be normal.

\subsection{Permutations}

We'll take a shortcut throught linear algebra to talk about the signs of permutaitons; the book constructs the notion from scratch. Recall that we can switch the rows of a matrix using the permutation matrix $P_{mn}$, by which left multiplication swaps the rows $m$ and $n$. Now, we talk about this as the cycle $(mn)$, so for example
\[ \begin{pmatrix}
0 & 1 & 0 \\
1 & 0 & 0 \\
0 & 0 & 1
\end{pmatrix} \sim \sigma = (12) \in S_3. \]
Essentially, we start with $I_n$ and permute the rows according to $\sigma$ to yield the corresponding permutation matrix $P_\sigma$.

\begin{definition}[Sign of Permutation]
Let $\varepsilon : S_n \to \{\pm 1\} \cong Z_2$ by $\varepsilon(\sigma) = \det P_\sigma$. Then $\varepsilon$ is the \emph{sign} of $\sigma$.
\end{definition}

Note that $\varepsilon$ is actually a group homomorphism since the determinant is multiplicative; that is $\varepsilon(\tau\sigma) = \det(P_{\tau\sigma}) = \det(P_\tau)\det(P_\sigma) = \varepsilon(\tau)\varepsilon(\sigma)$. Then we can quite naturally ask, what is the kernel of $\varepsilon$. We define the terms \emph{even} and \emph{odd} to mean permutations whose sign is $+1$ and $-1$ respectively. Then $\ker\varepsilon = A_n \leq S_n$ is the set of all even permutations. This gives us a rigorous definition of the alternating group.

Let's note that a two-cycle in $S_n$ is a transposition, and we have already proven on homework that every element in $S_n$ can be written as the product of two-cycles. We can quite easily conclude that every transposition has a sign of $-1$.

\begin{proposition}
Let $\sigma \in S_n$ be a $k$-cycle. Then $\varepsilon(\sigma) = (-1)^{k-1}$.
\end{proposition}

\begin{problem}
How large is $A_n$?
\end{problem}

Since $\varepsilon$ is surjective, we know by the First Isomorphism Theorem that $S_n / A_n \cong Z_2$ (since $\Im\varepsilon = Z_2$), so $\abs{A_n} = n!/2$.

\begin{theorem}
The alternating group on $n$ letters is simple if $n \geq 5$. This was proven by Galois in the 1830's and is the reason for quintic insolubility.
\end{theorem}

\subsection{Actions}

Recall that an action is a map $\phi : G \times A \to A$ by $\phi(g,a) = g \cdot a$. This yields a homomorphism $G \to S_A$ by $g \mapsto \sigma_g$, where $\sigma_g$ is bijective for a fixed $g \in G$. Recall also for $a \in A$ the stabalizer $G_a$ is the set of $g$ for which $ga = a$, and the kernel of the action is the set of $g \in G$ for which $ga = a$ for all $a \in A$. We said that an action is \emph{faithful} if the kernel of the action is the identity; that is, different elements of $g$ give different permutaitons on $A$. Furthermore, the orbit of $a$ is the set of $ga$ for all $g \in G$. We proved on homework that the orbits partition $A$.

\begin{definition}[Transitive]
An action is transitive if all elements of $A$ are in a single orbit; i.e., $a \sim b$ for all $a,b \in A$.
\end{definition}
\section{Friday, 5 October 2018}

\epigraph{``I mean, $\infty!$ is a big number!''}{Miki}
\epigraph{``Oh well, we'll cry later.''}{Miki}

\subsection{Orbits and Stabalizers}

Miki introduced a new proposition today which I think is just the Orbit-Stabalizer theorem.

\begin{proposition}
Given an action $G \times A \to A$ and an $a \in A$ we know that $\abs{O_a} = [G : G_a]$ which tells us that $\abs{G} = \abs{O_a}\abs{G_a}$.
\end{proposition}

\begin{proof}
Define a map $\pi : \{\text{cosets of $G_a$ in $G$}\} \to O_a$.  Note that this is just a map, not a homomorphism. Define $\pi$ to be $gG_a \mapsto g \cdot a$. We'll show it's well-defined and injective at the same time. Suppose we have that $gG_a = hG_a$ which happens if and only if $g^{-1}h \in G_a$ or $g^{-1}h \cdot a = a$, so $ha = ga$. Since anything in the orbit is $g \cdot a$ for some $g$, we also know that $\pi$ is surjective; then $\pi$ is a bijection and so $\abs{O_a} = [G : G_a]$.
\end{proof}

\subsection{Cycles in \texorpdfstring{$S_n$}{Sn}}

Let $\sigma \in S_n$ be or order $k$. We want to write it as the product of disjoint cycles.
Consider the set $A = \{1,\dotsc,n\}$ and let $G = \langle \sigma \rangle$. We construct the action $\langle \sigma \rangle \times A \to A$. Consider the orbit $O_a$ of $a \in A$ under $\langle \sigma \rangle$. We know by the orbit-stabalizer theorem that there is a bijection between the cosets $G_a$ and the orbit of $a$. Since $\langle \sigma \rangle$ is cyclic we know that $G_a = \langle \sigma^r \rangle$ is also cyclic. By the definition of our map $\pi$ from the Orbit-Stabalizer theorem, we know that $\pi(\sigma^iG_a) = \sigma^i a$. Then $O_a = \{a, \sigma a, \dotsc, \sigma^{r-1}a\}$ then on $O_a$ we can say that $\sigma$ acts as an $r$-cycle. Since the orbits collectively partition $A$ we know that they are disjoint, and so we know that we can write $\sigma$ as the product of disjoint cycles, which is unique up to the order of the cycles and up to cyclic permutation within each cycle.
Note that since $\langle \sigma \rangle$ is cyclic (and therefore abelian) the cosets $G_a$ are simply $G/G_a$.

\subsection{Actions of \texorpdfstring{$G$}{G} on itself}

Previously we defined two cannonical actions of $G$ on itself, via \emph{left multiplication} where $G \times G \to G : (g,a) \mapsto g\cdot a$, and \emph{conjugation}, where $G \times G \to G : (g,a) \mapsto gag^{-1}$. In the first case, we know that the action of left multiplication is faithful, and gave us an injective homomorphism from $G$ to $S_G$ (i.e., finite $G$ always is isomorphism to a subgroup of $S_n$).

\begin{example}
Let $G = \Z$ and our action is $(i,j) \mapsto i+j$, so $\sigma_i(j) = i+j$. Let's consider the orbits of $0$ and $1$ under $\sigma_2$. Observe that $\cdots -4 \to -2 \to 0 \to 2 \to 4 \to \cdots = O_0$ while $O_1$ is just the odd integers. Now consider $H = 4\Z \subset G$, and let's consider  how $\sigma_2$ acts of $G/H$, or on the cosets of $H$ in $G$. Well, we know that $H = \{\bar{0}, \bar{1},\bar{2},\bar{3}\}$. Note that $\sigma_2$ becomes $(\bar{0}\bar{2})(\bar{1}\bar{3})$.
\end{example}

\begin{theorem}
Consider $G \times G \to G$ via left multiplication, and consider how this action acts on $H \leq G$. We know it acts like $g \cdot (aH) = gaH$. We can show this is well defined by noting that if $aH = bH$ then we know that $ga = gbh$ and so our action is well defined. Let $A$ be set set of cosets of $H$ in $G$, and we get a map $\pi : G \to S_A$. We know the following things.
\begin{enumerate}
\item $G$ acts transitively on $A$;
\item $G_{1H} = H$;
\item The kernel of $\pi$ is the intersection of all $gHg^{-1}$ for all $g \in G$. This is actually the largest normal subgroup of $G$ contained in $H$.
\end{enumerate}
\end{theorem}

\begin{proof}
Left as an exercize to the reader (me).
\end{proof}
\section{Monday, 8 October 2018}

\epigraph{``Remember that $1+1 = 2$.''}{Miki}
\epigraph{``This proof should make you feel better after your exam.''}{Miki}

Recall from last lecture we described how $G$ can act on itself by either left multiplication or conjufgation. Today we'll cover in detail conjugation.
Remember that conjugation is an action defined as 
\[ G \times G \to G : ga \mapsto gag^{-1}. \]
We define the orbit $O_a$ of $a$ under conjugation to be the \emph{conjugacy class} of $a$. This is an equivalence relation (since we already know this holds for orbits). Then consider $S_1, S_2 \subset G$. There are conjugate if there exists a $g \in G$ such that $gS_1g^{-1} = S_2$. Note that these subsets better have the same cardinality. 

For any $x \in G$, we know that $\abs{O_x} = [G : G_x]$ where $G_x = G_G(x)$ is the centralizer of $x$ in $G$, as it turns out.

\begin{example}[Conjugacy Classes]
\begin{enumerate}
\part Consider $G = C_6$. Since $G$ is abelian, we know that $gxg^{-1} = x$ for all $x,g \in G$ so the orbit of $x$ is simply $\{x\}$.
\part Consider $D_8 = \langle r,s \mid \cdots \rangle$. What is the center of $G$? It's $Z(G) = \{1,r^2\}$. Let's consider that orbit of $1$ is $1$ and the orbit of $r^2$ is just $r^2$. This tells us that if $x \in Z(G)$ then $O_x = \{x\}$ under conjugation. Then let's consider some $x \notin Z(G)$. The size of the orbit must be strictly greater than $1$ (since otherwise it would commute with everything). Consider that the sabalizer must have at least three elements ($1,r^2,x$), and so must have at least order four. It can't have order eight since the identity will not be in the centralizer, and so we know that $\abs{O_x} = 2$. This tells us that the orbit of \emph{any non-center} element has order two, which tells us that we can find the orbits of any elements \emph{super quickly} since we just need to conjugate it \emph{once} and get a new element and we are done!
\end{enumerate}
\end{example}

\begin{theorem}[Class Equation]
Let $G$ be a group. The center of the group contains all conjugacy classes of size $1$. List the classes of size greater than or equal to $2$ as $)_{x_1} ,\dotsc,O_{x_n}$. Then 
\[ \abs{G} = \abs{Z(G)} + \sum_{i=1}^n \abs{O_{x_i}} \]
since the orbits partition $G$.
\end{theorem}

\begin{theorem}
Let $\abs{G} = p^n$ where $p$ is prime. We know a few things about such a group.
\begin{enumerate}
\item $\abs{Z(G)} > 1$.
\end{enumerate}
\end{theorem}

\begin{proof}[Proof of 1]
We know from the class equation that \[ \abs{G} = \abs{Z(G)} + \sum_{i=1}^n \abs{O_{x_i}}. \] For all $i$ we know that $O_{x_1} \geqslant 2$ and we know that it divides the order of the group. Then $\abs{O_{x_1}} = p^k$ for some $1 \leqslant k \leqslant n$. Then $p$ divides the order of the sum of the orders of the orbits, and so $p$ must divide the order of the center. In fact, this tells us that $\abs{Z(G)}$ is at least $p$.
\end{proof}

\begin{proposition}
For prime $p$, if $\abs{G} = p^2$ then
\begin{enumerate}
\item[(a)] $G$ is abelian, and 
\item[(b)] $G \cong C_{p^2}$ or $C_p \times C_p$.
\end{enumerate}
\end{proposition}

\begin{proof}[Proof of~\textup{(a)}]
Let $x \in G$, and assume by way of contradiction that $x \notin Z(G)$. Consider $H = \langle Z(G), x \rangle$. Since $G$ is a $p$ group, we know that the center cannot be one, and so $\abs{G} = p$ (since it must divide $p^2$ and if $\abs{G} = p^2$ then $x \in Z(G)$). And we know that $\abs{H} \geqslant \abs{Z(G)}$ so we know that $p \leqslant \abs{H} \leqslant p^2$ and the order must divide $p$, and so we know that $H = G$. But since $x$ commutes with everything in $H$ we know that $x \in Z(G)$.
\end{proof}

\begin{proof}[Proof of~\textup{(b)}]
Left as an exercise to us.
\end{proof}

\subsection{Conjugation in \texorpdfstring{$S_n$}{Sn}}

If you take an arbitrary element $\sigma$ of $S_n$, what can we reasonably expect the conjugacy class of $\sigma$ to look like? For example, consider $\sigma = (123)$. Well, $(14)(123)(14) = (423)$. We also know that $(256)(123)(652) = (153)$. Notice that we've found two $3$-cycles! We can hypothesis that $\abs{\tau \sigma \tau} = \abs{\sigma}$, and in fact we just replace the elements of the $3$-cycle with ``where the numbers in the conjugating cycles get sent.'' That is, if $\sigma = (a_1,\dotsc,a_k)$ then $\tau\sigma\tau^{-1} = (\tau(a_1)),\dotsc,\tau(a_k))$. We can infer a slightly stronger link here; in fact, $\sigma$ is conjugate to $\sigma'$ if and only if they have the same cycle structure.
\section{Friday, 12 October 2018}

\epigraph{``We have hope. But hope doesn't mean much.''}{Miki}

Let's return to the proposition we described last time, where we said that the equivalency classes in $S_n$ under conjugation are exactly the sets of permutations with the exact same cycle decomposition structure. That is, all elements of the form $(\cdot \cdot \cdot \cdot \cdot) \in S_n$ are conjugates with one another, and the same holds for $(\cdot \cdot \cdot)(\cdot \cdot)$ and all other cycle structures.

\begin{proposition} \hfill
\begin{parts}
\part[prf:cda] If $\sigma \in S_n$ is a $k-cycle$ where $\sigma = (a_1,\cdots,a_k)$, and $\tau \in S_n$, then $\tau\sigma\tau^{-1} = (\tau(a_1), \dotsc, \tau(a_k))$.
\part[prf:cdb] If $\sigma$ is a product of disjoint cycles $\sigma_i \cdots \sigma_r$ then $\tau\sigma\tau^{-1}$ is the product of disjoint cycles $\tau \sigma_i \tau^{-1}$.
\part[prf:cdc] Cycles $\sigma, \sigma'$ are conjugate if and only if they have the same cycle structure.
\end{parts}
\end{proposition}

\begin{proof}[Proof of\/~\ref{prf:cda}]
Let $A = \{1, \dotsc, n\}$ so that $\tau A = A$. Then $A = \{\tau(1), \dotsc, \tau(n)\}$. \note{FINISH THIS}
\end{proof}

\begin{proof}[Proof of\/~\ref{prf:cdb}]
Let $\sigma = \sigma_1 \cdots \sigma_r$. Then $\tau\sigma\tau^{-1}$ can be written as $\tau\sigma_1(\tau^{1}\tau) \cdots (\tau^{1}\tau) \sigma_r \tau^{-1}$, and by associativity the proposition holds. Since the cycles were disjoint to begin with, permuting each $\sigma_i$ under $\tau$ ensure that the products are still disjoint.
\end{proof}

\begin{proof}[Proof of\/~\ref{prf:cdc}]
The forward direction follows immediately from the previous two proofs.
Next, assume $\sigma, \sigma'$ have the same cycle structure. Then \[\sigma = (a_1^1 \cdots a_{k_1}^1)(a_1^2 \cdots a_{k_2}^2) \cdots (a_1^r \cdots a_{k_r}^r),\] and \[\sigma' = (b_1^1 \cdots b_{k_1}^1)(b_1^2 \cdots b_{k_2}^2) \cdots (b_1^r \cdots b_{k_r}^r).\] Then $A = \{1, \dotsc, n\} = \{a_i^j\} = \{b_i^j\}$. Then take $\tau(a_i^j) = b_i^j$, since this is just a permutation on the elements in $A$, so by \ref{prf:cda} and \ref{prf:cdb} this holds.
\end{proof}

\subsection{Proving the simplicity of \texorpdfstring{$A_5$}{A5}}

This is a big deal.

\begin{proof}
We want to show that $A_5$ (or any $A_n$, for that matter) has no proper normal subgroups. Recall the orbit stabalizer theorem, where $\abs{G} = \abs{G_x} \cdot \abs{O_x}$ for any $x \in G$. Recall also that if $N \normal G$ then $N$ is the union of conjugacy classes. Let's start by finding the class equation for $A_5$. Since $A_5$ must have even sign, we know that the only cycles in $A_5$ are of the form $e$, $(\cdot \cdot)$, $(\cdot \cdot \cdot)$, and $(\cdot \cdot \cdot \cdot \cdot)$. Let $O_{x}^{S_5}$ be the orbit of an element $x$ in $S_5$ while $O_{x}^{A_5}$ is the orbit in $A_5$. Note that $\abs{O_x^{A_5}} \leq \abs{O_x^{S_5}}$. Similarly, anything in $A_5$ which fixes $x$ must also fix $x$ in $S_5$ so $\abs{(S_5)_x} \geq \abs{(A_5)_x}$. We also know by Orbit-Stabalizer that $\abs{(A_5)_x} \cdot \abs{O_x^{A_5}} = \abs{A_5} = 60$ while $\abs{(S_5)_x} \cdot \abs{O_x^{S_5}} = \abs{S_5} = 120$. Combining these inequalities with the Orbit-Stabalizer theorem (and recognizing that everything here is an integer), we are left with the option that either the orbits are the same size and the centralizer in $A_5$ is half of the centralizer in $S_5$, or that the centralizers are the same and the orbits in $A_5$ are half that of the orbits in $S_5$.

Let's figure out which of these cases is true. Consider $x = (\cdot \cdot \cdot) = (123) \in S_5$ without a loss of generality. What is the size of the orbit of $x$ in $S_5$? Well, it's all three-cycles, so there are $2 \cdot {5 \choose 3} = 20$ elements in the orbit of $x$ in $S_5$. By Orbit-Stabalizer, the size of the sabalizer is then $120/20 = 6$. Note that $(45) \in (S_5)_x$ since it doesn't move $x$, but because $(45)$ is not in $A_5$ since it has the wrong sign, we know that it is the stablizer which has shrunk and the orbits have the same size.

Let's do the same thing with $x = (\cdot \cdot)(\cdot\cdot) = (12)(34)$ without a loss of generality. Then $\abs{O_x^{S_5}} = {5 \choose 1} \cdot 3 = 15$ elements in the orbit of $x$ in $S_5$. Since this is odd, we know that the orbit can't shrink so it \emph{again} must be the case that the stabalizer has shrunk.

Now let $x = ( \cdot \cdot \cdot \cdot \cdot)$. The orbit of $x$ is then of order $5!/5 = 4!$ while the stabalizer is of order $5!/4! = 5$. In this case, it is now the \emph{orbit} which has shrunk.

Then $\abs{A_5} = 1 + 20 + 15 + 2\cdot 12$ where $20$ comes from the $3$-cycles, $15$ comes from the double $2$-cycles, and the $24$ comes from the two $5$-cycles. Now suppose that $N \normal A_5$. We know it is the union of conjugacy classes and it contains the identity, so $\abs{N} = 1 + \{\text{some of }12,12,15,20\}$, and it must divide $\abs{A_5} = 60$. Note that this can happen \emph{only} if $\abs{N} = 1$ or $\abs{N} = 60$, so $A_5$ contains no proper normal subgroups and is simple.
\end{proof}
\section{Monday, 15 October 2018}

\epigraph{``Perfectly balanced, as all things should be.'' (when referring to left and right actions)}{Miki}
\epigraph{``Our theorem is gone! Oh no!''}{Miki}

\begin{problem}
Does right multiplication define an action of $S_4$ on itself?
\end{problem}
\begin{solution}
No, since we can find two elements for which $g_1(g_2(x)) \not= x \cdot (g_1g_2)$. Consider $(12)$ and $(23)$ acting on the identity. In general, right multiplication is an action if and only if the group is commutative since we are ``switching the order of the multiplication.''
\end{solution}

\subsection{Right Actions}
In order to fix this ``unfairness,'' we often define something called a right action $A \times G \to A$, where the associativity of the action is specified as $a \cdot (gh) = (a \cdot g) \cdot h$. This turns right multiplication into a ``right action.'' There really isn't any distinction between the two, which is why we just speak of ``the action.''

How do we turn left actions into right actions? Suppose we have a left action $g \cdot a$. Define $a \cdot g = g^{-1} \cdot a$; that is, the right action of $g$ on $a$ is just the left action by the inverse of $g$. This works since $(gh)^{-1} = h^{-1}g^{-1}$ so the order follows the rules for right multiplication/action.

\begin{problem}
Consider $\Z$ acting on itself through left addition, where $m \cdot n \mapsto m + n$, and consider that when we turn this into a right action. Then $n \cdot m \mapsto -m + n = n - n$, and we've just invented subtraction.
\end{problem}

\begin{example}
Consider $A_3 \normal S_3$, and consider conjugating $(123) \in A_3$ by something in $S_3$. We know we'll get either another three cycle or the identity, since we know that $gNg^{-1} = N$. Then if $g \in S_3$ there there exists a $\sigma_g : S_3 \to S_3$ which acts on $g$ by conjugation. The consider $\sigma_g |_N$ restricted to acting on $N$. Then we have a map from $N$ to itself. If $g \in N$ then we get the trivial map (since this is just $Z_3$), and otherwise we must not get the trivial map and so $(123) \mapsto (132)$ and \emph{vice versa}. In the latter case, we've created not just a random map but a homomorphism from $N$ to itself. This homomorphism $x \mapsto x^3$ in the group $\langle x \mid x^3 = 1 \rangle$, which is both injective and surjective and we know that this is a homomorphism since the generators satisfy the relations under the map since $x^6 = 1$.
\end{example}

\subsection{Group Automorphisms}

\begin{definition}[Automorphism]
A group automorphism is an isomorphism from $G$ to itself.
\end{definition}

For every group $G$ there is a group $\operatorname{Aut}(G)$ which is the group of all automorphism of $G$ under composition. Miki told us to prove for ourselves that this is actually a group. Now consider $G$ acting on itself through conjugation where $g \mapsto \sigma_g : x \mapsto gxg^{-1}$. For an normal subgroup of $G$ we know that $\sigma_g|_N : N \to N$, and so we have a homomorphism $\psi$ from $G$ to $\operatorname{Aut}(N)$ where $g \mapsto \sigma_g |_N$. The kernel of $\psi$ is the set of all elements in $G$ which commute with $N$, and so $\ker \psi = C_G(N)$. Then $G/C_G(N)$ is isomorphic to a subgroup of $\operatorname{Aut}(N)$ by the First Isomorphism Theorem.

There are two things to unpack here. First, how to we know that $\psi$ is actually a homomorphism? That is, why is $\sigma_g|_N \in \operatorname{Aut}(N)$? Well, consider that $\sigma_g(nn') = gnn'g^{-1} = gng^{-1} \cdot gn'g^{-1} = \sigma_g(n)\sigma_g(n')$, and so $\sigma_g |_N$ preserves the group operation. Next, how to we know that the map $g \mapsto \sigma_g$ is a homomorphism? That is, why does $\sigma_{gg'} = \sigma_{g}\sigma_{g'}$. Well, since conjugation is a well-defined action on $G$, this forms a homomorphism. Note that the restriction to $N$ isn't important here, but the reason we require normality since we won't we able to compose the conjugations since $gHg^{-1} \not= H$.

\begin{corollary}
Take $G = N$. Then we get a homomorphism from $G$ to its own automorphism group, and so $G/C_G(G) = G/Z(G)$ is isomorphism to a subgroup of $\operatorname{Aut}(G)$.
\end{corollary}

\begin{corollary}
Let $H \leq G$ be any subgroup of $G$. Then for all $g \in G$, $gHg^{-1} \cong H$, but they are not necessarily equal to one another.
\end{corollary}

\begin{corollary}
Let $H \leq G$ be any subgroup of $G$. Then $N_G(H) / C_G(H)$ is isomorphic to a subgroup of $\operatorname{Aut}(H)$, since the centralizer is always normal in the normalizer. This is really just a general case of the preceeding statements.
\end{corollary}

\begin{proof}
Since $H \normal N_G(H)$, we just let $G' = N_G(H)$ and apply the result.
\end{proof}
\section{Monday, 22 October 2018}

\epigraph{``You all look so unhappy.''}{Miki}
\epigraph{``$p$ is going to be prime for \emph{at least} two more days.''}{Miki}

\subsection{Clasifying Automorphisms}

Let's talk automorphisms!

\begin{definition}[Inner Automorphism]
Let $g \in G$ and let $\sigma_g : G \to G : x \mapsto gxg^{-1}$ be an automorphism (i.e., an automorphism by conjugation). Then $\sigma_g$ is an \emph{inner automorphism}. The collection of all inner automorphisms forms a group $\Inn(G) \leq \Aut(G)$ which is isomorphic to $G/\mathrm{Z}(G)$ by the first isomorphism theorem.
\end{definition}

\begin{example}
Let $G = \Z/n\Z$. We proved on homework that $\Aut(G) \cong (\Z/n\Z)^\times$, and so any $\sigma \in \Aut(G)$ is uniquely determined by the map which sends $1$ to $a$ for some unit $a$. Since $G$ is commutative, conjugation doesn't really do anything, so $\Inn(G) = \sigma_1$. Put another way, $\mathrm{Z}(G) = G$, so $\Inn(G)$ is as small as it could be.
\end{example}

\begin{example}
Let $G = D_8$. The center of $D_8$ is $\mathrm{Z}(D_8) = \langle r^2 \rangle$. We know that $\Inn(G) \cong G/\langle r^2 \rangle \cong K_4$.
\end{example}

\begin{definition}[Characteristic]
A subgroup $H \leq G$ is \emph{characteristic} if $\sigma(H) = H$ for any $\sigma \in \Aut(G)$. This is like a normal subgroup, except that a normal subgroup need only be preserved under \emph{inner automorphism} while a being characteristic subgroup is a stronger condition.
\end{definition}

\begin{example}
Let $G = D_8$ and let $H = \langle r^2 \rangle$. Since $H$ is the center, this is characteristic (this is true in general). Next let $K = \langle r \rangle \leq G$. Since $\Im(r)$ is either $r$ or $r^3$ (check the order under isomorphism) then $\sigma(\langle r \rangle) = \langle r \rangle$ for any $\sigma \in \Aut(D_8)$ and so it is characteristic.
\end{example}

Just to make the point explicit, if $H$ is characteristic in $G$ then it must be normal in $G$, but the reverse is not true. Additionally, if $H$ is the unique subgroup of a particular order in $G$ then it must be characteristic since there's nothing else it could be sent to under an automorphism since it's image must be a subgroup of the same order.

\subsection{Sylow \texorpdfstring{$p$}{p}-subgroups}

\begin{definition}
Let $p$ be prime. A $p$-subgroup is a subgroupd of order $p^n$ for $n \geq 0$.
\end{definition}

\begin{definition}
Let $\abs{G} = p^am$ where $p$ does not divide $m$. If there is a subgroup of order $p^a$ (there is) then a subgroup of this order is called a Sylow $p$-subgroup. The set of all such groups is written as $\Syl_p(G)$. The number of such groups is written as $n_p(G) = \abs{\Syl_p(G)}$.
\end{definition}

\begin{example}
If $p$ does not divide $\abs{G}$ the the only Sylow $p$-subgroup is the trivial subgroup. If $\abs{G} = p^a$ then the unique Sylow $p$-subgroup is $\Syl_p(G) = \{G\}$.
\end{example}

\begin{example}
Let $G = S_3$ which has order $2 \cdot 3$. Let $p = 2$, $m = 3$, and $a = 1$. Then the largest $\Syl_p$ subgroup is $C_2$, of which there are three such subgroups (things generated by $2$-cycles). If we let $p = 3$, then there is one Sylow $p$-subgroup, generated by a $3$-cycle.
\end{example}

\subsection{Sylow Theorems}

Throughout, let $p$ be prime and let $G$ be a group of order $p^am$ where $a > 0$ and $p$ does not divide $m$.

\begin{theorem}[Sylow I]
There exists a subgroup of $P \leq G$ where $\abs{P} = p^a$.
\end{theorem}

\begin{theorem}[Sylow II]
For each $p$, the Sylow $p$-subgroups are conjugate to one another.
\end{theorem}

\begin{theorem}[Sylow III]
The number of Sylow $p$-subgroups of $G$, written $n_p(G)$, divides $m$ and is congruent to $1 \bmod{p}$.
\end{theorem}

We'll prove these next time (with a lot of chocolate). Today we'll just talk about the implications of these theorems.

\begin{corollary}
There must exist an $x \in G$ whose order is $p$.
\end{corollary}
\begin{proof}
Let $y \in P$ be not the identity. Then $\abs{y} = p$, so for some $0 < b \leq a$ we know that $x = y^{p^{b-1}}$.
\end{proof}

\begin{corollary}
The Sylow $p$-subgroups are all conjugate.
\end{corollary}
\section{Wednesday, 24 October 2018}

Today was a presentation of the proof of the Sylow theorems found in the textbook. As such, notes are omitted in favor of reading the relevant section in the book (and I'm rather tired today and I don't want to type anything up).
\section{Friday, 26 October 2018}

Didn't go to class today! Something about direct products I think.
\section{Monday, 20 October 2018}

\epigraph{``Let's write down all finitely generated abelian groups. What fun.''}{Miki}

We have two goals for today.
\begin{enumerate}
\item Is $Z_{20} \times Z_{18} \cong Z_{36} \times Z_{10}$?
\item How do we classify \emph{all} finitely generatred abelian groups?
\end{enumerate}

To start answering these, we'll begin with a proposition.

\begin{proposition}
$Z_n \times Z_m \cong Z_{mn}$ if and only if $\gcd(m,n) = 1$.
\end{proposition}

\begin{proof}
Let $d = \gcd(m,n)$, and let $Z_m = \langle x \rangle$ and let $Z_n = \langle y \rangle$. Consider $G = Z_m \times Z_n = \{(x^a, y^b)\}$. Consider $(c,f) \in G$. Then $\abs{(c,f)} = \lcm (\abs{c}, \abs{f})$. If $d = 1$ then $\abs{(x,y)} = \lcm (m,n) = mn$, so $Z_{mn} \cong \langle (x,y) \rangle \leq G$. Since the orders are the same, it is isomorphic to the whole thing.
On the other hand, if $d > 1$, let $(c,f) \in G$, and consider $(c,f)^{mn/d} = (c^{mn/d}, f^{mn/d}) = (e,e)$, so every element has order strictly less than $mn$ since $d > 1$. Therefore $G \not\cong Z_{mn}$.
\end{proof}

\begin{example}
Consider $Z_9 \times Z_6 \not\cong Z_{54}$. Note that $Z_9 \times Z_6 \cong Z_9 \times Z_3 \times Z_2 = Z_{18} \times Z_3$.
\end{example}

\begin{example}
Use the proposition we just proved to ``factor'' the groups into the same decomposition. Ta-Da!
\end{example}

fdas

\subsection{Classifying Finitely-Generated Abelian Groups}

\begin{definition}[Free Abelian Group]
Let $\Z^r = \Z \times \cdots \times \Z$ ($r$ times) be the free abelian group of rank $r$.
\end{definition}

\begin{theorem}[Classification Theorem for Finitely Genreated Abelian Groups]
Let $G$ be a finitely generated abelian group. Then there is a unique decomposition of $G$ satisfying
\begin{enumerate}
\item $G \cong \Z^r \times Z_{n_1} \times \cdots Z_{n_s}$ for $r,n_i \in \Z$,
\item $n_i > 2$ for all $i$, and 
\item $n_{i+1}$ must divide $n_i$ for all $1 \leq i \leq s-1$.
\end{enumerate}
\end{theorem}

\subsection{Classifying Finitely-Generated Abelian Groups 2, Electric Boogaloo}

\begin{example}
Consider $Z_{60} \cong Z_{2^2} \times Z_{3} \times Z_{5}$. Notice now that all the components are $p$-subgroups.
\end{example}

\begin{theorem}
Let $\abs{G} = n = \prod p_i^{a_i}$, where $a_i \geq 1$. Then we can write $G$ uniquely (up to order of primes) as $G \cong A_1 \times \cdots \times A_k$ where $\abs{A_i} = p_i^{a_i}$, and for all $A = A_i$ where $\abs{A} = p^a$, we know that $A \cong Z_{p^{b_1}} \times \cdots \times Z_{p^{b_\ell}}$ where $b_1 \geq b_2 \geq \cdots \geq b_\ell$, where the sum of all $b_i$ is $a$.
\end{theorem}

\section{Wednesday, 31 October 2018}

\epigraph{``Oh. You are a gamer.''}{Miki}

\subsection{FGAGs for Dayyyyysssss}

Miki started us off with some exercises to apply what we learned last lecture about classifying finite abelian groups.

\begin{exercise}
Conver $G = Z_{36} \times Z_{12}$ into elementary divisor notation.

\begin{solution}
$G \cong (Z_{2^2} \times Z_{3^2}) \times (Z_{2^2} \times Z_3) \cong (Z_{2^2} \times Z_{2^2}) \times (Z_{3^2} \times Z_3)$.
\end{solution}
\end{exercise}

\begin{exercise}
Convert $(Z_{16} \times Z_{4} \times Z_2) \times (Z_9 \times Z_3)$ into invariant factor notation.

\begin{solution}
Group up the $i$\textsuperscript{th} terms in each parenthetical term, so $G \cong (Z^{16} \times Z_9) \times (Z_4 \times Z_3) \times Z_2 \cong Z_{144} \times Z_{12} \times Z_{2}$. 
\end{solution}
\end{exercise}

\begin{exercise}
Classify all abelian groups of order $24$.

\begin{solution}
Let's use the invariant factor notaion. If $p$ divides the order of $\abs{G}$ then $p$ divides $n_1$. The factorization of $24$ is $2^3 \times 3$. Then $n_1$ could be $24$, and $G_1 \cong Z_{24}$. It could be that $n_1 = 4 \cdot 3$, so $n_2 = 2$ and $G_{2} \cong Z_{12} \times Z_2$. It could be that $n_1 = 2 \cdot 3$, so $n_2 = 2$ and $n_3 = 2$, (can't be $n_2 = 4$ since $4$ doesn't divide $6$), so $G_3 \cong Z_6 \times Z_2 \times Z_2$.
\end{solution}

\begin{solution}
Let's use the elementary divisor notation. Let $\abs{H} = 24$. Then $H \cong A_1 \times A_2$ where $\abs{A_1} = 2^3$ and $\abs{A_3} = 3$. Then $A_2 \cong Z_3$. For $A_1$, we must take all non-increasing partitions of $3$, so $3 = 3, 2 + 1, 1+1+1$. The the possibilities for $A_1$ are $Z_{2^3}$, $Z_{2^2} \times Z_2$, and $Z_2 \times Z_2 \times Z_2$. Then $H$ is either $Z_3 \times Z^{2^4} \cong Z_24$ or $Z_3 \times Z_{2^2} \times Z_2 \cong Z_{12} \times Z_2$ or $Z_3 \times Z_2 \times Z_2 \times Z_2 \cong Z_6 \times Z_2 \times Z_2$.
\end{solution}
\end{exercise}

\subsection{The Shape of Things to Come}

Over the next few lectures, we'll cover how to take the product of groups which aren't abelian, and understanding how we can ``factor'' non abelian groups in the same way that we now know how to factor abelian groups.
Warning: it'll be the hardest single thing we do in this class.

\subsection{Commutators}

Let $G$ be a group with $x,y \in G$. We defined the \emph{commutator} to be $[x,y] = xyx^{-1}y^{-1}$. The commutator of any two elements of $G$ is one if and only if $xy = yx$. The commutator subgroup $G' = \langle [x,y] \mid x,y \in G \rangle$, which is normal in $G$ and the quotient $G/G'$ is abelian.

Suppose we had a homomorphism $\phi$ from $G$ to $H$, where $H$ is abelian. Then it must be that $G' \leq \ker \phi$, since $\phi(x)\phi(y) = \phi(y)\phi(x)$, so $[\phi(x),\phi(y)] = \phi([x,y])$ for all $x,y \in G$.

The quotient of $G$ by $G'$ is the largest abelian quotient of $G$, which means that the commutator group is the smallest subgroup for which the quotient is abelian.

\begin{example}
Let $G = S_3$. What is $G'$? Given the sign map $\pi : S_3 \to Z_2$, we know that $\ker \pi = A_3$, so we know that $G' \leq A_3$ since $Z_2$ is commutative. Then $G' = A_3$.
\end{example}

\begin{example}
Let $G = D_{12}$. Is it a direct product of some proper subgroups? Consider $K = \langle r^3 \rangle \leq Z(G)$ and let $H = \langle s, r^2 \rangle$. Notice that $\langle H,K \rangle \leq G$ and $\langle H,K \rangle$ contains $s$ and $r$, so it is equal to the whole group. It's also true that $H$ and $K$ commute with one another. As it turns out (note quite a proof yet) $G \cong K \times H$. 
\end{example}

\begin{theorem}
Let $G$ be a group and let $H,K$ be subgroups of $G$ satisfying
\begin{enumerate}
\item $H \normal G$ and $K \normal G$, 
\item $H \cap K = \{e\}$, and 
\item $\langle H,K \rangle = HK = G$.
\end{enumerate}
Then $G \cong H \times K$.
\end{theorem}

\begin{proof}
First, we show that $H$ and $K$ commute with one another. Consdier $[h,k]$. Notice that $(hkh^{-1})k^{-1} \in K$ and  $h(kh^{-1}k^{-1}) \in H$, so it's in both $H$ and $K$, but the intersection of $H$ and $K$ is $\{e\}$ which means that the commutator is the identity, which happens if and only if $H$ and $K$ commute with one another. Next, consider that $\abs{HK} = \abs{H}\cdot\abs{K}/ \abs{\{e\}} = \abs{H} \cdot \abs{K}$. Next, create a map $\phi : H \times K \to G$ where $(h,k) \mapsto hk$. We show that this is a homomorphism (and an isomorphism). Notice that $\phi(h,k)\phi(h',k') = hh'kk' = \phi(hh',kk')$ so $\phi$ is a homormphism. It is also injective. Suppose that $\phi(hk) = 1$, which tells us that $h = k^{-1}$, which means that $h = k = 1$ since $h \in H \cap K$. It is surjective, since the sizes of the groups are the same. Then $G \cong H \times K$ through $\phi$.
\end{proof}



\section{Friday, 2 November 2018}

\epigraph{``You're not going to like this.''}{Miki}

\subsection{Semidirect Products}
We want to generalize our understanding of the direct product to non-commutativity.

\begin{example}
Let $G = D_6$, and let $H = \langle r  \rangle \cong Z_3$ and let $K = \langle s \rangle \cong Z_2$. Notice $H \normal G$. Since their intersection is trivial, $HK = G$. Here, we know that $G \not\cong H \times K$ (since then it would be abelian) but it's really similar, since we can write any $g$ as a product of $h$ and $k$.
\end{example}

We kinda cheated with this example since we already know how $r$ and $s$ relate to one another, but what if we don't know how to ``conjugate'' them? What if we don't know how to multiply words?

\begin{definition}[Semidirect Product]
Let $H$ and $K$ be groups, and let $\phi : K \to \Aut(H)$ be a homomorphism (aside from the trivial one, this may not exist). Then $\phi$ defines an action of $K$ on $H$ where $k \cdot h = \phi(k)(h)$. Then $G = H \rtimes_\phi K$ is called the semidirect product, where $g \in G = (h,k)$ and $(h_1,k_1)(h_2,k_2) = (h_1(k_1 \cdot h_2), k_1,k_2)$. This is a generalization of conjugation.
\end{definition}

There are some important properties of this. First, $G$ is a group where $\abs{G} = \abs{H} \cdot \abs{K}$. Second, $H \leq G$ via $h \mapsto (h,1)$ and $K \leq G$ via $k \mapsto (1,k)$. Third, $H \cong \langle (h,1) \rangle \normal G$. Fourth, $H \cap K = \{e = (1,1)\}$. Finally, for all $k \in K$ and $h \in H$ we know that $k \cdot h = khk^{-1}$.

\begin{example}
Let $K = \langle k \rangle \cong Z_2$ and $H = \langle h \rangle  \cong Z_3$. What could $H \rtimes K$ be? First, let's find $\Aut(H) \cong (\Z/(3))^\times$, so there are only two possible choices for our defining homorphism; it can be either the identity, or one other map. Then let $\phi : k \mapsto e$, so $h \cdot h = h$ and the groups commute, so $H \rtimes_\phi K \cong H \times K$. Or, we could take that $\psi : k \mapsto \chi$ where $k \cdot h = h^2$ and where $\chi$ is the other element of $\Aut(H)$, and so $H \rtimes_\varphi K \cong S_3$. Notice that $H \rtimes_\phi K \not\cong H \rtimes_\varphi K$.
\end{example}

\begin{example}
What if we keep $H$ and $K$ but make $H \cong Z_2$ and $K \cong Z_3$ (i.e., we flip the group order)? Well, $k \cdot 1 = 1$ and then, by exhaustion, $k \cdot h = h$ (since that's all we can do with a homomorphism), so $H \rtimes K \cong H \times K$. This tells us that $H \rtimes K$ is, in general, not the same thing as $K \rtimes H$.
\end{example}
\section{Monday, 5 November 2018}

\epigraph{``Now everybody is happy. Or not, but it works.''}{Miki}

\subsection{More Semidirect Products}

A continuation of the discussion from last lecture.

\begin{example}
Suppose $H \cong Z_{20} \times Z_{45}$ (or any abelian group $\abs{H} > 2$). Since $H$ is abelian we know that the map $\gamma \in \Aut(H) : x \mapsto x^{-1}$ is a homomorphism. Then $\gamma$ is of order two. Let's take some semidirect products. Consider 
\begin{parts}
\part[prt:sd:a] $H \rtimes Z_2$ where $\phi(k) = \gamma$. Then $k \cdot h = h^{-1}$ for all $h \in H$.
\part[prt:sd:b] $H \rtimes Z_4$. Here we could also send $k$ to $\gamma$ since $\gamma^4 = 1$. Then $\phi(k^2) = 1$, so $k \cdot h = h^{-1}$, and $k^2 \cdot h = h$. This sends $Z_4$ to $Z_4 / \langle k^2 \rangle \cong Z_2$ to $\langle \gamma \rangle$.
\part[prt:sd:c] $H \rtimes S_3$. We can quotient $S_3$ by $A_3$, which is isomorphic to $Z_2$, and then we map this into $\Aut(H)$. Composition of these maps yields a nontrivial homomorphism $\phi$. Here, $\psi : S_3 \to S_3/A_3$ is the sign map, so for all $\sigma in \S_3$, we say that $\psi :\sigma \mapsto \gamma$ if the sign of sigma is $-1$, and $\sigma \mapsto 1$ if the sign of sigma is positive. 
\end{parts}
\end{example}

\subsection{Identitfying Groups as Semidirect Products}

Given a group $G$, how can we tell what its semidirect product decomposition could be?

\begin{theorem}
Let $G$ be a group with $H,K \leq G$ such that 
\begin{enumerate}
\item $H \normal G$,
\item $H \cap K = \{1\}$, and 
\item $G = HK$.
\end{enumerate}
Then $G \cong H \rtimes K$ where $k \cdot h = khk^{-1}$ for all $h \in H$ and for all $k \in K$.
\end{theorem}

This is really similar to the requirements for $G$ being the \emph{direct product} of $H$ and $K$; we just drop the requirement that $K \normal G$.

\begin{proof}
We know that $H \cap K = \{1\}$ so $\abs{H}\abs{K} = \abs{G}$, so for every $g \in G$ there is a unique way of writing it as $hk$. Then we can create a well-defined map $HK = G \to H \rtimes K$ via $\pi : g \mapsto (h,k)$. To show that this map is onto, observe that we can get all elements of the form $(1,k)$ and $(h,1)$, so any $(h,k)$ is in the image of $hk$ under our map. We know that $\abs{HK} = \abs{H \rtimes K}$ so they are isomorphic if we can show that $\pi$ is a homomorphism (turns out, it is). Proof of this is left as an exercise to the reader.
\end{proof}

\subsection{Classification}

Let $\abs{G} = pq$ where $p \leq q$. We showed with the Sylow theorems that $P \in \Syl_p(G)$ and $Q \in \Syl_q(G)$ imply that $Q \normal G$. We also showed that if $p$ doesn't divide $q - 1$ then $n_p = 1$ and $P \normal G$. In these cases, $G \cong Z_{pq}$. What happens if $p$ \emph{does} divide $q-1$? In this case we don't need to create a $Z_{pq}$, and this is where semidirect products come in. Here, we can apply the previous theorem to construct groups of the form $Q \rtimes_\phi P$ for some $\phi$.

Let's look at $\Aut(Q)$, which we know is isomorphic to $(\Z_q)^\times$ so the order is $q-1$. We also know\footnote{since $q$ is prime} that $\Aut(Q)$ is cyclic, so $\Aut(Q) \cong Z_{q-1}$. We know that $p$ divides $q-1$, so there is a unique subgroup $\langle \gamma \rangle \leq \Aut(Q)$ which is isomorphic to $Z_p$. We need to map a generator of $P$ into $\Aut(Q)$, so it must have order $1$ or $p$. Let $P = \langle x \rangle$. We have a few options. We know that $\phi(x) = \gamma^i$ for some $0 \leq i \leq p-1$ since we need $\phi(x)^p = 1$.
\begin{enumerate}
\item The case where $i = 0$, wherein have the trivial action. Then $P$ and $Q$ commute one another since $x \cdot y = y$, and so $G \cong Q \times P$.

\item The case where $i = 1$. Then $G \cong Q \rtimes_{\phi_1} R$ where $x \cdot y = \gamma(y)$. This involves mapping $x$ to $\gamma$. This semidirect product is nonabelian. What exactly $\gamma$ is depends on $P$ and $Q$, but we konw that it does exist since $p$ divides $q-1$, and so a $p$-order subgroup of $\Aut(Q)$ must exist.

\item The case where $i > 1$. We know that $x \mapsto \gamma^i$ since $P \cong \langle \gamma \rangle$. Then call $x'$ the element which is mapped to $\gamma$. Then we revert to the case where we defied $\phi_1$ by $x'$, so all cases where $i \not= 0$ are isomorphic, and the exact value depends on which $x'$ you choose. 
\end{enumerate}
\section{Wednesday, 7 November 2018}

\subsection{Classify Groups of Order \texorpdfstring{$12$}{12}}
Online. Look it up.

\subsection{``A Simple Song''}

This may be the best thing I have ever seen. Lyrics to follow shortly.

\begin{raggedright}
\raggedright\itshape\small
What are the orders of all simple groups? \\
I speak of the honest ones, not of the loops. \\
It seems that old Burnside their orders has guessed. \\
Except for the cyclic ones, even the rest. \\ \hspace{1em} \\

Groups made up with permutes will produce
some more: \\
For A\textsubscript{n} is simple, if n exceeds 4. \\
Then, there was Sir Matthew who came into
view \\
Exhibiting groups of an order quite new. \\ \hspace{1em} \\

Still others have come on to study this thing. \\
Of Artin and Chevalley now we shall sing. \\
With matrices finite they made quite a list. \\
The question is: Could there be others they've missed? \\ \hspace{1em} \\

Suzuki and Ree then maintained it's the case \\
That these methods had not reached the end of
the chase. \\
They wrote down some matrices, just four by
four, \\
That made up a simple group. Why not make
more? \\ \hspace{1em} \\

And then came the opus of Thompson and Feit, \\
Which shed on the problem remarkable light. \\
A group, when the order won't factor by two, \\
Is cyclic or solvable. That's what is true. \\ \hspace{1em} \\

Suzuki and Ree had caused eyebrows to raise, \\
But the theoreticians they just couldn't faze. \\
Their groups were not new: if you added a twist, \\
You could get them from old ones with a flick of the wrist. \\ \hspace{1em} \\

Still, some hardy souls felt a thorn in their side. \\
For the five groups of Mathieu all reason defied; \\
Not $A_n$, not twisted, and not Chevalley, \\
They called them sporadic and filed them away. \\ \hspace{1em} \\

Are Mathieu groups creatures of heaven or hell? \\
Zvonimir Janko determined to tell. \\
He found out what nobody wanted to know: \\
The masters had missed 1 7 5 5 6 0. \\ \hspace{1em} \\

The floodgates were opened! New groups were the rage! \\
(And twelve or more sprouted, to greet the new age.) \\
By Janko and Conway and Fischer and Held \\
McLaughlin, Suzuki, and Higman, and Sims. \\ \hspace{1em} \\

No doubt you noted the last lines don't rhyme. \\
Well, that is, quite simply, a sign of the time. \\
There's chaos, not order, among simple groups; \\
And maybe we'd better go back to the loops
\end{raggedright}

\subsection{Introductions to Rings}

\begin{definition}[Ring]
A \emph{ring} is a tuple $(R,+,\times)$ where $(R,+)$ is an abelian group, $\times$ is an associative binary operaation, and the distributive laws hold.
\end{definition}
\section{Friday, 9 November 2018}

\epigraph{``I'm an advertisement for Czech people.''}{Miki}

\subsection{Warmups}

\begin{exercise}
Are the following objects rings?
\begin{parts}
\part The set of all even integers with the operations you expect.
\part The set of all odd integers with the operations you expect.
\end{parts}
\end{exercise}

\begin{solution}
The even integers do form a ring (we don't require a ring to have identity) while the odd integers don't (there is no additive identity and it isn't even closed).
\end{solution}

\subsection{Quaternions, Division Rings, and Fields, Oh My!}

\begin{exercise}
What is $\C$?
\end{exercise}

\begin{solution}
We can think of it as a vector space over $\R$ with a basis $\{1,i\}$, which gives us how to multiply and add things in $\C$ if we keep in mind that $i^2 = 1$.
\end{solution}

We can extend this idea of $\C$ into a higher-dimensional object $\mathbb{H}$, known as the \emph{Hamiltonian quaternions}. This forms a vector space over $\R$ with a basis $\{1,i,j,k\}$ where the ring multiplication is defined by the group $Q_8$. Some ``fun'' facts about $\mathbb{H}$:
\begin{enumerate}
\item $\mathbb{H}$ is not commutative;
\item $\mathbb{H}$ is a ring with identity. For any $a + bi + cj + dk$, the inverse is $(a - bi - cj - dk)/(a^2+b^2+c^2+d^c)$.
\end{enumerate}

We can extend these definitions even further.

\begin{definition}[Division Ring]
A ring $R$ with identity $1$ is a \emph{division ring} if for all nonzero $a \in R$ there exists an $a^{-1} \in R$
\end{definition}

\begin{definition}[Field]
A \emph{field} is a commutative division ring.
\end{definition}

About all rings we can make some claims.

\begin{proposition}
Let $R$ be a ring. Then
\begin{enumerate}
\item $0a = a0 = 0$ for all $a \in R$;
\item $(-a)b = a(-b) = -(ab)$ for all $a,b \in R$;
\item $(-a)(-b) = ab$ for all $a,b \in R$;
\item If $R$ has an identity the it is unique and satisfies $(-a)1 = 1(-a) = -a$ for all $a \in R$.
\end{enumerate}
\end{proposition}

\begin{proof}
If you took 230, you've already done this. If not, it's a good if somewhat tedious exercise.
\end{proof}

\begin{definition}[Zero Divisor]
An nonzero $a \in R$ is a zero divisor if there exists a nonzero $b \in R$ for which $ab = 0$ or $ba = 0$.
\end{definition}

\begin{definition}[Unit]
Let $R$ be a ring with identity. A unit is any element $x \in R$ if there exists a $u = x^{-1} \in R$ for which $xu = ux = 1$. We denote the set of units of $R$ by $R^\times$.
\end{definition}

\begin{example}
Consider $R = \Z/6\Z$. Find all units and zero divisors.
\end{example}

\begin{solution}
The units are $1,5$, and the zero divisors are $2,3,4$.
\end{solution}

\begin{proof}[Proof that $2$ is not a unit]
Suppose by way of contradiction that there exist $x \in R$ such that $2x = 1$. Then $3(2x) = 3 = (3 \cdot 2)x = 0$, which is a contradiction. Generally, if $x \in R$ is a zero divisor then it is not a unit. However, the converse is not true (consider $2 \in \Z$, which is neither a unit nor a zero divisor).
\end{proof}

\begin{definition}[Integral Domain]
A commutative ring $R$ with identity is an integral domain if it has no zero divisors. This gives us the cancellation law! [The proof is super short and intuitive.]
\end{definition}

\begin{proposition}
Any finite integral domain is a field. The proof is pretty chill. Check it out some time!
\end{proposition}

\begin{definition}[Subring]
A subring $S \subseteq R$ is an object where $(S,+) \leq (R,+)$ and $S$ is closed under multiplication. For example, $2\Z \subset \Z$, $\Z \subset \Q$, $\Q \subset \R$, $\R \subset \C$, the set of continuous functions from $\R$ to $\R$ is a subring of all functions from $\R$ to $\R$.
\end{definition}

\begin{example}
Consider $\Q(i) = \{a + bi \mid a,b \in \Q\}$ where $i^2 = 1$. We know how to add and multiply this from middle school. This is actually a field equal to $\Q[i]$, where $\Q[i]$ is actually the field of rational polynomials over $1$ and $i$. Now consider $\Z[i]$ (this won't be a field), which is integral polynomials in $1$ and $i$. This isn't a field since $2$ has no inverse.
\end{example}

\begin{definition}[Norm]
Let the norm of $\Q(i)$ be a map $N : \Q(i) \to \Q$ defined by $a + bi \mapsto (a+bi)(a-bi) = a^2+b^2$. This map is multiplicative.
\end{definition}

\begin{proposition}
$N(x) \in \Z$ for all $x \in \Z[i]$, and $x$ is a unit if and only if $N(x) = \pm 1$.
\end{proposition}

\end{document}