%% Sanskrit
\section{August 30, 2018}

Linguistics was the model science in ancient india

Today, we'll talk about phonology --- range of sounds, how they're identified.
Phonology is the sound structure of a language, as opposed to syntax or semantics. For the time being we'll transliterate Sanskrit into the Latin script; tomorrow, we'll move on to devanagari. Sanskrit has been written in a variety of scripts, but devanagari is the most commonly used.
Much of the historic writings in Sanskrit are written in Devanagari or one of its predecessors.

Sanskrit's phonological system is very ``scientific'' --- it's very systematic and structured. Grammarians divide sounds in Sanskrit into two groups: the \emph{svara}, or vowels; and the \emph{vyañjana}, or consonants.

\subsection*{Svara}
In Sanskrit, svara (meaning sound, tone, or accent) are sounds which can be pronounced independently, contrasting with consonants which have an implied vowel associated with them.
There are 14 vowels in Sanskrit (really only 13) which are organized by two orthogonal qualities: the place of articulation and the length.

\begin{wrapfigure}{l}{0.25\textwidth}
\begin{tabular}{@{}cc@{}}
\toprule
Hrasva & Dīrga \\
\midrule
a & ā \\
i & ī \\
u & ū \\
ṛ & ṝ \\
ḷ & \emph{ḹ}\footnote{This vowel is never attested, and only exists to complete the chart} \\
\midrule
e & ai \\
o & au \\
\bottomrule
\end{tabular}
\end{wrapfigure}

Long and short vowels are distinguished by the number of \emph{mātrā} they consume.
You can think of mātrā as a kind of tempo marking for the language; short vowels, or \emph{hrvasa}, last for ``one beat'' of time, while long vowels, or \emph{dīrga}, last for two. In ancienct Vedic Sanskrit, there was a third category of vowels, called protracted vowels or \emph{pluta}, which lasted for three beats, but these were lost in Classical Sanskrit.
Vowels can also be simple (\emph{śuddha}, or ``pure''), or complex (\emph{samyukta}, or ``joined''). Samyukta are known in linguistics as diphthongs, and are made from the combination of two or more śuddha. In the table on the left, the samyukta are the bottom four.

Here are some examples of the vowels in Sanskrit words:

\begin{figure}[h]
\begin{center}
\begin{tabular}{@{}ll@{}}
\toprule
Abhava (absense, destruction) & Ātman (self) \\
Idam (this) & Īśvara (lord) \\
Ubhaya (both) & Abhūt (was) \\
Kṛṣṇa (Krishna) & Nṝṇām\footnote{The consonant ṝ is very rare, appearing only in nominative and genetive nouns} (Of men) \\
Kḷpti (arrangement) & --- \\
Etad (this) & Bhojana (food) \\
Chaitanya (a saint) & Saubhagya (good fortune) \\
\bottomrule
\end{tabular}
\end{center}
\end{figure}

\subsection*{Consonants}

 There are three kinds of consonants in SKRT. Consontants are classified by the place of articulation (sthāna):
 	- Velar or kan.t.hya
 	- soft pallette, palatal (talva)
 	- hard pallatte, retroflex (murtanyo)
 	- dental (danta)
 	- labial (os.t.ha)

 Also classified by the closure of the lips
 	- full contact, stops (sparśa)

 Voiced/Unvoiced
 	- ghośavat (voiced)
 	- aghośa 	(unvoiced)
 
 Apsiration
 	- in english, spin vs pin, cut vs king
 	- Alpha prana 	(unaspirated)
 	- maha prana	(aspirated)

