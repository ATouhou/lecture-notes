\documentclass{notes}

% Bibliography
\usepackage{biblatex}
\addbibresource{references.bib}

% Document Information
\title{Mathematics of Language}
\courseid{LING 224}
\place{Yale University}
\term{Fall}
\year{2018}

\blurb{
	These are lecture notes for LING 224a, ``Mathematics of Langauge,'' taught by Robert Frank at Yale University during the fall of 2018.
	These notes are not official, and have not been proofread by the instructor for the course.
	These notes live in my lecture notes respository at 
	\[\text{\url{https://github.com/jopetty/lecture-notes/tree/master/LING-224}.}\] 
	If you find any errors, please open a bug report describing the error, and label it with the course identifier, or open a pull request so I can correct it.
}

\begin{document}

\section*{Syllabus}

\begin{center}
\begin{tabular}{@{}rp{10cm}@{}}
\toprule
\textbf{Instructor} & Robert Frank, \url{bob.frank@yale.edu} \\
\textbf{Lecture} & MW 2:30--3:45 \textsc{pm}, WLH 114 \\
\textbf{Textbook} & \fullcite{PMW} \\
\bottomrule
\end{tabular} \\[3ex]
\end{center}

This course will be devoted to the study of formal systems that have proven useful in the science of language. We will discuss a range of mathematicals tructures and techniques and explore their applications in theories of grammatical competence and performance. A major goal of this course is to enable students to evaluate the strengths and weaknesses of existing formal theories of linguistic capacities, and profitably engage in formalization of theoretical proposals, constructing precise andc oherent definitions and rigorous proofs.

Learning this material is a bit like learning to dance:  you can’t do it by just watching.  Rather, in learning mathematics, it is absolutely crucial that you do some mathematics. To this end, there will be (roughly) weekly problem sets. These will in general be assigned on Wednesday and will be due one week later. These problem sets will account for 60\% of the grade. At approximately midway through the course and at the end of the term (precise dates to beannounced), I will assign longer problem sets with broader coverage, which you can call take-home exams if you like. Each of these will count for 15\% of the grade. The remaining 10\% of the grade will be given on the basis of constructive class participation.

I expect everyone to come to class and be actively engaged. I am confident that you will find it easier to master the course material by hearing it presented and also by asking questions when you don’t understand something. So as to avoid possible distractions from Facebook, email, surfing the web, etc., you are not permitted to use a laptop, smartphone, iPad, or the like during class, except if they are needed for a class activity.

All students must do their write-ups of problem sets, midterm and final in electronic form. I encourage everyone to use \LaTeX, a mark-up language which greatly facilitates the typesetting of mathematics. Versions of \LaTeX\ are available for Mac OS X, Windows and Linux, as well as in the cloud, and I will provide points to sources on the course website.

\printbibliography

\end{document}